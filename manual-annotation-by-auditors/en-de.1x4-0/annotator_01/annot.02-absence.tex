\documentclass[10pt]{article}

%\usepackage{times}
\usepackage{fancyhdr}

\pagestyle{fancy}
\fancyhf{}
\rhead{\textbf{Code: absence}}

\begin{document}

\subsection*{Source:}

The establishment of a specialised tax office for the administration of large tax entities is planned for 2012.
In Germany, a similar system does not yet exist which makes it more difficult to manage and analyse VAT control in general.
Germany is still more focused on the selection of respective individual cases especially based on data from VAT returns and recapitulative statements.
The analysis of those data is supported by different risk management components which are going to be enhanced.


3. Audit procedure



3.1 Audit procedure in the Czech Republic

The audit was performed in the period from May 2009 to February 2010 by Division II - Department of State Budget Incomes and by the territorial departments IX Pilsen, X České Budějovice, XIII Brno and XV Ostrava.
The audited entities were: the Ministry of Finance of the Czech Republic (hereinafter ”Czech Ministry of Finance”) and 25 tax offices: tax office Brno II, tax office Brno III, tax office Brno IV, tax office in Břeclav, tax office in Domažlice, tax office in Hodonín, tax office in Hradec Králové, tax office in Cheb, tax office in Jablonec nad Nisou, tax office in Karlovy Vary, tax office in Kralupy nad Vltavou, tax office in Kraslice, tax office Ostrava I, tax office Ostrava III, tax office in Písek, tax office in Pilsen, tax office for Prague 1, tax office for Prague 4, tax office for Prague 8, tax office in Rakovník, tax office in Sokolov, tax office in Tábor, tax office in Vysoké Mýto, tax office in Zlín and tax office in Žamberk.
The board of the Czech SAI approved the audit conclusions on 27 April 2010.


3.2 Audit procedure in Germany

The audit was conducted from February 2009 to February 2010 by the audit unit for indirect taxes and staff of the assigned regional audit offices in Berlin, Frankfurt/Main and Munich.
After selecting the case examples a number of federal states were notified of the audit mission.
Apart from interviews at the Federal Ministry of Finance, the audit took place in the following federal states: • Bavaria, • Hamburg, • Hesse, • North Rhine-Westphalia, • Saxony.


\pagebreak

\subsection*{Translation:}

Die Einrichtung eines spezialisierten Finanzamtes für die Verwaltung großer Steuerbehörden ist für 2012 geplant.
In Deutschland gibt es noch kein ähnliches System, das die Steuerung und Analyse der Mehrwertsteuerkontrolle im Allgemeinen erschwert.
Deutschland konzentriert sich nach wie vor stärker auf die Auswahl der jeweiligen Einzelfälle, insbesondere aufgrund von Daten aus Mehrwertsteuererklärungen und rekapitulativen Aussagen.
Die Analyse dieser Daten wird durch verschiedene Risikomanagement-Komponenten unterstützt, die verbessert werden sollen.


3. Prüfungsverfahren



3.1 Prüfungsverfahren in der Tschechischen Republik

Die Prüfung wurde in der Zeit von Mai 2009 bis Februar 2010 von der Abteilung II - Abteilung für Staatshaushaltseinkommen und von den Gebietskörperschaften IX Pilsen, X. Budweis, X. III. Brünn und X. V. Ostrava durchgeführt.
Die geprüften Stellen waren: das Finanzministerium der Tschechischen Republik (nachfolgend „Tschechisches Finanzministerium“) und 25 Finanzämter: Finanzamt Brünn II, Finanzamt Brünn III, Finanzamt Brünn IV, Finanzamt in Böhmen, Finanzamt in Domaclice, Finanzamt in Hodonín, Finanzamt in Hradec Králové, Finanzamt in Cheb, Finanzamt in Jablonec nad Nisou, Finanzamt in Karlovy Vary, Finanzamt in Prag, Steueramt in Prag, Steueramt in Prag, Steueramt in Prag, Finanzamt in Prag, Steueramt in Prag, Steueramt in Prag, Finanzamt in der Stadt in der Stadt in der Stadt, Finanzamt in der Stadt, Finanzamt in der Stadt Prag, Finanzamt in der Stadt in der Stadt in der Stadt in der Stadt in der Stadt Prag, Finanzamt in der Stadt.
Der Verwaltungsrat der Tschechischen ORKB billigte die Prüfungsergebnisse am 27. April 2010.


3.2 Prüfungsverfahren in Deutschland

Die Prüfung wurde von Februar 2009 bis Februar 2010 von der Wirtschaftsprüfungsstelle für indirekte Steuern und Mitarbeiter der zugeordneten Landesprüfungsämter in Berlin, Frankfurt am Main und München durchgeführt.
Nach Auswahl der Fallbeispiele wurden mehrere Bundesländer über die Prüfungsmission informiert.
Neben Interviews im Bundesfinanzministerium fanden die Prüfungen in folgenden Bundesländern statt: Bayern, Hamburg, Hessen, Nordrhein-Westfalen, Sachsen.


\end{document}
