\documentclass[10pt]{article}

%\usepackage{times}
\usepackage{fancyhdr}

\pagestyle{fancy}
\fancyhf{}
\rhead{\textbf{Code: health}}

\begin{document}

\subsection*{Source:}



Furthermore, Germany has committed itself to providing other EC Member States with the following information through a structured automatic exchange:

a. The supply of new means of transport to entrepreneurs without using their VAT ID, b. supplies by distance or catalogue selling to other Member States which are not taxable in Germany, c. significant differences between the intra-Community acquisitions declared by a taxpayer and the VIES data submitted by other Member States, d. intra-Community supplies of goods which the German taxpayer did not report in RS, because the supplies were treated as domestic transactions, e. potential “missing trader” which made intra-Community transactions independent of the validity of their VAT ID and f. entrepreneurs who supplied goods to potential “missing traders” or recipients who misused their VAT ID.


Comparison and evaluation

Sending of RFI or spontaneous information is initiated both in the CR and Germany by competent local tax offices that send filed SCAC forms to the CLO.
While the Czech CLO uses bilingual SCAC forms (CzechEnglish) and translates dispatched information into English or German language, respectively, German tax offices comply with the instruction, according to which they have to write RFI or spontaneous information exclusively in German on a single-language form.
The German CLO does not provide a translation of German texts.
When reviewing RFIs transmitted between the CR and Germany in the period from 1 May 2004 to 31 December 2005, the SAO found that the Czech tax administration settled 74,5 \% of received RFIs in a given three months deadline, while RFIs sent to Germany were settled by the German tax administration in the given deadline in 40,6 \% cases only.
Although the German CLO refers the tax offices to the 3-month-deadline it does not monitor the elapsed time limits.
The outcome of the missing reminders is the high quantity of delayed answers to RFIs.
Because of these problems the BRH recommended a steady reminding exercise of the CLO.
The situation has already improved.
According to our latest information the German CLO now reminds the Tos tax offices of answering the incompleted requests for information.


VAT risk management



Risk management in the CR

In the CR, several units at various organisational levels of the tax administration are concerned with VAT risk management.


\pagebreak

\subsection*{Translation:}



Darüber hinaus hat sich Deutschland verpflichtet, anderen EU-Mitgliedstaaten durch einen strukturierten automatischen Austausch folgende Informationen zur Verfügung zu stellen:

a. Die Lieferung neuer Transportmittel an Unternehmer ohne Verwendung ihrer Mehrwertsteuer-Identifikation, b. Fern- oder Kataloglieferungen an andere Mitgliedstaaten, die in Deutschland nicht steuerpflichtig sind, c. erhebliche Unterschiede zwischen den innergemeinschaftlichen Übernahmen, die von einem Steuerzahler angemeldet wurden, und den von anderen Mitgliedstaaten vorgelegten VIES-Daten, d. h. innergemeinschaftliche Lieferungen von Gütern, die der deutsche Steuerzahler nicht in der RS gemeldet hat, weil die Lieferungen als Inlandsgeschäfte behandelt wurden, e.


Vergleich und Bewertung

Das Senden von RFI oder spontanen Informationen wird sowohl in der Tschechischen Republik als auch in Deutschland von den zuständigen lokalen Finanzämtern initiiert, die eingereichte SCAC-Formulare an die CLO senden.
Während die tschechische CLO zweisprachige SCAC-Formulare (Tschechisch-Englisch) verwendet und versandte Informationen in englische bzw. deutsche Sprache übersetzt, erfüllen die deutschen Finanzämter die Anweisung, wonach sie RFI oder spontane Informationen ausschließlich in deutscher Sprache in einer einsprachigen Form schreiben müssen.
Die deutsche CLO stellt keine Übersetzung deutscher Texte zur Verfügung.
Bei der Überprüfung der zwischen der Tschechischen Republik und Deutschland im Zeitraum vom 1. Mai 2004 bis zum 31. Dezember 2005 übermittelten RFI stellte die SAO fest, dass die tschechische Steuerverwaltung 74,5 \% der erhaltenen RFI innerhalb einer Frist von drei Monaten beglichen hatte, während die nach Deutschland entsandten RFI von der deutschen Steuerverwaltung innerhalb der vorgegebenen Frist nur in 40,6 \% Fällen beglichen wurden.
Obwohl die deutsche CLO die Finanzämter auf den 3-Monats-Deadline verweist, überwacht sie die verstrichenen Fristen nicht.
Das Ergebnis der fehlenden Mahnungen ist die hohe Menge an verspäteten Antworten auf RFIs.
Wegen dieser Probleme empfahl der BRH eine stetige Erinnerungsübung an die CLO.
Die Situation hat sich bereits verbessert.
Nach unseren neuesten Informationen erinnert die deutsche CLO nun die Tos-Finanzämter an die Beantwortung der unvollständigen Auskunftsersuchen.


MwSt-Risikomanagement



Risikomanagement in der Tschechischen Republik

In der CR beschäftigen sich mehrere Einheiten auf verschiedenen Organisationsebenen der Steuerverwaltung mit dem MwSt-Risikomanagement.


\end{document}
