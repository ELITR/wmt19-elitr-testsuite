\documentclass[10pt]{article}

%\usepackage{times}
\usepackage{fancyhdr}

\pagestyle{fancy}
\fancyhf{}
\rhead{\textbf{Code: artist}}

\begin{document}

\subsection*{Source:}

To review a non-resident taxpayer’s VAT return or the fulfilment of his obligations, the BRH considered all these means of administrative cooperation as too labour-intensive compared to the small amounts of VAT that have to be paid for taxable supplies of transport of passengers.
Especially an audit of a non-resident taxpayer’s accounting leads to disproportional costs for both taxpayers and tax administration.
The latter has to audit the foreign accounting that does not only mean to understand the foreign language but to also know the foreign accounting standards.
So this audit is likely to take more time than an audit of a resident taxpayer, yet the audit is restricted to those passenger transports, that take place in Germany.
On the other hand, the non-resident taxpayer has not only to accept the audit rights of his national tax administration but also audits of other Member States wherever he transports passengers.
Moreover, for each Member State in that only one supply of transport of passengers takes place, taxpayers have to fulfil any duties of taxation in all of these Member States, respectively in all different languages and using all tax forms of these EC Member States.
So the actual taxation of crossborder transport of passengers causes unreasonable bureaucratic costs.


Comparison and evaluation

The legal situation in both countries is different.
In the CR, the transport of passengers between the individual Member States of the EC and also between the Member States and third countries is exempt from VAT with the right for tax deduction, on the basis of the Act on Accession of the CR.
In Germany, the part of the transport, that takes place within the German territory, is a taxable transaction.
If a non-resident taxpayer provides such transport for the first time, he is obliged to announce it to a competent tax office in advance.
The tax office registers taxpayers and taxpayers have to declare the taxable portion of transport in a tax return in Germany.
Applying the law in Germany is very time-consuming.
At least three of four Czech taxpayers registered in the tax office Chemnitz-Süd do their business in the transport of passengers.
Compared to the efforts required tax revenue is rather low.


\pagebreak

\subsection*{Translation:}

Um die Mehrwertsteuererklärung eines gebietsfremden Steuerzahlers oder die Erfüllung seiner Verpflichtungen zu überprüfen, betrachtete der BRH all diese Mittel der administrativen Zusammenarbeit als zu arbeitsintensiv im Vergleich zu den geringen Mehrwertsteuerbeträgen, die für steuerpflichtige Transporte von Fahrgästen gezahlt werden müssen.
Vor allem eine Prüfung der Rechnungslegung eines gebietsfremden Steuerzahlers führt zu unverhältnismäßigen Kosten sowohl für die Steuerzahler als auch für die Steuerverwaltung.
Letztere muss die Auslandsbuchhaltung prüfen, die nicht nur bedeutet, die Fremdsprache zu verstehen, sondern auch die ausländischen Rechnungslegungsstandards zu kennen.
Diese Prüfung dürfte also mehr Zeit in Anspruch nehmen als eine Prüfung eines ansässigen Steuerzahlers, doch die Prüfung beschränkt sich auf jene Personentransporte, die in Deutschland stattfinden.
Andererseits muss der gebietsfremde Steuerzahler nicht nur die Prüfungsrechte seiner nationalen Steuerverwaltung akzeptieren, sondern auch die Prüfungen anderer Mitgliedstaaten, wo immer er Passagiere befördert.
Darüber hinaus müssen die Steuerzahler für jeden Mitgliedstaat, in dem nur eine einzige Beförderung von Fahrgästen stattfindet, in allen diesen Mitgliedstaaten, bzw. in allen verschiedenen Sprachen und unter Verwendung aller Steuerformen dieser EU-Mitgliedstaaten, Steuerpflichten erfüllen.
Die tatsächliche Besteuerung des grenzüberschreitenden Personenverkehrs verursacht also unangemessene bürokratische Kosten.


Vergleich und Bewertung

Die Rechtslage in beiden Ländern ist eine andere.
In der Tschechischen Republik ist der Personenverkehr zwischen den einzelnen Mitgliedstaaten der Europäischen Gemeinschaft und auch zwischen den Mitgliedstaaten und Drittstaaten auf der Grundlage des Gesetzes über den Beitritt der Tschechischen Republik von der Mehrwertsteuer befreit, mit dem Recht auf Steuerabzug.
In Deutschland ist der Teil des Transports, der innerhalb des deutschen Staatsgebiets stattfindet, eine steuerpflichtige Transaktion.
Stellt ein gebietsfremder Steuerzahler erstmals einen solchen Transport zur Verfügung, ist er verpflichtet, ihn vorab bei einem zuständigen Finanzamt anzukündigen.
Das Finanzamt registriert Steuerzahler und Steuerzahler müssen den steuerpflichtigen Teil des Transports in einer Steuererklärung in Deutschland angeben.
Die Anwendung des Gesetzes in Deutschland ist sehr zeitaufwendig.
Mindestens drei von vier tschechischen Steuerzahlern, die im Finanzamt Chemnitz-Süd registriert sind, erledigen ihre Geschäfte im Personenverkehr.
Im Vergleich zu den erforderlichen Anstrengungen sind die Steuereinnahmen eher gering.


\end{document}
