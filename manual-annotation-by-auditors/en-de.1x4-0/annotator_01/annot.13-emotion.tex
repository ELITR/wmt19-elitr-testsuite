\documentclass[10pt]{article}

%\usepackage{times}
\usepackage{fancyhdr}

\pagestyle{fancy}
\fancyhf{}
\rhead{\textbf{Code: emotion}}

\begin{document}

\subsection*{Source:}

Through an inquiry at the competent local tax administrator in the CR, the SAO found that, on the basis of an investigation by the tax administrator, the Czech taxpayer submitted additional VAT returns for the individual taxable periods in 2004, in which he increased the value of the goods acquired from other Member States and simultaneously made claims for deduction of VAT on acquisition of goods from other Member States.
However, the Czech tax administrator could not unambiguously determine from these additional VAT returns whether they also include the value of supplies from the above-mentioned German supplier.
On the basis of a request from the SAO, the BRH found that the German supplier was registered for VAT in September 2003.
However, because he did not request the allocation of a VAT ID on registration, he was allocated only an identification number for transactions in Germany.
Although it is not usually possible to allocate a taxpayer a VAT ID retroactively in Germany, in this case, in 2006 the German tax official assigned to this German taxpayer a VAT ID retroactively as of 1 May 2004.
The German CLO allocates a VAT ID retroactively if the German entrepreneur did not cause the delay of the allocation; for example if the CLO caused the delay after the taxpayer’s application of a VAT ID.
In this case the CLO allocates VAT ID retroactively because the entrepreneur does not provide an own valid VAT ID for an intra-Community supply according to § 6 UStG.
Vice versa a German recipient does not provide a valid VAT ID of his supplier to declare an intra-Community acquisition.
Because of these different processes between the Member States the German CLO allocated VAT ID retroactively in this special case.
2.
As part of international exchange of information, the Czech tax administrator requested review of the VAT registration of the German taxpayer.
VIES gives a registration date of 5 March 2005, while the date 1 October 2004 is given in the certificate of registration of the German taxpayer issued by the competent local German tax office.
The German authorities stated in the answer sent from Germany that the German tax entity was registered on 1 October 2004, but that, for organisational reasons, he was allocated a VAT ID in Germany only on 16 February 2005.


\pagebreak

\subsection*{Translation:}

Durch eine Anfrage beim zuständigen lokalen Steuerverwalter in der Tschechischen Republik stellte der ORKB fest, dass der tschechische Steuerpflichtige auf der Grundlage einer Untersuchung des Steuerverwalters zusätzliche Mehrwertsteuererklärungen für die einzelnen steuerbaren Zeiträume im Jahr 2004 abgegeben hatte, in denen er den Wert erhöhte der aus anderen Mitgliedstaaten erworbenen Waren und gleichzeitig geltend gemachte Vorsteuerabzug beim Erwerb von Waren aus anderen Mitgliedstaaten.
Der tschechische Steuerverwalter konnte jedoch aus diesen zusätzlichen Mehrwertsteuererklärungen nicht eindeutig bestimmen, ob er auch den Wert der Lieferungen des oben genannten deutschen Lieferanten einschließt.
Auf Ersuchen der ORKB stellte die BRH fest, dass der deutsche Lieferant im September 2003 zur Mehrwertsteuer angemeldet wurde.
Da er jedoch bei der Registrierung keine Umsatzsteuer-Identifikationsnummer beantragt hatte, wurde ihm für Transaktionen in Deutschland nur eine Identifikationsnummer zugewiesen.
Obwohl es in der Regel nicht möglich ist, einem Steuerpflichtigen in Deutschland rückwirkend eine Umsatzsteueridentifikationsnummer zuzuweisen, hat der deutsche Steuerbeamte diesem Steuerpflichtigen 2006 rückwirkend zum 1. Mai 2004 eine Mehrwertsteueridentifikationsnummer zugewiesen.
Das deutsche CLO weist rückwirkend eine Umsatzsteuer-ID zu, wenn der deutsche Unternehmer die Verzögerung der Zuteilung nicht verursacht hat; Zum Beispiel, wenn der CLO die Verzögerung verursacht hat, nachdem der Steuerpflichtige eine Umsatzsteuer-ID beantragt hat.
In diesem Fall weist das CLO die Umsatzsteuer-ID rückwirkend zu, da der Unternehmer für eine innergemeinschaftliche Lieferung gemäß § 6 UStG keine eigene gültige Umsatzsteuer-ID angibt.
Umgekehrt gibt ein deutscher Empfänger keine gültige Umsatzsteueridentifikationsnummer seines Lieferanten an, um einen innergemeinschaftlichen Erwerb zu erklären.
Aufgrund dieser unterschiedlichen Verfahren zwischen den Mitgliedstaaten hat die deutsche CLO in diesem speziellen Fall die Umsatzsteuer-ID rückwirkend zugewiesen.


2

Im Rahmen des internationalen Informationsaustauschs beantragte der tschechische Steuerverwalter eine Überprüfung der Umsatzsteueranmeldung des deutschen Steuerpflichtigen.
Das MIAS gibt ein Registrierungsdatum vom 5. März 2005 an. Das Datum des 1. Oktober 2004 ist in der vom zuständigen deutschen Steueramt ausgestellten Registrierungsurkunde des deutschen Steuerpflichtigen angegeben.
Die deutschen Behörden erklärten in der von Deutschland übermittelten Antwort, dass die deutsche Steuerbehörde am 1. Oktober 2004 registriert wurde, dass er aus organisatorischen Gründen jedoch nur am 16. Februar 2005 in Deutschland eine Umsatzsteuer-Identifikationsnummer erhalten hatte.


\end{document}
