\documentclass[10pt]{article}

%\usepackage{times}
\usepackage{fancyhdr}

\pagestyle{fancy}
\fancyhf{}
\rhead{\textbf{Code: preview}}

\begin{document}

\subsection*{Source:}

Such notification should be sent no later than within three months of the date of acknowledgement of the receipt of the request.
Upon receipt of the request, the Czech Ministry of Finance shall verify if the request meets all legal requirements. After that, the Czech Ministry of Finance shall transfer the request through a financial directorate to the competent local tax office for processing.
In the event the request for recovery is accepted, the tax administrator shall i. a. identify the debtor's assets, e. g. through the extracts from the Cadastre of Real Estate, identification of bank accounts, registrations of motor vehicles with the Traffic Inspectorate etc.
When six months have elapsed from the date of the request receipt acknowledgement, the Czech Ministry of Finance should notify the applicant authority of the outcome of the investigation that has been undertaken in order to obtain the requested information.
The applicant authority should also be informed in the event that the tax entities in question cannot be contacted and the claim can therefore not be recovered.
Any amount successfully recovered by the tax administrator shall be transferred to the applicant authority within one month from the date on which such claim was recovered.
The Czech tax administrator making efforts to recover foreign claims may use an option to allow a moratorium on the payment of the claim or its payment by instalments, providing the applicant state gives its consent to that.
The interest or penalty charge is assessed to the debtor for the recovery period in accordance with Act No. 337/1992 Coll., on administration of taxes and fees.
Such interest or penal charges shall pertain to the state that has requested the recovery.
Concerning received requests for recovery, the Czech Ministry of Finance did not submit the requests to the competent tax administrators in time, particularly during 2006 and 2007.
A similar situation was identified with respect to the requests for international assistance for the recovery of the claims received from the tax administrators in the Czech Republic.
The audit revealed that the Czech Ministry of Finance did not proceed in accordance with the Act on the international assistance for the recovery of claims because it failed to notify a competent authority of the Member State of the progress made with the recovery within the six months time period from the day when the receipt of the request was acknowledged.


\pagebreak

\subsection*{Translation:}

Eine solche Mitteilung sollte spätestens innerhalb von drei Monaten nach dem Tag der Bestätigung des Eingangs des Antrags versandt werden.
Nach Eingang des Antrags prüft das tschechische Finanzministerium, ob der Antrag allen rechtlichen Anforderungen entspricht, danach überträgt das tschechische Finanzministerium den Antrag über eine Finanzdirektion an das zuständige örtliche Finanzamt zur Bearbeitung.
Für den Fall, dass der Antrag auf Rückforderung angenommen wird, hat der Steuerverwalter u.a. das Vermögen des Schuldners zu identifizieren, z.B. durch Auszüge aus dem Immobilienkataster, Identifizierung von Bankkonten, Zulassungen von Kraftfahrzeugen bei der Verkehrsinspektion usw.
Nach Ablauf von sechs Monaten ab dem Datum der Empfangsbestätigung des Antrags sollte das tschechische Finanzministerium der antragstellenden Behörde das Ergebnis der Untersuchung mitteilen, die durchgeführt wurde, um die angeforderten Informationen zu erhalten.
Die antragstellende Behörde sollte auch darüber informiert werden, dass die betreffenden Steuerbehörden nicht kontaktiert werden können und die Forderung daher nicht zurückgefordert werden kann.
Jeder vom Steuerverwalter erfolgreich zurückgeforderte Betrag wird innerhalb eines Monats ab dem Tag, an dem diese Forderung zurückgefordert wurde, an die antragstellende Behörde überwiesen.
Der tschechische Steuerverwalter, der sich um die Rückforderung ausländischer Forderungen bemüht, kann eine Option nutzen, um ein Moratorium für die Zahlung der Forderung oder deren Zahlung in Raten zuzulassen, sofern der antragstellende Staat seine Zustimmung dazu erteilt.
Die Zins- oder Strafgebühr wird dem Schuldner für die Rückforderungsfrist gemäß dem Gesetz Nr. 337/1992 Slg. über die Verwaltung von Steuern und Gebühren berechnet.
Solche Zinsen oder Strafgebühren gelten für den Staat, der die Rückforderung beantragt hat.
Was die eingegangenen Anträge auf Rückforderung betrifft, so hat das tschechische Finanzministerium die Anträge nicht rechtzeitig bei den zuständigen Steuerverwaltern eingereicht, insbesondere nicht in den Jahren 2006 und 2007.
Eine ähnliche Situation wurde im Hinblick auf die Anträge auf internationale Hilfe bei der Beitreibung der von den Steuerverwaltern in der Tschechischen Republik eingegangenen Forderungen festgestellt.
Die Prüfung ergab, dass das tschechische Finanzministerium nicht im Einklang mit dem Gesetz über die internationale Unterstützung bei der Beitreibung von Forderungen vorgegangen ist, weil es es versäumt hat, einer zuständigen Behörde des Mitgliedstaats die Fortschritte bei der Beitreibung innerhalb von sechs Monaten ab dem Tag, an dem der Eingang des Antrags anerkannt wurde, mitzuteilen.


\end{document}
