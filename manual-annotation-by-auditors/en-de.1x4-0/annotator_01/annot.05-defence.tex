\documentclass[10pt]{article}

%\usepackage{times}
\usepackage{fancyhdr}

\pagestyle{fancy}
\fancyhf{}
\rhead{\textbf{Code: defence}}

\begin{document}

\subsection*{Source:}

Bruchsal, Ettlingen, Freudenstadt, Karlsruhe-Stadt, Leonberg, Mühlacker, Singen, Tuttlingen, Cham, Fürstenfeldbruck, München-Körperschaften, München II, IV und München V, Starnberg, Hamburg-Altona, Hamburg-Hansa, Hamburg-Mitte, Hamburg-Nord und Hamburg für Großunternehmen, Hannover-Nord, Winsen-Luhe, Bonn-Innenstadt, Köln-Mitte, Köln-Porz, Köln-Süd, Sankt Augustin, Pirmasens, Bautzen, Chemnitz-Süd, Dresden II, Freiberg, Leipzig I und Zschopau.
Additionally, discussions were held with the tax offices Augsburg-Stadt, Aurich, Düsseldorf-Altstadt, Eisleben, Kelheim und Verden.
The audit findings were reported to the MoF in May 2007.


Cooperation of the two audit institutions

As part of preparing parallel audits, several meetings of representatives of the two audit institutions were held in the period prior to their commencement.
Participants in these meetings became acquainted with the organisation of the tax administrations, legislation on VAT and the VAT administration system in the two countries.
The subject matter of the testing phase was agreed at the meeting that was held in May 2006.
The results of the testing phase were evaluated at the next joint meeting in August 2006.


At this meeting, also a timetable was agreed for further procedure, and the following subjects for parallel audits were selected:

1. Registration of VAT payers.
2.
International passenger bus transportation – legislation in the CR and Germany.
3. International exchange of information, in particular the processing of requests for information and furthermore how the tax offices reviewed selected problem cases mentioned in the requests for information, and in the spontaneous information supplied between the CR and Germany.
4. VAT returns.
5.
Recapitulative statements.
6.
Statistical information on supply and acquisition of goods between the CR and the Germany.
7. VAT risk management.
At a meeting of the SAO and the BRH, held in December 2006, the two parties became acquainted with the audit procedure in the CR and Germany and discussed results achieved in the parallel audits.
A report on the results of the parallel audits of the administration of VAT between the CR and Germany was elaborated at the working meeting in July 2007.


4. Administration of value added tax in the CR and Germany



Registrace plátců k DPH



A) Taxpayers with their offi cial seat, place of business or commercial premises in the CR



Pursuant to Act No. 337/1992 Coll., on administration of taxes and fees (hereinafter “Act No. 337/1992 Coll.”) the term tax entity refers to a taxpayer, a taxable person and the legal successor of a natural or legal person that is a tax entity definedby law. 



\pagebreak

\subsection*{Translation:}

Bruchsal, Ettlingen, Freudenstadt, Karlsruhe-Stadt, Leonberg, Mühlacker, Singen, Tuttlingen, Cham, Fürstenfeldbruck, München-Körperschaften, München II, IV und München V., Starnberg, Hamburg-Altona, Hamburg-Hamburg, Nord und Hamburg für Großunternehmen, Hannover-Nord, Winsen-Luhe, Bonn-Innenstadt, Köln-Mitte, Köln-Porz, Köln-Süd, Sankt Augustin, Pirmasens, Bautzen, Chemnitz-Süd, Dresden II, Freiberg, Leipzig I und Zschopau .
Darüber hinaus wurden Gespräche mit den Finanzämtern Augsburg-Stadt, Aurich, Düsseldorf-Altstadt, Eisleben, Kelheim und Verden geführt.
Die Prüfungsfeststellungen wurden dem MoF im Mai 2007 übermittelt.


Zusammenarbeit der beiden Rechnungskontrollbehörden

Im Rahmen der Vorbereitung paralleler Prüfungen fanden in der Zeit vor ihrem Beginn mehrere Sitzungen von Vertretern der beiden Rechnungskontrollbehörden statt.
Die Teilnehmer an diesen Treffen lernten die Organisation der Steuerverwaltungen, die Mehrwertsteuergesetzgebung und das MwSt-Verwaltungssystem in beiden Ländern kennen.
Das Thema der Testphase wurde auf der Sitzung im Mai 2006 vereinbart.
Die Ergebnisse der Testphase wurden auf der nächsten gemeinsamen Sitzung im August 2006 ausgewertet.


Bei diesem Treffen wurde auch ein Zeitplan für das weitere Verfahren vereinbart und die folgenden Themen für parallele Prüfungen ausgewählt:



1. Registrierung der Mehrwertsteuerzahler



2

Internationaler Personenbusverkehr - Gesetzgebung in der Tschechischen Republik und in Deutschland.
3. Internationaler Informationsaustausch, insbesondere die Bearbeitung von Auskunftsersuchen, und wie die Finanzämter ausgewählte, in den Auskunftsersuchen genannte Problemfälle und in den spontanen Informationen zwischen der CR und Deutschland geprüft haben.
4. Mehrwertsteuererklärungen.


5

Zusammenfassende Aussagen.


6



Statistische Informationen zu Lieferung und Erwerb von Waren zwischen der Tschechischen Republik und Deutschland

7. MwSt-Risikomanagement.
Bei einem Treffen von ORKB und BRH im Dezember 2006 lernten die beiden Parteien das Prüfungsverfahren in der Tschechischen Republik und in Deutschland kennen und diskutierten die Ergebnisse der parallelen Prüfungen.
Auf der Arbeitssitzung im Juli 2007 wurde ein Bericht über die Ergebnisse der parallelen Prüfungen der MwSt-Verwaltung zwischen der Tschechischen Republik und Deutschland erstellt.


4. Verwaltung der Mehrwertsteuer in der Tschechischen Republik und in Deutschland



Registrieren Sie sich für DPH



A) Steuerpflichtige mit Amtssitz, Geschäftssitz oder Geschäftsraum in der Tschechischen Republik

Gemäß Gesetz Nr. 337/1992 Slg. Bezieht sich der Begriff Steuerbehörde auf die Verwaltung von Steuern und Gebühren (nachstehend „Gesetz Nr. 337/1992 Slg.“) Auf einen Steuerpflichtigen, einen Steuerpflichtigen und den Rechtsnachfolger einer natürlichen oder juristischen Person juristische Person, bei der es sich um eine nach Gesetz bestimmte steuerliche Einheit handelt.


\end{document}
