\documentclass[10pt]{article}

%\usepackage{times}
\usepackage{fancyhdr}

\pagestyle{fancy}
\fancyhf{}
\rhead{\textbf{Code: anthrax}}

\begin{document}

\subsection*{Source:}

The reason was increasing of scope of delivery and changes of technical conditions.
In case of some social buildings construction the tender documentation showed mistakes which resulted in necessity of extra works (precise location missing, technical and constructional research insufficient, drawings missing).
Tender documentation for construction of a big library described the elements of interior only without specifying product quality requirements.
As a result, some sub suppliers provided some very expensive and above standard equipment.


Lots and criteria according to which the contract is to be awarded

In the Czech Republic, no obligation of dividing public works projects into partial lots or awarding partial contracts is in place.
Analysis of three big parts of a road circle construction proved that within their public procurement in parts (individually by selected installations), savings amounting up to 31.6 million EUR could have been achieved, out of total of achieved public contract prices amounting to 620.3 million EUR.
For example, in case of offer price for 6 individual construction objects as 6 separate parts of one public contract, the savings of 2.5 million EUR (i.e. 25\%) could have been achieved.
If the construction works of 20 bridges had been awarded as a separate part of one public contract, the savings amounting to 11.5 million EUR (i.e. 25\%) could have been achieved.
In case of road constructions, contracting authority imposed very extensive requirements for references as suitability criteria.
In tender documentation of open procedure for one part of a motorway construction, contracting authority requested, among others, such experience of a supplier, proving his realization of at least three road construction projects worth 28 million EUR each, finalized or put into operation within the last 5 years.
In the Czech Republic, there are not more than 9 suppliers complying with the above prerequisite.
Open public procurement for another part of a motorway construction included 6 separate construction works, where the contracting authority demanded minimum yearly construction volume of construction works worth € 480 million within the last 3 years.
Podle provedeného rozboru získalo dvě třetiny finančního objemu zakázek v oblasti pozemních komunikací (62,5 mld. Kč z 92,5 mld. Kč) v letech 2008 až 2009 pět firem buď samostatně, nebo v rámci sdružení.
In case of together 46 contract award procedures worth € 528 million, there were just two or even one only tenderer.


\pagebreak

\subsection*{Translation:}

Der Grund war der zunehmende Umfang der Lieferung und die Veränderung der technischen Bedingungen.
Bei einigen Sozialgebäuden zeigte die Ausschreibungsdokumentation Fehler, die zur Notwendigkeit von zusätzlichen Arbeiten führten (genaue Standortnachteile, technische und bauliche Untersuchungen unzureichend, fehlende Zeichnungen).
Die Ausschreibungsunterlagen für den Bau einer großen Bibliothek beschrieb nur die Elemente des Interieurs, ohne die Anforderungen an die Produktqualität anzugeben.
Infolgedessen lieferten einige Unterlieferanten einige sehr teure und über dem Standard liegende Geräte.


Viele Kriterien, nach denen der Auftrag vergeben werden soll

In der Tschechischen Republik besteht keine Verpflichtung, öffentliche Bauvorhaben in Teilgrundstücke aufzuteilen oder Teilaufträge zu vergeben.
Die Analyse von drei großen Teilen eines Kreisstraßenbaus ergab, dass bei der teilweisen Vergabe öffentlicher Aufträge (einzeln nach ausgewählten Anlagen) Einsparungen in Höhe von bis zu 31,6 Mio. EUR erzielt werden konnten, von insgesamt erzielten öffentlichen Auftragspreisen in Höhe von 620,3 Mio. EUR.
So hätten beispielsweise bei einem Angebotspreis für sechs einzelne Bauobjekte als 6 Einzelteile eines öffentlichen Auftrags Einsparungen in Höhe von 2,5 Mio. EUR (d. h. 25\%) erzielt werden können.
Hätte man die Bauarbeiten an 20 Brücken in einem separaten Teil eines öffentlichen Auftrags vergeben, wären Einsparungen in Höhe von 11,5 Mio. EUR (d. h. 25\%) möglich gewesen.
Im Falle des Straßenbaus hat der Auftraggeber sehr umfangreiche Anforderungen an Referenzen als Eignungskriterien gestellt.
Bei der Dokumentation des offenen Verfahrens für einen Teil des Autobahnbaus verlangte der Auftraggeber unter anderem die Erfahrung eines Lieferanten, der nachweisen konnte, dass er innerhalb der letzten 5 Jahre mindestens drei Straßenbauprojekte im Wert von jeweils 28 Mio. EUR realisiert, fertiggestellt oder in Betrieb genommen hat.
In der Tschechischen Republik gibt es nicht mehr als 9 Anbieter, die die oben genannte Voraussetzung erfüllen.
Die öffentliche Auftragsvergabe für einen anderen Teil des Autobahnbaus umfasste sechs separate Baumaßnahmen, bei denen der Auftraggeber innerhalb der letzten drei Jahre ein jährliches Mindestbauvolumen von 480 Mio. EUR verlangte.
Podle provedeného rozboru získalo dvě třetiny finančního objemu zakázek v oblasti pozemních komunikací (62,5 mld. Kč z 92,5 mld. Kč) v letech 2008 až 2009 pět firem buť samostatně, nebo v rámci sdružení.
Bei insgesamt 46 Auftragsvergabeverfahren im Wert von 528 Mio. EUR gab es nur zwei oder sogar nur einen Bieter.


\end{document}
