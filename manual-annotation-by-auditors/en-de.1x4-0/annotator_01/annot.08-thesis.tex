\documentclass[10pt]{article}

%\usepackage{times}
\usepackage{fancyhdr}

\pagestyle{fancy}
\fancyhf{}
\rhead{\textbf{Code: thesis}}

\begin{document}

\subsection*{Source:}

The audit period covered the years 2006 through 2008 including relevant information dating from the time preceding the audit up to the time of audit conclusion.


2. Summary of findings and evaluation of parallel audits

The follow-up audit of the administration of VAT in the Czech Republic and in Germany was carried out on the basis of an agreement between the two SAIs.
During previous parallel audits, suspicious cases of intra-Community transactions were detected.
Some of them merited further review.
On the basis of audit findings the two SAIs produced recommendations on VAT management.
The follow-up audit was conducted to evaluate the action taken in response to these recommendations and to review the suspicious cases selected.
The audit also covered other topics such as the selected cases of high risk intra-Community transactions, mutual assistance for the recovery of claims and the VAT audit of large tax entities.


When auditing VAT administration in selected areas in the two countries, auditors developed the following audit findings:



1. Recommendations from the previous audit



• Recommendation 1: Conditions for registration of taxpayers should be harmonised within the EU

By comparing the systems of registration for VAT in the two countries auditors found discrepancies in the previous parallel audits.
Since the different ways of registration caused problems, the SAIs recommended harmonising the conditions for registration of taxpayers within the EU.
This issue was taken up at EU level.
The draft of the amended Council (EC) Regulation No. 1798/2003 on administrative cooperation in the field of value added tax and repealing Regulation (EEC) No. 218/92 (hereinafter the “Council (EC) Regulation No. 1798/2003”) had proposed to expand the current databases within EU, including a database of the tax identification numbers of the VAT payers.
It also provided for common standards for registration and deregistration of taxpayers.
The SAIs expected Member States to agree on adequate standards supporting their decision to keep unreliable traders as early as possible out of the VAT system.
However, the latest version of this document being subject of political agreement among Member States within ECOFIN does no longer cover those issues mentioned before.
The two SAIs regret, that the idea of common minimum standards has been abandoned.
They still consider this idea relevant to combat VAT fraud preventively.


• Recommendation 2: Monthly submission of recapitulative statements



\pagebreak

\subsection*{Translation:}

Prüfungszeitraum waren die Jahre 2006 bis 2008 einschließlich maßgeblicher Informationen aus der Zeit vor dem Prüfungszeitraum bis zum Abschluss der Prüfung.


2. Zusammenfassung der Ergebnisse und Bewertung der parallelen Prüfung 

Die Rechnungshöfe der Tschechischen Republik und der Bundesrepublik Deutschland haben eine Kontrollprüfung zur Verwaltung der USt vereinbart.


Bei der vorherigen parallelen Prüfung hatten sie Verdachtsfälle mit innergemeinschaftlichen Umsätzen aufgedeckt, 

die eine nochmalige Prüfungzweckmäßig erschienen ließen.
Aufgrund der Prüfungsfeststellungen gaben die beiden Rechnungshöfe Empfehlungen zur Verwaltung der USt ab.
Ziel der Kontrollprüfung war es zu evaluieren, wie diese Empfehlungen umgesetzt wurden. Zudem sollten die ermittelten Verdachtsfälle untersucht werden.
Die Prüfung befasste sich auch mit anderen Sachverhalten. Hierzu gehörten ausgewählte Fälle risikoreicher innergemeinschaftlicher Umsätze, die gegenseitige Amtshilfe bei der Beitreibung von USt-Forderungen und die USt-Prüfung bei Großbetrieben.


Dabei haben die Rechnungshöfe festgestellt: 



1. Aufgrund der vorherigen Prüfung abgegebene Empfehlungen 



• Empfehlung 1: Die Regelungen für die Registrierung von Steuerpflichtigen sollten innerhalb der EU harmonisiert werden 

Bei der ersten Prüfung verglichen die Rechnungshöfe die Registrierungsverfahren zur USt in den beiden Staaten und stellten Unterschiede fest, die Probleme verursachten.
Die Rechnungshöfe empfahlen deshalb, die Registrierung der Steuerpflichtigen innerhalb der EU zu harmonisieren.
Dies wurde auch auf EU-Ebene aufgegriffen.
Der Entwurf für die Novellierung der Verordnung Nr. 1798/2003 des Rates sieht vor, die derzeitigen Datenbanken in der EU einschließlich einer Datenbank mit der Identifizierungsnummer der USt-Pflichtigen auszuweiten.
Der Entwurf sah auch gemeinsame Regelungen für die Registrierung der Steuerpflichtigen und ihre etwaige Löschung vor.
Die Rechnungshöfe erwarteten, dass die Mitgliedstaaten sich entsprechend ihrer Grundsatzentscheidung auf geeignete Regelungen einigen, um unzuverlässige Unternehmer so früh wie möglich aus dem USt-System zu entfernen.
Über die letzte Fassung des Entwurfs erzielten die Mitgliedstaaten im ECOFIN politisches Einvernehmen.
Die genannten gemeinsamen Regelungen sind darin nicht mehr enthalten. Die beiden Rechnungshöfe bedauern das.
Sie halten solche Regelungen nach wie vor für wichtig, um Umsatzsteuerbetrug vorbeugend zu bekämpfen.
• Empfehlung 2: Monatliche Vorlage der Zusammenfassenden Meldungen Als Ergebnis der ersten Prüfung empfahlen die Rechnungshöfe den Mitgliedstaaten, dass die Zusammenfassenden Meldungen (ZM) künftig monatlich eingereicht werden sollten.


\end{document}
