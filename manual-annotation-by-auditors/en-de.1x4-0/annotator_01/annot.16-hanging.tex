\documentclass[10pt]{article}

%\usepackage{times}
\usepackage{fancyhdr}

\pagestyle{fancy}
\fancyhf{}
\rhead{\textbf{Code: hanging}}

\begin{document}

\subsection*{Source:}

Cases of risky intra-Community transactions carried out by taxpayers from EU presented in EUROCANET were selected for the audit (see chapter 4.3).
The objectives of improved administrative cooperation and enhanced effectiveness of combating VAT fraud should be achieved also through establishing the EUROFISC network, meant to assist improving cooperation of the responsible authorities to a level enabling disclosure of fraud already at an early stage.
Later on, the network should become a resource for assessing risk exposures in intra-Community transactions.
ECOFIN approved establishment of the EUROFISC network already on 7 October 20089 and stipulated the key principles to govern the network at the same time.
A model for EUROFISC can be seen in EUROCANET, the system already operative for the swift and selective exchange of information developed by the Belgian tax authorities in cooperation with the other Member States and with support from the European Commission and OLAF.
The objective of the system is to detect risky business transactions and entities involved in organised VAT fraud.


4.1.2 Evaluation of the recommendations in Germany



The evaluation of the recommendations of the first report resulted in the following findings:



Recommendation 1



Conditions for registration of taxpayers should be harmonised within the EU

The tax administration in Germany established a well functioning system for the registration of taxpayers that is based on a detailed questionnaire.
The Federal Ministry of Finance brought this system to the attention of the Commission in October 2007.
The system itself is being improved continuously.
The data of the questionnaire are now being recorded electronically and will subsequently be available for research so that the data are part of risk management during registration.
The Commission made a proposal to amend Regulation 1798/2003. This proposal included common standards for the registration and deregistration of taxpayers.
The Member States needed further explanation and detailed elaboration of these issues.
The German SAI requested the Federal Ministry of Finance to make sure that Germany will not go below the standards achieved.


Recommendation 2



Monthly submission of recapitulative statements

On April 8, 2010 a law was adopted for the implementation of EU directives into national law.
The law includes provisions regarding the monthly submission of recapitulative statements for intra-Community supplies.
The provisions take effect as of 1 July 2010.


Recommendation 3



\pagebreak

\subsection*{Translation:}

Es wurden dazu von Steuerpflichtigen innerhalb der EU ausgeführte und in EUROCANET erfasste risikobehaftete innergemeinschaftliche Umsätze ausgewählt (siehe Kapitel 4.3).
Durch das Netzwerk EUROFISC soll die Amtshilfe in Steuersachen optimiert und der USt-Betrug wirksamer bekämpft werden. Die Zusammenarbeit der zuständigen Behörden soll verbessert werden, um Betrugsfälle möglichst frühzeitig zu erkennen.
Später könnte EUROFISC als Hilfsmittel genutzt werden, um das Risikopotenzial bei innergemeinschaftlichen Umsätzen abzuschätzen.
Der ECOFIN-Rat6 hat EUROFISC bereits am 7. Oktober 2008 gebilligt und gleichzeitig die Grundsätze für die Arbeitsweise des Systems verabschiedet.
Als Vorbild für EUROFISC dient EUROCANET, das von der belgischen Finanzverwaltung in Zusammenarbeit mit den anderen Mitgliedstaaten sowie mit Unterstützung der Europäischen Kommission und von OLAF7 entwickelt worden ist.
Zweck des Systems ist es, risikobehaftete Umsätze aufzudecken und Unternehmer, die am organisierten USt-Betrug beteiligt sind, zu ermitteln.


4.1.2 Umsetzung der Empfehlungen in Deutschland



Die Bewertung der Empfehlungen aus dem ersten Bericht führte zu folgenden Feststellungen:



Empfehlung 1



Harmonisierung der Voraussetzungen für die Registrierung von Steuerpflichtigen innerhalb der EU

Die deutsche Finanzverwaltung hat anhand eines detaillierten Fragebogens ein gut funktionierendes System für die Registrierung von Steuerpflichtigen entwickelt.
Das Bundesfinanzministerium hat dieses System der Kommission im Oktober 2007 vorgestellt.
Das System selbst wird ständig verbessert.
Die im Fragebogen angegebenen Daten werden jetzt elektronisch erfasst und stehen später für Nachforschungen zur Verfügung. Sie sind somit Teil des Risikomanagements bei der Registrierung.
Die Kommission hatte einen Vorschlag8 zur Änderung der Verordnung 1798/2003 gemacht. Dieser sah u. a. gemeinsame Standards für die Registrierung und Löschung von Steuerpflichtigen vor.
Bei den Mitgliedstaaten bestand hierzu noch Klärungs- und Präzisierungsbedarf.
Der Bundesrechnungshof hatte dem Bundesfinanzministerium empfohlen, nicht hinter die bereits erreichten Standards zurückzugehen.


Empfehlung 2



Monatliche Abgabe der ZM

Am 8. April 2010 wurde ein Gesetz9 zur Umsetzung der EU-Richtlinien in nationales Recht verabschiedet.


Es enthält u. a. Bestimmungen zur monatlichen Abgabe der ZM für innergemeinschaftliche Lieferungen

und tritt zum 1. Juli 2010 in Kraft.


Empfehlung 3



\end{document}
