\documentclass[10pt]{article}

%\usepackage{times}
\usepackage{fancyhdr}

\pagestyle{fancy}
\fancyhf{}
\rhead{\textbf{Code: silkie}}

\begin{document}

\subsection*{Source:}

Moreover the two SAIs are of the opinion that reliable statistics should disclose the annual number of requests as well as the nature of claims concerned (VAT, income tax etc.) and amounts of the claims recovered per Member State.
The two SAIs consider it useful that the Commission provides for a more detailed analysis of the statistics submitted by the Member States, because this may help the Member States to assess their own effectiveness in combating VAT fraud.
As the recovery rate can be calculated per Member State, the two SAIs consider it helpful that Member States with a low rate should increase their efforts to reach at least the average recovery rate of the EU.
If a Member State so desires, the Commission should take an active part in this process.
Concerning the requests for recovery received, the Czech Ministry of Finance submitted the requests to the competent tax administrators with delay in particular during 2006 and 2007.
A similar situation was identified with respect to the requests for international assistance for the recovery of the claims received from the tax administrators in the Czech Republic.
The audit revealed that the Czech Ministry of Finance did not proceed in accordance with the Act No. 191/2004 Coll. for the international assistance for the recovery of claims, because it failed to notify a competent authority of the Member State of the progress of the recovery efforts within the six months time period from the day of acknowledgement of receiving the request.


4.5 VAT audit of large tax entities



4.5.1 VAT audit of large tax entities in the Czech Republic



Všeobecné informace

Správa DPH u VDS byla zajišťována obdobně jako u ostatních daňových subjektů.
The procedure for the large tax entity audit is based on Act No. 337/1992 Coll., on administration of taxes and fees.
The Czech Ministry of Finance defined large tax entities as natural or legal persons founded or established for the purpose of doing business and whose net turnover for the taxation period declared in their individual or corporate income tax return amounted to CZK 2 bn and more.
Classification of a tax entity as large tax entity was carried out following its income tax return submission for the respective taxation period, and for 2003 for the first time.


\pagebreak

\subsection*{Translation:}

Darüber hinaus sind die beiden ORKB der Ansicht, dass verlässliche Statistiken die jährliche Anzahl der Anträge sowie die Art der betreffenden Anträge (Mehrwertsteuer, Einkommenssteuer usw.) und die Beträge der pro Mitgliedstaat eingezogenen Anträge enthalten sollten.
Die beiden ORKB halten es für sinnvoll, dass die Kommission eine detailliertere Analyse der von den Mitgliedstaaten vorgelegten Statistiken vorlegt, da dies den Mitgliedstaaten helfen kann, ihre eigene Wirksamkeit bei der Bekämpfung des MwSt-Betrugs zu bewerten.
Da die Wiederfindungsrate pro Mitgliedstaat berechnet werden kann, halten die beiden ORKBs es für hilfreich, dass Mitgliedstaaten mit einer niedrigen Quote ihre Anstrengungen verstärken sollten, um mindestens die durchschnittliche Wiederherstellungsrate der EU zu erreichen.
Wenn ein Mitgliedstaat dies wünscht, sollte sich die Kommission aktiv an diesem Prozess beteiligen.
In Bezug auf die eingegangenen Anträge auf Wiedereinziehung legte das tschechische Finanzministerium die Anträge den zuständigen Steuerverwaltern insbesondere in den Jahren 2006 und 2007 mit Verspätung vor.
Eine ähnliche Situation wurde bei den Anträgen auf internationale Unterstützung bei der Beitreibung der von den Steuerverwaltern in der Tschechischen Republik eingegangenen Forderungen festgestellt.


Die Prüfung ergab, dass das tschechische Finanzministerium nicht gemäß dem Gesetz Nr. 191/2004 Coll. für die internationale Unterstützung bei der Beitreibung von Forderungen, weil sie eine zuständige Behörde des Mitgliedstaats nicht innerhalb von sechs Monaten nach dem Tag des Eingangs des Antrags über den Fortschritt der Beitreibungsmaßnahmen unterrichtet hat



4.5 Mehrwertsteuerprüfung großer Steuereinheiten



4.5.1 Mehrwertsteuerprüfung großer Steuereinheiten in der Tschechischen Republik



Všeobecné informace

Správa DPH u VDS ist ein interessanter Ort, um die Umgebung zu erkunden.
Das Verfahren für die Prüfung großer Steuerunternehmen basiert auf dem Gesetz Nr. 337/1992 Sl.
Das tschechische Finanzministerium definierte große Steuereinheiten als natürliche oder juristische Personen, die zu Geschäftszwecken gegründet oder niedergelassen wurden und deren Nettoumsatz für den Steuerzeitraum, der in ihrer Einzel- oder Körperschaftsteuererklärung angegeben wurde, 2 Mrd. CZK und mehr betrug.
Die Einstufung einer Steuereinrichtung als große Steuereinrichtung erfolgte nach der Vorlage der Einkommensteuererklärung für den jeweiligen Steuerzeitraum und erstmals für 2003.


\end{document}
