\documentclass[10pt]{article}

%\usepackage{times}
\usepackage{fancyhdr}

\pagestyle{fancy}
\fancyhf{}
\rhead{\textbf{Code: regime}}

\begin{document}

\subsection*{Source:}

Pursuant to the Provisions of § 37 (1) of Act No. 337/1992 Coll., the tax office may repeatedly impose a fine of up to CZK 2,000,000 (EUR 67,150) on a person who does not meet his obligations of a non-monetary nature following from this or a special tax law or imposed in a decision pursuant to this law.
Automated tax information system, which the tax office uses for technical support for administration of taxes.
The SAO found that the tax offices mostly did not make use of the possibility of imposing a fine on taxpayers for failure to submit a VAT return.


Submitting and processing of VAT returns in Germany

In Germany we differentiate between preliminary monthly or quarterly VAT returns and the annual VAT return.
Those taxpayers who are obliged to submit preliminary VAT returns nevertheless have to submit an annual VAT return.
Even small entrepreneurs who are not obliged to pay VAT have to submit an annual VAT return.
Taxpayers generally have to submit quarterly VAT returns by the 10th day of the month following the quarter in which VAT arises (§ 18 (2) UStG).
However, if the VAT liability exceeded EUR 6,136 in the previous calendar year or in case of a newly established business the entrepreneur is obliged to submit monthly VAT returns by the 10th day following the respective month.
The taxpayer may also choose to submit monthly VAT returns if there was a refund in the preceding year, which exceeded EUR 6,136 (§ 18 (2a) UStG).
VAT returns need not be submitted if the annual VAT did not exceed EUR 512.
The corresponding VAT liability has to be paid by the 10th day of the month following the tax period in which the VAT arose.
A refund may lead to a more detailed investigation by the tax authorities.
Deadlines (by law) for the processing of VAT returns and for the refund of excessive VAT do not exist.
To achieve the refund in time, it is possible to offer a security for the repayable amount.
It is a basic principle to process VAT returns and refunds promptly.
Under German VAT law this time limit may be extended (§ 18 (6) UStG).
Such extension has to be applied for separately.


\pagebreak

\subsection*{Translation:}

Gemäß den Bestimmungen von § 37 (1) des Gesetzes Nr. 337/1992 kann das FA wiederholt eine Geldstrafe von bis zu 2 Millionen CZK gegen Personen verhängen, die ihren nichtmonetären Pflichten aus diesem oder einem besonderen Steuergesetz oder einer Entscheidung gemäß diesem Gesetz nicht nachkommen.
Automated tax information system – Automatisiertes Steuerinformationssystem, welches die tschechischen Finanzämter zur technischen Unterstützung der Steuererhebung einsetzen.
Der NKÚ stellte bei seiner Prüfung fest, dass die FÄ von der Möglichkeit, bei Nichtabgabe der Umsatzsteuererklärung eine Geldstrafe gegen den betreffenden Steuerpflichtigen zu verhängen, selten Gebrauch machten.


Abgabe und Bearbeitung von Umsatzsteuererklärungen in Deutschland

In Deutschland wird zwischen monatlichen oder vierteljährlichen Umsatzsteuer-Voranmeldungen und der jährlichen Umsatzsteuererklärung unterschieden.
Diejenigen Steuerpflichtigen, die zur Abgabe von Umsatzsteuer-Voranmeldungen verpflichtet sind, müssen trotzdem eine jährliche Umsatzsteuererklärung abgeben.
Selbst Kleinunternehmer, die nicht zur Zahlung der Umsatzsteuer verpflichtet sind, müssen eine jährliche Umsatzsteuererklärung abgeben.
Die Steuerpflichtigen sind grundsätzlich verpflichtet, bis zum 10. Tag nach Ablauf des Kalenderquartals, in dem die Umsatzsteuer entsteht, eine Umsatzsteuer-Voranmeldung abzugeben (§ 18 (2) UStG).
Betrug die Umsatzsteuerschuld jedoch im Vorjahr mehr als 6.136 € oder handelt es sich um ein neugegründetes Unternehmen, ist der Unternehmer zur Abgabe monatlicher Umsatzsteuer-Voranmeldungen jeweils spätestens bis zum 10. Tag nach Ablauf des betreffenden Monats verpflichtet.
Der Steuerpflichtige kann sich auch für die Abgabe monatlicher Umsatzsteuer-Voranmeldungen entscheiden, wenn er im Vorjahr einen Betrag über 6.136 € erstattet bekommen hat (§ 18 (2a) UStG).
Hat die jährliche Umsatzsteuer nicht mehr als 512 € betragen, sind weder Umsatzsteuer-Voranmeldungen noch Umsatzsteuererklärungen abzugeben.
Eine sich ergebende Umsatzsteuerschuld ist 10 Tage nach Ablauf des Besteuerungszeitraums zu bezahlen.
Eine Rückerstattung kann zu einer gründlicheren Prüfung durch die Finanzverwaltung führen.
(Gesetzliche) Fristen für die Bearbeitung von Umsatzsteuererklärungen und für die Rückerstattung von Umsatzsteuerüberzahlungen bestehen nicht.
Um eine zeitnahe Rückerstattung zu erhalten, kann eine Sicherheit für den Rückerstattungsbetrag angeboten werden.
Grundsätzlich sind Umsatzsteuererklärungen, auch Erstattungsfälle, zügig zu bearbeiten.
Das deutsche Umsatzsteuergesetz ermöglicht eine Verlängerung dieser (Abgabe- bzw. Zahlungs-)Fristen (§ 18 (6) UStG).
Die Fristverlängerung muss gesondert beantragt werden.


\end{document}
