\documentclass[10pt]{article}

%\usepackage{times}
\usepackage{fancyhdr}

\pagestyle{fancy}
\fancyhf{}
\rhead{\textbf{Code: serpent}}

\begin{document}

\subsection*{Source:}



Summarised evaluation and recommendations

The parallel audits of the NKU and the BRH focused on two issues which are essential both in the Czech Republic and in Germany for the awarding of public works contracts: the application of EU procurement law and the prevention of corruption.
The provisions of EU law provide the opportunity for international competition within the EU in order to create a European Single Market.
Measures designed to prevent corruption do not only aim at avoiding or at least reducing corruption. They are also to prevent financial damage and potential loss of confidence in the integrity of government.
The audit findings generated by the two SAIs with respect to the application of procurement law in both countries and the comparison of the respective provisions of procurement law provide an insight into the common and differing features that may be helpful for both countries in further developing their procurement law and its application.


Application of EU procurement law

Both countries have transposed into national law the EU legislation concerning the awarding of public works contracts.
Thus, an important common basis exists in both countries for the awarding of public works contracts and the compliance with the essential requirements in this field: competition; equal treatment of the participating enterprises and transparency of the contract award procedure.
The construction administrations of both countries have underpinned these requirements by partly very extensive administrative regulations.
The Czech Republic has largely adopted the provisions on public works contracts requiring EU-wide tendering for those contract award procedures that are below the EU threshold and are therefore governed by national law.
However, the SAIs‘ audits revealed that the construction administrations did not always fully comply with all relevant provisions of EU-law.
The construction administration in Germany should continue to improve the quality of its contract award procedures in order to eliminate the deficiencies pointed out with respect to the implementation of EU requirements.
In the Czech Republic, the audits furthermore identified the risks impacting on the value for money achieved in contract award procedures.
For instance, incomplete specification of the public contract subject, resulting in need of extra works, often awarded in contradiction to the law. Furthermore, insufficient specification of the expected price was provided. Also, qualification requirements limited the possible number of bidders, thus restricting competition, namely in the area of transport constructions.


\pagebreak

\subsection*{Translation:}



Zusammenfassung der Empfehlungen und Empfehlungen

Die parallelen Prüfungen der NKU und der BRH konzentrierten sich auf zwei Themen, die sowohl in der Tschechischen Republik als auch in Deutschland für die Vergabe öffentlicher Bauaufträge von wesentlicher Bedeutung sind: die Anwendung des EU-Vergaberechts und die Verhinderung von Korruption.
Die Bestimmungen des EU-Rechts bieten die Möglichkeit für den internationalen Wettbewerb innerhalb der EU, um einen europäischen Binnenmarkt zu schaffen.
Maßnahmen zur Verhinderung von Korruption zielen nicht nur darauf ab, Korruption zu vermeiden oder zumindest zu reduzieren. Sie sollen auch finanziellen Schaden und potenziellen Vertrauensverlust in die Integrität der Regierung verhindern.
Die Prüfungsfeststellungen der beiden ORKB in Bezug auf die Anwendung des Vergaberechts in beiden Ländern und der Vergleich der entsprechenden Bestimmungen des Vergaberechts geben einen Einblick in die gemeinsamen und unterschiedlichen Merkmale, die beiden Ländern bei der Weiterentwicklung ihrer Beschaffung hilfreich sein können Gesetz und seine Anwendung.


Anwendung des EU-Vergaberechts

Beide Länder haben die EU-Rechtsvorschriften über die Vergabe öffentlicher Bauaufträge in nationales Recht umgesetzt.
So besteht in beiden Ländern eine wichtige gemeinsame Basis für die Vergabe öffentlicher Bauaufträge und die Einhaltung der grundlegenden Anforderungen in diesem Bereich: Gleichbehandlung der beteiligten Unternehmen und Transparenz des Vergabeverfahrens.
Die Baubehörden beider Länder haben diese Anforderungen durch teilweise sehr umfangreiche Verwaltungsvorschriften gestützt.
Die Tschechische Republik hat weitgehend die Bestimmungen über öffentliche Bauaufträge verabschiedet, die EU-weite Ausschreibungen für diejenigen Vergabeverfahren erfordern, die unterhalb der EU-Schwelle liegen und daher nationalem Recht unterliegen.
Die Prüfungen der ORKB haben jedoch ergeben, dass die Bauverwaltungen nicht immer alle einschlägigen Bestimmungen des EU-Rechts vollständig einhalten.
Die Baubehörde in Deutschland sollte die Qualität der Vergabeverfahren weiter verbessern, um die Mängel bei der Umsetzung der EU-Anforderungen zu beseitigen.
In der Tschechischen Republik haben die Prüfungen außerdem die Risiken identifiziert, die sich auf das Preis-Leistungs-Verhältnis im Rahmen von Vergabeverfahren auswirken.
Zum Beispiel, eine unvollständige Angabe des öffentlichen Auftragsgegenstands, die zusätzliche Arbeiten erforderlich macht, wird oftmals im Widerspruch zum Gesetz vergeben. Darüber hinaus wurde der erwartete Preis nicht ausreichend angegeben. Durch die Qualifikationsanforderungen wurde auch die mögliche Anzahl von Bietern begrenzt, was den Wettbewerb einschränkte, insbesondere im Bereich der Verkehrsbauten.


\end{document}
