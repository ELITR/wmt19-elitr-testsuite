\documentclass[10pt]{article}

%\usepackage{times}
\usepackage{fancyhdr}

\pagestyle{fancy}
\fancyhf{}
\rhead{\textbf{Code: bicycle}}

\begin{document}

\subsection*{Source:}

However, the reminder, the threat and the assessment of three additional fines take too much time in cases where a taxpayer may be involved in a cross-border carousel fraud.
Those cases that were finally reported to the competent tax offices, were usually completed within one or two months.
The tax office refused the tax exemption of the declared intra-Community supplies in 73 \% of these cases.
If taxpayers submitted the right VAT ID the tax office asked them to submit a corrected RS to the CLO, because the tax offices are not able to correct the VIES data by themselves.
The BRH found that neither the tax offices nor the CLO monitored the submission of the corrected RS. So often taxpayers did not comply with the request.
Due to the fact that the CLO does not have access to the VAT return data a comparison of the data on supply of goods declared in the VAT return and in the RS does not take place before submitting those data to the CLO of the other Member States.
This check is possible only for the tax offices and is usually made when preparing a VAT audit.
Data received from other Member States are processed in the CLO.
Incorrect VAT IDs are selected and submitted to the respective Member States.
The data are used for purposes of acquisition control in the tax offices.
As soon as they are in the system the CLO checks them against the data of previous calendar quarters.
In cases that comply with certain criteria (see 5.2) control information is created and sent to the competent tax offices.
This is done permanently whereas a compilation of the acquisition data for all taxpayers is created later and submitted to the tax offices.
The RMS installed for the assessment procedure then checks those data against the data of intraCommunity acquisitions.
This system completely covers those entrepreneurs, who are not fully entitled to VAT deduction.


Comparison and evaluation

In the CR, taxpayers that delivered intra-Community supplies to a person registered for VAT in another Member State, submit a RS within 25 days after the end of the calendar quarter to a competent local tax office.
Tax offices process RS and transmit the data to the CLO.


\pagebreak

\subsection*{Translation:}

Die Mahnung, die Drohung und die Bewertung von drei zusätzlichen Geldbußen nehmen jedoch zu viel Zeit in Anspruch, wenn ein Steuerpflichtiger in einen grenzübergreifenden Karussellbetrug verwickelt sein könnte.
Die Fälle, die schließlich den zuständigen Finanzämtern gemeldet wurden, wurden in der Regel innerhalb von ein bis zwei Monaten abgeschlossen.
In 73\% dieser Fälle lehnte das Finanzamt die Steuerbefreiung der deklarierten innergemeinschaftlichen Lieferungen ab.
Wenn die Steuerpflichtigen die richtige Umsatzsteuer-ID einreichten, forderte das Finanzamt sie auf, dem CLO eine korrigierte RS vorzulegen, da die Finanzämter nicht in der Lage sind, die VIES-Daten selbst zu korrigieren.
Die BRH stellte fest, dass weder die Finanzämter noch die CLO die Vorlage der korrigierten RS überwachten. So oft kamen die Steuerzahler der Aufforderung nicht nach.
Da die CLO keinen Zugang zu den MwSt-Rückgabendaten hat, findet ein Vergleich der Daten über die Lieferung von Waren, die in der Mehrwertsteuererklärung und in der RS deklariert sind, nicht vor der Übermittlung dieser Daten an den CLO der anderen Mitgliedstaaten statt.
Diese Prüfung ist nur für die Finanzämter möglich und wird in der Regel bei der Vorbereitung einer Mehrwertsteuerprüfung vorgenommen.
Die von anderen Mitgliedstaaten erhaltenen Daten werden im CLO verarbeitet.
Falsche Mehrwertsteuerausweise werden ausgewählt und den jeweiligen Mitgliedstaaten vorgelegt.
Die Daten werden für Zwecke der Akquisitionskontrolle in den Finanzämtern verwendet.
Sobald sie sich im System befinden, prüft der CLO sie anhand der Daten früherer Kalenderquartale.
In Fällen, die bestimmten Kriterien entsprechen (siehe 5.2), werden Kontrollinformationen erstellt und an die zuständigen Finanzämter übermittelt.
Dies geschieht permanent, während eine Zusammenstellung der Erfassungsdaten für alle Steuerzahler später erstellt und den Finanzämtern vorgelegt wird.
Das für das Bewertungsverfahren installierte RMS prüft diese Daten dann auf die Daten von innergemeinschaftlichen Erwerben.
Dieses System gilt vollständig für Unternehmer, die keinen vollen Anspruch auf Vorsteuerabzug haben.


Vergleich und Bewertung

In der Tschechischen Republik legen Steuerpflichtige, die innergemeinschaftliche Lieferungen an eine in einem anderen Mitgliedstaat für die Mehrwertsteuer registrierte Person erbracht haben, innerhalb von 25 Tagen nach Ablauf des Kalenderviertels bei einem zuständigen örtlichen Finanzamt eine RS vor.
Die Finanzämter verarbeiten RS und übermitteln die Daten an die CLO.


\end{document}
