\documentclass[10pt]{article}

%\usepackage{times}
\usepackage{fancyhdr}

\pagestyle{fancy}
\fancyhf{}
\rhead{\textbf{Code: dealer}}

\begin{document}

\subsection*{Source:}

In isolated cases, the contract notices published at national level included more information – e.g. concerning performance deadlines – or were published earlier than the notices published in the European Union’s Official Journal.


Ex-ante transparency

The units of one construction administration had commissioned a service company with publishing the national notices.
The latter published the notices on its own procurement platform for the use of which fees are charged.
The construction authorities had however not checked whether this company had complied with the requirement to pass on the notices to the Federal Government’s free internet portal.
Sample audits showed that only one third of the notices had been published on the latter portal.


NKÚ



Contract notice

In case of the construction of a water treatment plant, the contracting authority in its publication of award procedure to suppliers did not indicate the criteria it intended to apply in selecting tenderers.
The contracting authority was not able to prove the way of bid selection.
Reduction of number of bidders was, according to previous legislation, duly in place.
In case of a road circle construction the contracting authority did not specify the figures for evaluation of compliance with qualification criteria and did not indicate the number of bidders to be invited nor in publication of restricted procedure opening neither in the contract award documentation.
The contracting authority evaluated the qualification requirements of all 6 bidders in the restricted procedure as being complied with, but one bidder was excluded because he met the qualification requirements on the last sixth position.
These steps of contracting authority were not transparent, because no information concerning the qualification particulars evaluation and the method of evaluation had ever been published.


Publication of notices

In case of ex-military areas reconstructions, the contracting authority did not publish the written invitation to tender.


Conclusions and recommendations / summarised findings

The two-stage information procedure with prior information and contract notice is an important instrument for implementing the transparency requirement that applies to EU-wide tendering procedures and thus for enhancing competition and preventing corruption.
By extending the obligation to publish prior information to national tendering procedures in the Czech Republic and by applying this obligation in Germany in cases of open and restricted tendering, both countries have implemented EU requirements and thus increased the transparency of national tendering procedures.


\pagebreak

\subsection*{Translation:}

In Einzelfällen enthielten die auf nationaler Ebene veröffentlichten Vertragsmitteilungen mehr Informationen - z. B. über Leistungsfristen - oder wurden früher als die im Amtsblatt der Europäischen Union veröffentlichten Mitteilungen veröffentlicht.


Transparenz von Ex-ante

Die Einheiten einer Bauverwaltung hatten ein Dienstleistungsunternehmen damit beauftragt, die nationalen Ausschreibungen zu veröffentlichen.
Diese hat die Ausschreibungen auf ihrer eigenen Beschaffungsplattform veröffentlicht, für deren Nutzung Gebühren erhoben werden.
Die Baubehörden hatten jedoch nicht geprüft, ob dieses Unternehmen der Anforderung nachgekommen war, die Hinweise an das kostenlose Internetportal der Bundesregierung weiterzuleiten.
Musteraudits ergaben, dass nur ein Drittel der Mitteilungen auf dem letztgenannten Portal veröffentlicht worden war.


NKD hat sich neu aufgestellt



Vertrag bis zur Kündigung

Im Falle des Baus einer Wasseraufbereitungsanlage hat der Auftraggeber in seiner Veröffentlichung des Vergabeverfahrens an Lieferanten nicht angegeben, welche Kriterien er bei der Auswahl der Bieter anzuwenden gedenkt.
Der Auftraggeber war nicht in der Lage, die Art der Angebotsauswahl nachzuweisen.
Die Verringerung der Zahl der Bieter war nach früheren Rechtsvorschriften ordnungsgemäß erfolgt.
Im Falle eines Kreisstraßenbaus nannte der Auftraggeber weder in der Auftragsvergabedokumentation noch in der Ausschreibung die Zahl der einzugeladenen Bieter oder in der Veröffentlichung der Eröffnung eines eingeschränkten Verfahrens die Zahlen für die Bewertung der Erfüllung der Qualifizierungskriterien.
Der Auftraggeber bewertete die Qualifikationsanforderungen aller 6 Bieter im beschränkten Verfahren als erfüllt, ein Bieter wurde jedoch ausgeschlossen, weil er die Qualifikationsanforderungen auf dem letzten sechsten Platz erfüllte.
Diese Schritte des Auftraggebers waren nicht transparent, da keine Informationen über die Qualifikationsspezifikationen und die Bewertungsmethode veröffentlicht worden waren.


Veröffentlichung von Mitteilungen

Im Falle der Rekonstruktionen von ehemaligen militärischen Gebieten hat der Auftraggeber die schriftliche Ausschreibung nicht veröffentlicht.


Schlussfolgerungen und Empfehlungen / zusammengefasste Ergebnisse

Das zweistufige Informationsverfahren mit vorheriger Unterrichtung und Ausschreibung ist ein wichtiges Instrument zur Umsetzung der Transparenzanforderungen, die für EU-weite Ausschreibungsverfahren gelten, und somit zur Verbesserung des Wettbewerbs und zur Verhinderung von Korruption.
Durch die Ausweitung der Verpflichtung zur Veröffentlichung vorheriger Informationen auf nationale Ausschreibungsverfahren in der Tschechischen Republik und die Anwendung dieser Verpflichtung in Deutschland bei offenen und eingeschränkten Ausschreibungen haben beide Länder EU-Anforderungen umgesetzt und damit die Transparenz der nationalen Ausschreibungsverfahren erhöht.


\end{document}
