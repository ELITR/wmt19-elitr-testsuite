\documentclass[10pt]{article}

%\usepackage{times}
\usepackage{fancyhdr}

\pagestyle{fancy}
\fancyhf{}
\rhead{\textbf{Code: preview}}

\begin{document}

\subsection*{Source:}

Die entsprechende Mitteilung ist spätestens drei Monate nach Eingangsdatum der Empfangsbestätigung des Ersuchens zu übersenden.
Nach Erhalt des Ersuchens prüft das Finanzministerium, ob das Ersuchen alle rechtlichen Anforderungen erfüllt.
Danach soll es über die betreffende Finanzdirektion das Ersuchen an das zuständige FA übermitteln. Wird dem Beitreibungsersuchen stattgegeben, hat das FA u. a. die Vermögensverhältnisse des Schuldners zu ermitteln, z. B. durch Auszüge aus dem Grundbuch, Erkundigung über vorhandene Bankkonten und Registrierung von Kfz bei der Zulassungsstelle.
Nach Ablauf von sechs Monaten ab dem Eingangsdatum der Empfangsbestätigung des Ersuchens soll das Finanzministerium der CZ die ersuchende Behörde über das Ergebnis der angestellten Ermittlungen unterrichten.
Die ersuchende Behörde ist auch dann zu unterrichten, wenn die betreffenden Steuerpflichtigen nicht erreichbar sind und die Forderung deshalb nicht beigetrieben werden kann.
Alle erfolgreich beigetriebenen Summen sind innerhalb eines Monats nach erfolgter Beitreibung an die ersuchende Behörde zu überweisen.
Das mit der Beitreibung befasste tschechische FA kann mit Zustimmung des ersuchenden Staates die Forderung stunden oder Ratenzahlung vereinbaren.


Die dem Schuldner für den Verzugszeitraum auferlegten Zinsen oder Säumniszuschläge gemäß dem Gesetz Nr. 337/1992 betreffend die Verwaltung von Steuern und Gebühren

stehen dem ersuchenden Staat zu.
Insbesondere in den Jahren 2006 und 2007 hat das Finanzministerium der CZ dem zuständigen FA die eingegangenen Beitreibungsersuchen verzögert zugeleitet.
Gleiches galt für die Weiterleitung der von den tschechischen FÄ gestellten Beitreibungsersuchen an die zuständige Behörde im ersuchten Staat.
Die Prüfung ergab, dass das Finanzministerium gegen das Gesetz über die zwischenstaatliche Amtshilfe für die Beitreibung von Forderungen verstoßen hat. Es hatte versäumt, die zuständige Behörde des ersuchenden Mitgliedstaats über den Sachstand des Beitreibungsverfahrens innerhalb von sechs Monaten ab Bestätigung des Eingangs des Ersuchens zu informieren.


\pagebreak

\subsection*{Translation:}

The corresponding communication shall be sent no later than three months after the date of receipt of the confirmation of receipt of the request.
Upon receipt of the request, the Ministry of Finance will assess whether the request fulfils all legal requirements.
The request shall then be transmitted to the competent FA via the relevant Financial Directorate. If the request for collection is granted, the FA shall, among other things, determine the debtor's assets, e.g. by extracts from the Land Registry, inquiring about existing bank accounts and registration of motor vehicles with the registration office.
After six months from the receipt date of the receipt confirmation of the request, the Ministry of Finance of the CZ shall inform the requesting authority of the outcome of the enquiry.
The requesting authority must also be informed if the taxable persons concerned are not reachable and the claim cannot therefore be settled.
All sums successfully contributed must be transferred to the requesting authority within one month of the successful recovery.
The Czech FA concerned with the recovery may, with the consent of the requesting State, agree to the demand for overtime or instalment payment.


Interest or default surcharges imposed on the debtor for the period of delay pursuant to Law No. 337 / 1992 concerning the management of taxes and fees

We stand with the requesting state.
In particular, in 2006 and 2007, the Ministry of Finance of the CZ sent the relevant FA delayed requests for recovery.
The same was true for the forwarding of the requests for recovery made by the Czech authorities to the competent authority in the requested State.
The audit found that the Ministry of Finance had breached the law on intergovernmental assistance for the recovery of claims by failing to inform the competent authority of the requesting Member State of the state of play of the recovery procedure within six months of confirmation of receipt of the request.


\end{document}
