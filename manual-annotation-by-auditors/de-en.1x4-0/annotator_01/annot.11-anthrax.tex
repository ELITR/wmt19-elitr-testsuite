\documentclass[10pt]{article}

%\usepackage{times}
\usepackage{fancyhdr}

\pagestyle{fancy}
\fancyhf{}
\rhead{\textbf{Code: anthrax}}

\begin{document}

\subsection*{Source:}

da sich die Leistungsumfänge wesentlich vergrößert und die technischen Voraussetzungen geändert hatten.
Ausschreibungsunterlagen für Sozialbauten enthielten ungenaue Ortsangaben, unzureichende bautechnische Voruntersuchungen und unvollständige Konstruktionszeichnungen, wodurch zusätzliche Leistungen beauftragt werden mussten.
Beim Bau eines großen Bibliotheksgebäudes enthielt die Leistungsbeschreibung der Innenausstattung keine Anforderungen zur Produktqualität.
In Folge dieser fehlenden Vorgaben bauten einige Subunternehmen überteuerte und unangemessen hochwertige Ausstattungsgegenstände ein.


Lose und Eignungskriterien

In der Tschechischen Republik ist die Aufteilung von Baumaßnahmen in Teillose bzw. die Vergabe von Teilaufträgen nicht verpflichtend.
Eine Prüfung der drei großen Abschnitte des Baus eines Autobahnrings ergab, dass bei einer losweisen Vergabe ausgewählter Teile Einsparungen bis zu 31,6 Mio. Euro bei Gesamtkosten von 620,3 Mio. Euro möglich gewesen wären.
So hätten allein bei sechs verschiedenen Bauwerken, die Gegenstand eines einzigen Auftrags waren, bei einer losweisen Vergabe 2,5 Mio. Euro (25 \%) eingespart werden können.
Bei einer losweisen Vergabe des Baus von 20 Brücken wären gegenüber einer Gesamtvergabe Einsparungen von 11,5 Mio. Euro (25 \%) möglich gewesen.
Bei der Vergabe von Bauleistungen im Straßenbau gab die zentrale Vergabestelle als Eignungskriterien sehr umfangreiche Referenzanforderungen vor.
Beim Bau eines Teils einer Autobahn verlangte die Vergabestelle von den Bietern in einem offenen Verfahren u.a. den Nachweis von mindestens drei durchgeführten Straßenbauprojekten im Wert von jeweils 28 Mio. Euro aus den letzten fünf Jahren.
In der Tschechischen Republik erfüllten höchstens neun Anbieter diese Anforderungen.
Die öffentliche Vergabe eines anderen Teilstücks einer Autobahn umfasste sechs separate Bauvorhaben, bei denen die Vergabestelle als Eignungskriterium für die Unternehmen ein jährliches Bauvolumen von mindestens 480 Mio. Euro in den letzten drei Jahren vorgegeben hatte.
Zwei Drittel der in den Jahren 2008 und 2009 für Straßenbauvorhaben aufgebrachten Mittel (d.h. 2,5 Mrd. Euro von insgesamt 3,7 Mrd. Euro) entfielen auf Beauftragungen von insgesamt fünf Unternehmen, die die Aufträge allein oder als Arbeitsgemeinschaft durchführten.
Bei 46 Vergabeverfahren mit einem Auftragsumfang von 528 Mio. Euro gaben nur zwei bzw. sogar nur ein Bieter ein Angebot ab.


\pagebreak

\subsection*{Translation:}

since the scope of services had increased considerably and the technical requirements had changed.
Tendering documents for social buildings contained inaccurate location information, inadequate preliminary engineering investigations and incomplete design drawings, which meant that additional services had to be commissioned.
During the construction of a large library building, the description of the performance of the interior did not contain requirements for product quality.
As a result of these missing requirements, some subcontractors installed overpriced and unreasonably high-quality equipment.


Loose and eligibility criteria

In the Czech Republic, the division of construction activities into partial contracts or the award of partial contracts is not mandatory.
An examination of the three large sections of the construction of a motorway ring showed that if selected parts were awarded on a one-off basis, savings of up to EUR 31.6 million would have been possible at a total cost of EUR 620.3 million.
Six different buildings, which were the subject of a single contract alone, could have saved EUR 2.5 million (25\%) if they had been awarded on a one-off basis.
If the construction of 20 bridges had been awarded on a one-off basis, savings of EUR 11.5 million (25\%) would have been possible compared to a total allocation.
When awarding construction works in road construction, the central procurement authority set very extensive reference requirements as eligibility criteria.
In an open procedure for the construction of part of a motorway, the contracting authority required the bidders to prove, among other things, that at least three road projects worth EUR 28 million had been carried out in the last five years.
In the Czech Republic, a maximum of nine providers met these requirements.
The public procurement of another section of a motorway involved six separate construction projects, for which the contracting authority had set an annual construction volume of at least EUR 480 million over the last three years as an eligibility criterion for the companies.
Two thirds of the funds raised for road projects in 2008 and 2009 (i.e. EUR 2.5 billion out of a total of EUR 3.7 billion) were for contracts awarded by a total of five companies which carried out the contracts alone or as a working group.
In 46 tendering procedures with an order volume of EUR 528 million, only two or even one bidder made an offer.


\end{document}
