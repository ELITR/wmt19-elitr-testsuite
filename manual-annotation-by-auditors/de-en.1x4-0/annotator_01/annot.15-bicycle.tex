\documentclass[10pt]{article}

%\usepackage{times}
\usepackage{fancyhdr}

\pagestyle{fancy}
\fancyhf{}
\rhead{\textbf{Code: bicycle}}

\begin{document}

\subsection*{Source:}

Allerdings benötigt die Erinnerung sowie die Androhung und Festsetzung von bis zu drei aufeinanderfolgenden Zwangsgeldern zu viel Zeit, wenn ein Steuerpflichtiger möglicherweise an einem grenzüberschreitenden Karussellbetrug beteiligt ist.
Die Fälle, die letztendlich den zuständigen FÄ gemeldet wurden, wurden in der Regel innerhalb von ein bis zwei Mona-ten abgeschlossen.
Das jeweils zuständige FA verweigerte die Steuerbefreiung für die gemeldeten innergemeinschaftlichen Lieferungen in 73 \% dieser Fälle9.
Legte der Steuerpflichtige die richtige USt-IdNr. vor, forderte das FA ihn zur Abgabe einer berichtigten ZM an das CLO auf, weil die FÄ die MIAS-Daten nicht selbst berichtigen können.
Der BRH stellte fest, dass weder die FÄ noch das CLO die Abgabe der berichtigten ZM kontrollierten.
Deshalb kamen die Steuerpflichtigen der Aufforderung oft nicht nach. Da das CLO keinen Zugriff auf die Daten der Umsatzsteuererklärungen hat, erfolgt kein Abgleich der in den UmsatzsteuerVoranmeldungen bzw. der Umsatzsteuererklärung gemachten Angaben über Warenlieferungen mit den entsprechenden Angaben in den ZM, bevor diese Daten den CLO der anderen Mitgliedstaaten zugeleitet werden.
Dieser Abgleich kann nur von den FÄ vorgenommen werden. Dies geschieht üblicherweise bei der Vorbereitung einer Umsatzsteuer-Sonderprüfung.
Aus anderen Mitgliedstaaten eingehende Daten werden beim CLO verarbeitet.
Unrichtige USt-IdNr. werden erkannt und den betroffenen Mitgliedstaaten gemeldet.
Die Daten werden für die Kontrolle der Erwerbsbesteuerung in den FÄ genutzt.
Sobald sie im System verfügbar sind, gleicht sie das CLO mit den Daten vorangegangener Kalenderquartale ab.
Erfüllen sie bestimmte Kriterien (siehe 5.2), werden Kontrollmitteilungen erstellt und an die zuständigen FÄ übermittelt.
Dies geschieht fortlaufend, wohingegen eine Auflistung der Erwerbsdaten aller Umsatzsteuerpflichtigen später erstellt und den FÄ zugeleitet wird.
Das Risikomanagementsystem (RMS) für das Veranlagungsverfahren gleicht dann diese Daten mit denjenigen zu innergemeinschaftlichen Erwerben ab.
Das System überprüft vollständig (nur) diejenigen Unternehmer, die nicht in vollem Umfang zum Abzug der Vorsteuer berechtigt sind.


Vergleich und Evaluierung

In der Tschechischen Republik müssen Steuerpflichtige, die innergemeinschaftliche Lieferungen an eine in einem anderen Mitgliedstaat zur Umsatzsteuer registrierte Person getätigt haben, innerhalb von 25 Tagen nach Ende des Kalenderquartals eine ZM an das örtlich zuständige FA abgeben.
Die FÄ bearbeiten die ZM und leiten die Daten an das CLO weiter.


\pagebreak

\subsection*{Translation:}

However, the reminder, the threat and the assessment of three additional fines take too much time in cases where a taxpayer may be involved in a cross-border carousel fraud.
Those cases that were finally reported to the competent tax offices, were usually completed within one or two months.
The tax office refused the tax exemption of the declared intra-Community supplies in 73 \% of these cases.
If taxpayers submitted the right VAT ID the tax office asked them to submit a corrected RS to the CLO, because the tax offices are not able to correct the VIES data by themselves.
The BRH found that neither the tax offices nor the CLO monitored the submission of the corrected RS. So often taxpayers did not comply with the request.
Due to the fact that the CLO does not have access to the VAT return data a comparison of the data on supply of goods declared in the VAT return and in the RS does not take place before submitting those data to the CLO of the other Member States.
This check is possible only for the tax offices and is usually made when preparing a VAT audit.
Data received from other Member States are processed in the CLO.
Incorrect VAT IDs are selected and submitted to the respective Member States.
The data are used for purposes of acquisition control in the tax offices.
As soon as they are in the system the CLO checks them against the data of previous calendar quarters.
In cases that comply with certain criteria (see 5.2) control information is created and sent to the competent tax offices.
This is done permanently whereas a compilation of the acquisition data for all taxpayers is created later and submitted to the tax offices.
The RMS installed for the assessment procedure then checks those data against the data of intraCommunity acquisitions.
This system completely covers those entrepreneurs, who are not fully entitled to VAT deduction.


Comparison and evaluation

In the CR, taxpayers that delivered intra-Community supplies to a person registered for VAT in another Member State, submit a RS within 25 days after the end of the calendar quarter to a competent local tax office.
Tax offices process RS and transmit the data to the CLO.


\end{document}
