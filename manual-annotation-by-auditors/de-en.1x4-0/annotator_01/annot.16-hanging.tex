\documentclass[10pt]{article}

%\usepackage{times}
\usepackage{fancyhdr}

\pagestyle{fancy}
\fancyhf{}
\rhead{\textbf{Code: hanging}}

\begin{document}

\subsection*{Source:}

Es wurden dazu von Steuerpflichtigen innerhalb der EU ausgeführte und in EUROCANET erfasste risikobehaftete innergemeinschaftliche Umsätze ausgewählt (siehe Kapitel 4.3).
Durch das Netzwerk EUROFISC soll die Amtshilfe in Steuersachen optimiert und der USt-Betrug wirksamer bekämpft werden. Die Zusammenarbeit der zuständigen Behörden soll verbessert werden, um Betrugsfälle möglichst frühzeitig zu erkennen.
Später könnte EUROFISC als Hilfsmittel genutzt werden, um das Risikopotenzial bei innergemeinschaftlichen Umsätzen abzuschätzen.
Der ECOFIN-Rat6 hat EUROFISC bereits am 7. Oktober 2008 gebilligt und gleichzeitig die Grundsätze für die Arbeitsweise des Systems verabschiedet.
Als Vorbild für EUROFISC dient EUROCANET, das von der belgischen Finanzverwaltung in Zusammenarbeit mit den anderen Mitgliedstaaten sowie mit Unterstützung der Europäischen Kommission und von OLAF7 entwickelt worden ist.
Zweck des Systems ist es, risikobehaftete Umsätze aufzudecken und Unternehmer, die am organisierten USt-Betrug beteiligt sind, zu ermitteln.


4.1.2 Umsetzung der Empfehlungen in Deutschland



Die Bewertung der Empfehlungen aus dem ersten Bericht führte zu folgenden Feststellungen:



Empfehlung 1



Harmonisierung der Voraussetzungen für die Registrierung von Steuerpflichtigen innerhalb der EU

Die deutsche Finanzverwaltung hat anhand eines detaillierten Fragebogens ein gut funktionierendes System für die Registrierung von Steuerpflichtigen entwickelt.
Das Bundesfinanzministerium hat dieses System der Kommission im Oktober 2007 vorgestellt.
Das System selbst wird ständig verbessert.
Die im Fragebogen angegebenen Daten werden jetzt elektronisch erfasst und stehen später für Nachforschungen zur Verfügung. Sie sind somit Teil des Risikomanagements bei der Registrierung.
Die Kommission hatte einen Vorschlag8 zur Änderung der Verordnung 1798/2003 gemacht. Dieser sah u. a. gemeinsame Standards für die Registrierung und Löschung von Steuerpflichtigen vor.
Bei den Mitgliedstaaten bestand hierzu noch Klärungs- und Präzisierungsbedarf.
Der Bundesrechnungshof hatte dem Bundesfinanzministerium empfohlen, nicht hinter die bereits erreichten Standards zurückzugehen.


Empfehlung 2



Monatliche Abgabe der ZM

Am 8. April 2010 wurde ein Gesetz9 zur Umsetzung der EU-Richtlinien in nationales Recht verabschiedet.


Es enthält u. a. Bestimmungen zur monatlichen Abgabe der ZM für innergemeinschaftliche Lieferungen

und tritt zum 1. Juli 2010 in Kraft.


Empfehlung 3



\pagebreak

\subsection*{Translation:}

For this, risky intra-Community transactions carried out by taxpayers within the EU and recorded in EUROCANET were selected (see section 4.3).
The purpose of the EUROFISC network is to optimize mutual assistance in tax matters and combat VAT fraud more effectively. Cooperation between competent authorities should be improved in order to detect fraud as early as possible.
Later, EUROFISC could be used as a tool to estimate the risk potential of intra-community transactions.
The ECOFIN Council6 approved EUROFISC on 7 October 2008, and at the same time adopted the principles for the operation of the system.
The model for EUROFISC is provided by EUROCANET, developed by the Belgian tax authorities in cooperation with the other Member States and supported by the European Commission and OLAF7.
The purpose of the system is to detect risky revenues and to identify entrepreneurs involved in organized VAT scams.


4.1.2 Implementation of the recommendations in Germany



The evaluation of the recommendations from the first report led to the following findings:



Recommendation 1



Harmonization of conditions for the registration of taxpayers within the EU

The German tax authorities have developed a well-functioning system for the registration of taxpayers on the basis of a detailed questionnaire.
The Federal Ministry of Finance presented this system to the Commission in October 2007.
The system itself is constantly being improved.
The data given in the questionnaire are now electronically recorded and will later be available for further investigation. They are thus part of the risk management at the registration.
The Commission had made a proposal8 to amend Regulation 1798/2003. It provided, inter alia, common standards for the registration and cancellation of taxpayers.
The Member States still needed clarification and clarification.
The Federal Court of Auditors recommended that the Federal Ministry of Finance refrain from going back beyond the already achieved standards.


Recommendation 2



Monthly delivery of the ZM

On April 8, 2010, a law9 on the transposition of EU directives into national law was passed.


It contains, inter alia, provisions on the monthly submission of the CM for intra-Community supplies

and comes into force on 1 July 2010.


Recommendation 3



\end{document}
