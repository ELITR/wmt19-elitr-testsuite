\documentclass[10pt]{article}

%\usepackage{times}
\usepackage{fancyhdr}

\pagestyle{fancy}
\fancyhf{}
\rhead{\textbf{Code: health}}

\begin{document}

\subsection*{Source:}



Außerdem hat sich Deutschland verpflichtet, den anderen EGMitgliedstaaten im Rahmen des strukturierten automatischen Auskunftsaustauschs folgende Informationen zu übermitteln:

a. Die Lieferung neuer Fahrzeuge an Unternehmer ohne UStIdNr., b. Fernverkäufe (Versandhandel) in andere EG-Mitgliedstaaten, die nicht in Deutschland der Umsatzsteuer unterliegen, c. Erhebliche Abweichungen zwischen den von einem Steuerpflichtigen gemeldeten Erwerben und den von anderen Mitgliedstaaten übermittelten MIAS-Daten, d. innergemeinschaftliche Lieferungen von Gegenständen, die vom Steuerpflichtigen unzutreffend in Deutschland der Besteuerung unterworfen wurden, e. potentielle „Missing Traders“, die innergemeinschaftliche Umsätze getätigt haben, unabhängig davon, ob deren UStIdNr. noch gültig ist oder bereits gelöscht wurde sowie f. Steuerpflichtige, die innergemeinschaftliche Lieferungen an potentielle „Missing Trader“ oder an Abnehmer getätigt haben, die eine USt-IdNr. missbräuchlich verwendet haben.


Vergleich und Evaluierung

Die Versendung von Auskunftsersuchen bzw. Spontanauskünften erfolgt sowohl in der Tschechischen Republik als auch in Deutschland auf Initiative der örtlich zuständigen FÄ, die dem CLO die entsprechenden SCACVordrucke übermitteln.
Während das tschechische CLO zweisprachige SCACVordrucke (tschechisch-englisch) benutzt und erteilte Auskünfte in die englische bzw. deutsche Sprache übersetzt, folgen die deutschen FÄ der Anweisung, Auskunftsersuchen oder Spontanauskünfte auf einem einsprachigen Vordruck nur auf deutsch abzufassen.
Das deutsche CLO fertigt keine Übersetzungen deutscher Texte.
Bei der Prüfung von Auskunftsersuchen zwischen der Tschechischen Republik und Deutschland im Zeitraum 1. Mai 2004 bis 31. Dezember 2005 stellte der NKÚ fest, dass die tschechische Finanzverwaltung 74,5\% der eingegangenen Auskunftsersuchen innerhalb der Dreimonatsfrist erledigte, während die an Deutschland gerichteten Auskunftsersuchen von der deutschen Finanzverwaltung nur in 40,6\% der Fälle innerhalb dieser Frist erledigt wurden.
Zwar weist das deutsche CLO die FÄ auf die Dreimonatsfrist hin, überwacht jedoch nicht deren Einhaltung.
Aufgrund des Fehlens von Erinnerungen kommt es zu einem hohen Anteil verfristet beantworteter Auskunftsersuchen.
Deshalb hat der BRH dem deutschen CLO ein kontinuierliches Erinnerungsverfahren empfohlen.
Die Situation hat sich mittlerweile geändert.
Nach den uns vorliegenden Informationen erinnert das deutsche CLO die Finanzämter inzwischen an die Beantwortung der ausstehenden Auskunftsersuchen.


Umsatzsteuer-Risikomanagement



Risikomanagement in der Tschechischen Republik

In der Tschechischen Republik sind mehrere Organisationseinheiten auf verschiedenen Ebenen der Finanzverwaltung mit dem Risikomanagement im Bereich der Umsatzsteuer befasst.


\pagebreak

\subsection*{Translation:}



Furthermore, Germany has committed itself to providing other EC Member States with the following information through a structured automatic exchange:

a. The supply of new means of transport to entrepreneurs without using their VAT ID, b. supplies by distance or catalogue selling to other Member States which are not taxable in Germany, c. significant differences between the intra-Community acquisitions declared by a taxpayer and the VIES data submitted by other Member States, d. intra-Community supplies of goods which the German taxpayer did not report in RS, because the supplies were treated as domestic transactions, e. potential “missing trader” which made intra-Community transactions independent of the validity of their VAT ID and f. entrepreneurs who supplied goods to potential “missing traders” or recipients who misused their VAT ID.


Comparison and evaluation

Sending of RFI or spontaneous information is initiated both in the CR and Germany by competent local tax offices that send filed SCAC forms to the CLO.
While the Czech CLO uses bilingual SCAC forms (CzechEnglish) and translates dispatched information into English or German language, respectively, German tax offices comply with the instruction, according to which they have to write RFI or spontaneous information exclusively in German on a single-language form.
The German CLO does not provide a translation of German texts.
When reviewing RFIs transmitted between the CR and Germany in the period from 1 May 2004 to 31 December 2005, the SAO found that the Czech tax administration settled 74,5 \% of received RFIs in a given three months deadline, while RFIs sent to Germany were settled by the German tax administration in the given deadline in 40,6 \% cases only.
Although the German CLO refers the tax offices to the 3-month-deadline it does not monitor the elapsed time limits.
The outcome of the missing reminders is the high quantity of delayed answers to RFIs.
Because of these problems the BRH recommended a steady reminding exercise of the CLO.
The situation has already improved.
According to our latest information the German CLO now reminds the Tos tax offices of answering the incompleted requests for information.


VAT risk management



Risk management in the CR

In the CR, several units at various organisational levels of the tax administration are concerned with VAT risk management.


\end{document}
