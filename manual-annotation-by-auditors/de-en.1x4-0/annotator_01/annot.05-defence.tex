\documentclass[10pt]{article}

%\usepackage{times}
\usepackage{fancyhdr}

\pagestyle{fancy}
\fancyhf{}
\rhead{\textbf{Code: defence}}

\begin{document}

\subsection*{Source:}

Bruchsal, Ettlingen, Freudenstadt, Karlsruhe-Stadt, Leonberg, Mühlacker, Singen, Tuttlingen, Cham, Fürstenfeldbruck, München-Körperschaften, München II, IV und München V, Starnberg, Hamburg-Altona, Hamburg-Hansa, Hamburg-Mitte, Hamburg-Nord und Hamburg für Großunternehmen, Hannover-Nord, Winsen-Luhe, Bonn-Innenstadt, Köln-Mitte, Köln-Porz, Köln-Süd, Sankt Augustin, Pirmasens, Bautzen, Chemnitz-Süd, Dresden II, Freiberg, Leipzig I und Zschopau.
Außerdem wurden Gespräche mit den FÄ Augsburg-Stadt, Aurich, Düsseldorf-Altstadt, Eisleben, Kelheim und Verden geführt.
Die Prüfungsergebnisse wurden dem BMF im Mai 2007 mitgeteilt.


Zusammenarbeit der beiden Rechnungshöfe

Im Vorfeld der parallelen Prüfung fanden mehrere Treffen von Vertretern beider Rechnungshöfe statt.
Die Teilnehmer an diesen Treffen machten sich mit der Organisation der Finanzverwaltung, der Gesetzgebung zur Umsatzsteuer und dem System der Umsatzsteuerverwaltung im jeweils anderen Land vertraut.
Der Gegenstand der Testphase wurde bei dem Treffen im Mai 2006 vereinbart.
Die Ergebnisse der Testphase wurden beim nächsten gemeinsamen Treffen im August 2006 ausgewertet.
Bei diesem Treffen wurde ein Zeitplan für das weitere Verfahren erarbeitet und folgende Prüfungsthemen für die parallelen Prüfungen bestimmt: 1.
Registrierung von Steuerpflichtigen.
2.
Grenzüberschreitende Personenbeförderungen mit Kraftomnibussen – Gesetzgebung in der Tschechischen Republik und Deutschland.
3. Internationaler Auskunftsaustausch, insbesondere das Vorgehen der Finanzbeamten bei der Bearbeitung von Auskunftsersuchen und der Überprüfung ausgewählter Problemfälle, die in den Auskunftsersuchen genannt waren, sowie Spontanauskünfte zwischen der Tschechischen Republik und Deutschland.
4. Steuererklärungen.
5.
Zusammenfassende Meldungen.
6.
Statistische Informationen zu Lieferungen und Erwerben zwischen der Tschechischen Republik und Deutschland.
7. Risikomanagement im Umsatzsteuerbereich.
Bei einem Treffen zur Abstimmung der weiteren Prüfung des NKÚ und des BRH im Dezember 2006 machten sich beide Seiten mit dem Prüfungsverfahren in der Tschechischen Republik und Deutschland vertraut und besprachen die bereits erzielten Ergebnisse der parallelen Prüfungen.


THISsegmentISintentionallyLEFTblank



4. Verwaltung der Umsatzsteuer in der Tschechischen Republik und Deutschland



THISsegmentISintentionallyLEFTblank



THISsegmentISintentionallyLEFTblank



THISsegmentISintentionallyLEFTblank



\pagebreak

\subsection*{Translation:}

Bruchsal, Ettlingen, Freudenstadt, Karlsruhe-Stadt, Leonberg, Mühlacker, Singen, Tuttlingen, Cham, Fürstenfeldbruck, Munich Corporations, Munich II, IV and Munich V, Starnberg, Hamburg-Altona, Hamburg-Hansa, Hamburg-Mitte, Hamburg-North and Hamburg for large companies, Hannover-North, Winsen-Luhe, Bonn-Innenstadt, Cologne-Mitte, Cologne-Porz, Cologne-Süd, Sankt Augustin, Pirmasens, Bautzen, Chemnitz-Süd, Dresden II, Freiberg, Leipzig and Zopschau.
Talks were also held with the FÄ Augsburg-Stadt, Aurich, Düsseldorf-Altstadt, Eisleben, Kelheim and Verden.
The results were communicated to the BMF in May 2007.


Cooperation between the two Court of Auditors

In the run-up to the parallel audit, several meetings of representatives of both Court of Auditors took place.
The participants in these meetings familiarized themselves with the organization of the tax administration, the legislation on the sales tax and the system of the sales tax administration in the other country.
The subject of the test phase was agreed at the meeting in May 2006.
The results of the test phase were evaluated at the next joint meeting in August 2006.
At this meeting, a timetable for the further procedure was drawn up and the following examination topics were identified for the parallel tests: 1.
Registration of tax fugitives.
2.
Cross-border passenger transport by bus and coach - legislation in the Czech Republic and Germany.
3. International exchange of information, in particular the approach of financial officials to the processing of requests for information and the verification of selected problem cases mentioned in the requests for information, as well as spontaneous exchanges between the Czech Republic and Germany.
4. Tax returns.
5.
Summarized messages.
6.
Statistical information on deliveries and purchases between the Czech Republic and Germany.
7. Risk management in the sales tax area.
At a meeting in December 2006 to coordinate the further examination of the NKÚ and the BRH, both sides familiarized themselves with the examination procedure in the Czech Republic and Germany and discussed the results of the parallel tests already achieved.


THISsegmentISintentionallyLEFTblank



4. Management of VAT in the Czech Republic and Germany



THISsegmentISintentionallyLEFTblank



THISsegmentISintentionallyLEFTblank



THISsegmentISintentionallyLEFTblank



\end{document}
