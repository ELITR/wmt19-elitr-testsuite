\documentclass[10pt]{article}

%\usepackage{times}
\usepackage{fancyhdr}

\pagestyle{fancy}
\fancyhf{}
\rhead{\textbf{Code: servant}}

\begin{document}

\subsection*{Source:}

Dass drei große Mitgliedstaaten (Deutschland, Italien und das Vereinigte Königreich) nicht teilnahmen, hat das Europäische Parlament Ende 200811 als Schwäche des Systems betrachtet, die durch Einrichtung eines EUROFISC-Netzwerks mit obligatorischer Beteiligung behoben werden sollte.


4.3.2 Feststellungen in Deutschland

Der Bundesrechnungshof wertete beim Bundeszentralamt für Steuern vorliegende Daten einschließlich solcher aus EUROCANET aus.
Sie dienten als Grundlage für die Auswahl von sechs deutschen Steuerpflichtigen, deren Akten überprüft wurden.
Sowohl Deutschland als auch eine Anzahl anderer Länder verweigerten die aktive Beteiligung an EUROCANET.
Deshalb hat Deutschland einen passiven Status, d. h. es schickt selbst keine Daten nach Brüssel, sondern ist nur Empfänger von EUROCANET-Daten.


Begründet wird dies damit, dass „aus Sicht der Bundesregierung

keine zweifelsfreie Rechtsgrundlage [existiert], die es erlauben würde, das in Deutschland bestehende Steuergeheimnis zu durchbrechen.
Hinzu kommt, dass der Bundesregierung keine zuverlässigen Erkenntnisse vorliegen, wie von den Mitgliedstaaten übersandte Informationen im Rahmen von EUROCANET verarbeitet werden“.12 Seit Oktober 2007 wertet das Bundeszentralamt für Steuern die EUROCANET-Daten aus und übermittelt sie den Finanzbehörden des Bundes und der Länder auf der Grundlage eines abgestimmten Verfahrens.
Die deutschen FÄ erhalten Kontrollmitteilungen über die in EUROCANET erfassten unmittelbaren Geschäftsbeziehungen der bei ihnen registrierten Steuerpflichtigen mit Unternehmen aus anderen EU-Staaten.
Das jeweilige deutsche FA erhält jedoch keine Informationen darüber, ob das Unternehmen aus dem vorgenannten anderen EU-Mitgliedstaat Geschäftsbeziehungen mit einem in einem anderen Mitgliedstaat ansässigen Unternehmen hat, das ebenfalls mit dem deutschen Unternehmen aus der Kontrollmitteilung in Geschäftsverbindung steht.


4.3.3 Evaluierung

Die CZ ist aktiver Teilnehmer an EUROCANET, während Deutschland nur einen passiven Status hat.
Vom Standpunkt eines aktiven Teilnehmers aus gesehen ist es unbefriedigend, dass andere Mitgliedstaaten die aktive Teilnahme verweigern.
Dementsprechend musste die Prüfung in diesem Bereich unterschiedlich durchgeführt werden.
Wesentliche Unterschiede hinsichtlich des Erfolgs bei der Aufdeckung von USt-Betrug konnten jedoch nicht festgestellt werden. Grund dafür ist, dass EUROCANET nur eine Komponente des Risikomanagementsystems darstellt.


\pagebreak

\subsection*{Translation:}

The fact that three large Member States (Germany, Italy and the United Kingdom) did not participate was seen by the European Parliament at the end of 200811 as a weakness of the system, which should be remedied by establishing a mandatory EUROFISC network.


4.3.2 Findings in Germany

The Federal Court of Audit evaluated data available at the Federal Central Tax Office, including those from EUROCANET.
They served as the basis for the selection of six German taxpayers whose files were reviewed.
Both Germany and a number of other countries refused to actively participate in EUROCANET.
Therefore, Germany has a passive status, ie it does not send any data to Brussels itself, but is only a recipient of EUROCANET data.


This is justified by the fact that "from the point of view of the Federal Government

there is no unambiguous legal basis that would allow the tax confidentiality in Germany to be breached.
In addition, the Federal Government has no reliable information on how information sent by the Member States is processed within the framework of EUROCANET. "12 Since October 2007, the Federal Central Tax Office has been evaluating the EUROCANET data and forwarding it to the tax authorities of the federal and state governments the basis of a concerted procedure.
The German FTAs receive control notifications concerning the direct business relations of taxpayers registered with them in EUROCANET with companies from other EU countries.
However, the relevant German FA does not receive any information as to whether the company from the above-mentioned other EU Member State has business relations with a company established in another Member State which is also in business relationship with the German company in the control notification.


4.3.3 Evaluation

The CZ is an active participant in EUROCANET, while Germany has only a passive status.
From the point of view of an active participant, it is unsatisfactory that other Member States refuse active participation.
Accordingly, the test had to be carried out differently in this area.
However, there were no significant differences in the detection of VAT fraud. This is because EUROCANET is only one component of the risk management system.


\end{document}
