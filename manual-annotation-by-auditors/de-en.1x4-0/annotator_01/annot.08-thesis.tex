\documentclass[10pt]{article}

%\usepackage{times}
\usepackage{fancyhdr}

\pagestyle{fancy}
\fancyhf{}
\rhead{\textbf{Code: thesis}}

\begin{document}

\subsection*{Source:}

Prüfungszeitraum waren die Jahre 2006 bis 2008 einschließlich maßgeblicher Informationen aus der Zeit vor dem Prüfungszeitraum bis zum Abschluss der Prüfung.


2. Zusammenfassung der Ergebnisse und Bewertung der parallelen Prüfung 

Die Rechnungshöfe der Tschechischen Republik und der Bundesrepublik Deutschland haben eine Kontrollprüfung zur Verwaltung der USt vereinbart.


Bei der vorherigen parallelen Prüfung hatten sie Verdachtsfälle mit innergemeinschaftlichen Umsätzen aufgedeckt, 

die eine nochmalige Prüfungzweckmäßig erschienen ließen.
Aufgrund der Prüfungsfeststellungen gaben die beiden Rechnungshöfe Empfehlungen zur Verwaltung der USt ab.
Ziel der Kontrollprüfung war es zu evaluieren, wie diese Empfehlungen umgesetzt wurden. Zudem sollten die ermittelten Verdachtsfälle untersucht werden.
Die Prüfung befasste sich auch mit anderen Sachverhalten. Hierzu gehörten ausgewählte Fälle risikoreicher innergemeinschaftlicher Umsätze, die gegenseitige Amtshilfe bei der Beitreibung von USt-Forderungen und die USt-Prüfung bei Großbetrieben.


Dabei haben die Rechnungshöfe festgestellt: 



1. Aufgrund der vorherigen Prüfung abgegebene Empfehlungen 



• Empfehlung 1: Die Regelungen für die Registrierung von Steuerpflichtigen sollten innerhalb der EU harmonisiert werden 

Bei der ersten Prüfung verglichen die Rechnungshöfe die Registrierungsverfahren zur USt in den beiden Staaten und stellten Unterschiede fest, die Probleme verursachten.
Die Rechnungshöfe empfahlen deshalb, die Registrierung der Steuerpflichtigen innerhalb der EU zu harmonisieren.
Dies wurde auch auf EU-Ebene aufgegriffen.
Der Entwurf für die Novellierung der Verordnung Nr. 1798/2003 des Rates sieht vor, die derzeitigen Datenbanken in der EU einschließlich einer Datenbank mit der Identifizierungsnummer der USt-Pflichtigen auszuweiten.
Der Entwurf sah auch gemeinsame Regelungen für die Registrierung der Steuerpflichtigen und ihre etwaige Löschung vor.
Die Rechnungshöfe erwarteten, dass die Mitgliedstaaten sich entsprechend ihrer Grundsatzentscheidung auf geeignete Regelungen einigen, um unzuverlässige Unternehmer so früh wie möglich aus dem USt-System zu entfernen.
Über die letzte Fassung des Entwurfs erzielten die Mitgliedstaaten im ECOFIN politisches Einvernehmen.
Die genannten gemeinsamen Regelungen sind darin nicht mehr enthalten. Die beiden Rechnungshöfe bedauern das.
Sie halten solche Regelungen nach wie vor für wichtig, um Umsatzsteuerbetrug vorbeugend zu bekämpfen.
• Empfehlung 2: Monatliche Vorlage der Zusammenfassenden Meldungen Als Ergebnis der ersten Prüfung empfahlen die Rechnungshöfe den Mitgliedstaaten, dass die Zusammenfassenden Meldungen (ZM) künftig monatlich eingereicht werden sollten.


\pagebreak

\subsection*{Translation:}

The period covered by the audit was 2006 to 2008, including relevant information from the period before the audit until the audit was completed.


2. Summary of the results and evaluation of the parallel examination

The Courts of Auditors of the Czech Republic and the Federal Republic of Germany have agreed on a control audit of the administration of the USt.


In the previous parallel investigation, they had uncovered suspected cases involving intra-Community transactions,

which made a re-examination seem expedient.
Based on the findings of the audit, the two Court of Auditors made recommendations on the administration of the USt.
The aim of the inspection was to evaluate how these recommendations were implemented and to investigate the suspected cases identified.
The audit also dealt with other matters, including selected cases of risky intra-Community transactions, mutual assistance in the recovery of USt claims and the USt audit of large companies.


The Court of Auditors has found that:



1. Recommendations made on the basis of the previous examination



• Recommendation 1: Rules for the registration of taxable persons should be harmonised within the EU

In the first audit, the Court of Auditors compared the registration procedures for the USt in the two countries and found differences that caused problems.
The Court of Auditors therefore recommended harmonising the registration of taxable persons within the EU.
This has also been taken up at EU level.
The draft amendment to Council Regulation No 1798 / 2003 provides for the extension of the existing databases in the EU, including a database with the identification number of those subject to USt.
The draft also provided for common rules for the registration of taxable persons and their possible deletion.
The Court of Auditors expected Member States to agree, in line with its decision in principle, on appropriate arrangements to remove unreliable traders from the USt system as early as possible.
The Member States reached political agreement in ECOFIN on the final version of the draft.
The aforementioned common rules are no longer included, and the two Court of Auditors regret this.
They continue to believe that such arrangements are important in order to prevent VAT fraud.
• Recommendation 2: Monthly presentation of the Summary Notices As a result of the initial examination, the Court of Auditors recommended to the Member States that in future Comprehensive Notices (TIAs) should be submitted monthly.


\end{document}
