\documentclass[10pt]{article}

%\usepackage{times}
\usepackage{fancyhdr}

\pagestyle{fancy}
\fancyhf{}
\rhead{\textbf{Code: absence}}

\begin{document}

\subsection*{Source:}

Im Jahre 2012 soll ein Sonderfinanzamt für Großbetriebe eingerichtet werden.
In Deutschland besteht noch kein vergleichbares System. Dies erschwert die Steuerung und Analyse der USt-Kontrolle insgesamt.
Die deutsche Finanzverwaltung konzentriert sich stärker auf die Auswahl einschlägiger Einzelfälle und stützt sich insbesondere auf Daten aus den USt-Voranmeldungen bzw. USt-Erklärungen und den ZM.
Verschiedene Risikomanagement-Komponenten, die noch verbessert werden sollen, unterstützen die Auswertung dieser Daten.


3. Prüfungsverfahren 



3.1 Prüfungsverfahren in der Tschechischen Republik 

Die Abteilung II - Staatliche Haushaltseinnahmen sowie die Regionalabteilungen IX Pilsen, X České Budějovice, XIII Brno und XV Ostrava prüften im Zeitraum Mai 2009 bis Februar 2010.


Geprüfte Stellen waren das tschechische Finanzministerium und 25 FÄ: Finanzamt (FA) Brno II, FA Brno III, FA Brno IV, FA in Břeclav, FA in Domažlice, FA in Hodonín, FA in Hradec Králové, FA in Cheb, FA in Jablonec nad Nisou, FA in Karlovy Vary, FA in Kralupy nad Vltavou, FA in Kraslice, FA Ostrava I, FA Ostrava III, FA in Písek, FA in Pilsen, FA Prag 1, FA Prag 4, FA Prag 8, FA in Rakovník, FA in Sokolov, FA in Tábor, FA in Vysoké Mýto, FA in Zlín und FA in Žamberk. 



Der Senat des tschechischen Rechnungshofes billigte die Prüfungsfeststellungen am 27. April 2010. 



3.2 Prüfungsverfahren in Deutschland 

Das Prüfungsgebiet Verkehrsteuern und Mitarbeiter der zugeordneten PÄB Berlin, Frankfurt/Main und München prüften vom Februar 2009 bis Februar 2010.


Nach Auswahl der Beispielfälle kündigten sie die Prüfung in mehreren Ländern an:

• Bayern, • Hamburg, • Hessen, • Nordrhein-Westfalen, • Sachsen. Daneben führten sie Interviews im Bundesministerium der Finanzen (Bundesfinanzministerium).


\pagebreak

\subsection*{Translation:}

A special Financial Office for large enterprises is to be set up in 2012.
Germany does not yet have a comparable system. This makes the overall control and analysis of the USt control more difficult.
The German tax administration focuses more on the selection of relevant individual cases and relies in particular on data from the USt pre-registrations and / or the USt declarations and the ZM.
Various risk management components, which are still to be improved, support the evaluation of this data.
3 in the review process.


3.1 Examination procedure in the Czech Republic

Department II - State Revenue and Regional Departments IX Pilsen, X České Budějovice, XIII Brno and XV Ostrava carried out audits between May 2009 and February 2010.
Verified bodies were the Czech Ministry of Finance and 25 FÄ: Tax Office (FA) Brno II, FA Brno III, FA Brno IV, FA in Břeclav, FA in Domažlice, FA in Hodonín, FA in Hradec Králové, FA in Cheb, FA in Jablonec nad Nisou, FA in Karlovy Vary, FA in Karlovy Vary, FA in Kralupy nad Vltavou, FA in Kraslice, FA in Ostrava I FA, FA Ostrava III, FA in Písek, FA Pilsen in Prague, FA 1, FA FA 4 FA Prague, FA 8 FA in Rakovník, FA in Sokolov, FA in Zamberto FA in Zlín in Zamberto FA, FA in Zákín in Sokín and FA in Zamberto FA.
The Senate of the Czech Court of Auditors approved the audit findings on 27 April 2010.


3.2 Testing procedures in Germany

The examination area Traffic Taxes and employees of the assigned PÄB Berlin, Frankfurt / Main and Munich tested from February 2009 to February 2010.


After selecting the sample cases, they announced the examination in several countries:

• Bavaria, • Hamburg, • Hesse, • North Rhine-Westphalia, • Saxony. In addition, they conducted interviews in the Federal Ministry of Finance.


\end{document}
