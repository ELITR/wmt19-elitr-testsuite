\documentclass[10pt]{article}

%\usepackage{times}
\usepackage{fancyhdr}

\pagestyle{fancy}
\fancyhf{}
\rhead{\textbf{Code: regime}}

\begin{document}

\subsection*{Source:}

Gemäß den Bestimmungen von § 37 (1) des Gesetzes Nr. 337/1992 kann das FA wiederholt eine Geldstrafe von bis zu 2 Millionen CZK gegen Personen verhängen, die ihren nichtmonetären Pflichten aus diesem oder einem besonderen Steuergesetz oder einer Entscheidung gemäß diesem Gesetz nicht nachkommen.
Automated tax information system – Automatisiertes Steuerinformationssystem, welches die tschechischen Finanzämter zur technischen Unterstützung der Steuererhebung einsetzen.
Der NKÚ stellte bei seiner Prüfung fest, dass die FÄ von der Möglichkeit, bei Nichtabgabe der Umsatzsteuererklärung eine Geldstrafe gegen den betreffenden Steuerpflichtigen zu verhängen, selten Gebrauch machten.


Abgabe und Bearbeitung von Umsatzsteuererklärungen in Deutschland

In Deutschland wird zwischen monatlichen oder vierteljährlichen Umsatzsteuer-Voranmeldungen und der jährlichen Umsatzsteuererklärung unterschieden.
Diejenigen Steuerpflichtigen, die zur Abgabe von Umsatzsteuer-Voranmeldungen verpflichtet sind, müssen trotzdem eine jährliche Umsatzsteuererklärung abgeben.
Selbst Kleinunternehmer, die nicht zur Zahlung der Umsatzsteuer verpflichtet sind, müssen eine jährliche Umsatzsteuererklärung abgeben.
Die Steuerpflichtigen sind grundsätzlich verpflichtet, bis zum 10. Tag nach Ablauf des Kalenderquartals, in dem die Umsatzsteuer entsteht, eine Umsatzsteuer-Voranmeldung abzugeben (§ 18 (2) UStG).
Betrug die Umsatzsteuerschuld jedoch im Vorjahr mehr als 6.136 € oder handelt es sich um ein neugegründetes Unternehmen, ist der Unternehmer zur Abgabe monatlicher Umsatzsteuer-Voranmeldungen jeweils spätestens bis zum 10. Tag nach Ablauf des betreffenden Monats verpflichtet.
Der Steuerpflichtige kann sich auch für die Abgabe monatlicher Umsatzsteuer-Voranmeldungen entscheiden, wenn er im Vorjahr einen Betrag über 6.136 € erstattet bekommen hat (§ 18 (2a) UStG).
Hat die jährliche Umsatzsteuer nicht mehr als 512 € betragen, sind weder Umsatzsteuer-Voranmeldungen noch Umsatzsteuererklärungen abzugeben.
Eine sich ergebende Umsatzsteuerschuld ist 10 Tage nach Ablauf des Besteuerungszeitraums zu bezahlen.
Eine Rückerstattung kann zu einer gründlicheren Prüfung durch die Finanzverwaltung führen.
(Gesetzliche) Fristen für die Bearbeitung von Umsatzsteuererklärungen und für die Rückerstattung von Umsatzsteuerüberzahlungen bestehen nicht.
Um eine zeitnahe Rückerstattung zu erhalten, kann eine Sicherheit für den Rückerstattungsbetrag angeboten werden.
Grundsätzlich sind Umsatzsteuererklärungen, auch Erstattungsfälle, zügig zu bearbeiten.
Das deutsche Umsatzsteuergesetz ermöglicht eine Verlängerung dieser (Abgabe- bzw. Zahlungs-)Fristen (§ 18 (6) UStG).
Die Fristverlängerung muss gesondert beantragt werden.


\pagebreak

\subsection*{Translation:}



In accordance with the provisions of Section 37 (1) of Law No. 337/1992, the FA may repeatedly impose a fine of up to CZK 2 million on persons who fail to comply with their non-monetary obligations under this or a special tax law or decision under this Act ,

Automated tax information system - Automated tax information system used by the Czech tax authorities to provide technical support to tax collection.
During its audit, the NKÚ found that the FAO seldom made use of the possibility of imposing a fine on the relevant taxpayer in the event of non-submission of the VAT return.


Submission and processing of VAT returns in Germany

In Germany, a distinction is made between monthly or quarterly VAT returns and the annual VAT return.
Those taxpayers who are required to submit advance VAT returns must nevertheless submit an annual VAT return.
Even small business owners who are not required to pay VAT must submit an annual VAT return.
The taxpayers are generally obligated to submit an advance sales tax return by the 10th day after expiry of the calendar quarter in which the value added tax arises (§ 18 (2) UStG).
In the previous year, however, if the VAT liability amounted to more than € 6,136 or if it is a newly founded company, the entrepreneur is obliged to submit monthly advance VAT returns no later than the 10th day after the end of the respective month.
The taxpayer may also decide to submit monthly advance VAT returns if he has been reimbursed in the previous year for an amount of € 6,136 (§ 18 (2a) UStG).
If the annual sales tax does not exceed 512 €, neither VAT returns nor VAT returns are required.
A resulting VAT liability is payable 10 days after the end of the taxable period.
A refund can lead to a more thorough audit by the tax authorities.
(Legal) deadlines for the processing of VAT returns and for the refund of VAT overpayments do not exist.
In order to receive a timely refund, a security can be offered for the refund amount.
In principle, VAT returns, including reimbursement cases, must be processed quickly.
The German sales tax law allows for an extension of these (submission or payment) periods (§ 18 (6) UStG).
The extension must be requested separately.


\end{document}
