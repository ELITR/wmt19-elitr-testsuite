\documentclass[10pt]{article}

%\usepackage{times}
\usepackage{fancyhdr}

\pagestyle{fancy}
\fancyhf{}
\rhead{\textbf{Code: artist}}

\begin{document}

\subsection*{Source:}

Der BRH betrachtet diese Mittel der Amtshilfe in Steuersachen zur Überprüfung der Umsatzsteuererklärung bzw. der Einhaltung der Steuerpflicht durch einen nicht-ansässigen Busunternehmer als zu aufwendig im Verhältnis zu dem geringen Aufkommen aus der Umsatzsteuer auf steuerpflichtige Personenbeförderungen.
Insbesondere die Prüfung der Bücher eines nicht-ansässigen Steuerpflichtigen verursacht unverhältnismäßig hohe Kosten für den Steuerpflichtigen selbst und für die Finanzverwaltung.
Die Letztere muss die ausländischen Bücher prüfen, wofür nicht nur die Kenntnis der Fremdsprache, sondern auch der ausländischen Buchhaltungsgrundsätze erforderlich ist.
Dementsprechend ist bei einer solchen Prüfung mit einem höheren Zeitaufwand als bei einem gebietsansässigen Steuerpflichtigen zu rechnen, obwohl sich die Prüfung auf die in Deutschland stattfindenden Personenbeförderungen beschränkt.
Andererseits muss der nichtansässige Steuerpflichtige nicht nur die Prüfungsrechte seiner nationalen Finanzverwaltung hinnehmen, sondern auch diejenigen der anderen Mitgliedstaaten, soweit er dort Personenbeförderungen durchführt.
Außerdem muss der Steuerpflichtige für jeden Mitgliedstaat, in dem auch nur eine einzige Personenbeförderung durchgeführt wird, alle bestehenden steuerrechtlichen Pflichten in den verschiedenen Sprachen und anhand der Formulare der Finanzverwaltungen der betreffenden EG-Mitgliedstaaten erfüllen.
Damit führt die korrekte Besteuerung grenzüberschreitender Personenbeförderungsleistungen zu einem unverhältnismäßig hohen Verwaltungsaufwand.


Vergleich und Evaluierung

Die Rechtslage ist in den beiden Staaten unterschiedlich.
In der Tschechischen Republik ist die Personenbeförderung zwischen verschiedenen EG-Mitgliedstaaten sowie zwischen den Mitgliedstaaten und Drittländern aufgrund des Gesetzes zum EG-Beitritt der Tschechischen Republik von der Umsatzsteuer befreit.
In Deutschland ist die Beförderungsleistung entsprechend der auf deutsches Staatsgebiet entfallenden Beförderungsstrecke umsatzsteuerpflichtig.
Erbringt ein nicht-ansässiger Steuerpflichtiger erstmalig eine solche Beförderungsleistung, so muss er dies vorher beim zuständigen FA anmelden.
Dieses registriert den Steuerpflichtigen, der den steuerpflichtigen Anteil der Beförderungsleistung in einer Steuererklärung in Deutschland deklarieren muss.
Der Gesetzesvollzug in Deutschland ist sehr zeitaufwendig.
Mindestens drei von vier beim FA Chemnitz-Süd registrierten tschechischen Steuerpflichtigen haben die Personenbeförderung als Geschäftszweck.
Das betreffende Steueraufkommen ist im Verhältnis zum Verwaltungsaufwand recht niedrig.


\pagebreak

\subsection*{Translation:}

The BRH considers these means of administrative assistance in tax matters for the verification of the VAT return or the compliance with the tax obligation by a non-resident bus operator to be too expensive in relation to the small amount of value added tax on personal passenger transport.
In particular, the audit of the books of a non-resident taxpayer causes disproportionate costs for the taxpayer itself and for the tax authorities.
The latter has to check the foreign books, which requires not only the knowledge of the foreign language but also the foreign accounting principles.
Accordingly, such a test can be expected to cost more time than a resident taxpayer, even though the test is limited to passenger transport in Germany.
On the other hand, the non-resident taxpayer must not only accept the audit rights of his national treasury, but also those of the other Member States, insofar as he carries out passenger transport there.
In addition, for each Member State in which even a single passenger service is carried out, the taxable person must fulfill all existing tax obligations in the various languages and the forms of the financial administrations of the EC Member States concerned.
Thus, the correct taxation of international passenger transport services leads to a disproportionate administrative burden.


Comparison and evaluation

The legal situation is different in the two states.
In the Czech Republic, passenger transport between EC Member States and between Member States and third countries is exempt from VAT on the basis of the Act of Accession of the Czech Republic.
In Germany, the transport service is subject to VAT in accordance with the distance covered by German national territory.
If a non-resident taxpayer furnishes such a transport service for the first time, he must first register this with the competent FA.
It registers the taxpayer, who must declare the taxable part of the transport service in a tax declaration in Germany.
The law enforcement in Germany is very time consuming.
At least three out of four Czech taxpayers registered with FA Chemnitz-Süd have passenger transport as their business purpose.
The tax revenue in question is relatively low in relation to the administrative burden.


\end{document}
