\documentclass[10pt]{article}

%\usepackage{times}
\usepackage{fancyhdr}

\pagestyle{fancy}
\fancyhf{}
\rhead{\textbf{Code: emotion}}

\begin{document}

\subsection*{Source:}

Durch eine Anfrage beim zuständigen Finanzamt in der Tschechischen Republik stellte der NKÚ fest, dass der tschechische Steuerpflichtige im Jahr 2004 ergänzende Umsatzsteuererklärungen für die einzelnen (unterjährigen) Besteuerungszeiträume abgegeben hatte, in denen er einen höheren Wert der Erwerbe aus anderen Mitgliedstaaten erklärte und gleichzeitig den Abzug von Vorsteuer aus den Erwerben aus den anderen Mitgliedstaaten begehrte.
Der zuständige Sachbearbeiter des tschechischen Finanzamts konnte anhand dieser zusätzlichen Umsatzsteuererklärungen nicht eindeutig feststellen, ob in den erklärten Erwerben auch der Wert der Lieferungen des o.g. deutschen Lieferanten enthalten war.
In Beantwortung eines Auskunftsersuchens des NKÚ stellte der BRH fest, dass der deutsche Lieferant im September 2003 als umsatzsteuerpflichtig registriert worden war.
Je-doch hatte er bei der Registrierung nicht die Zuteilung einer USt-IdNr., sondern nur einer aus- schließlich für Deutschland geltenden Steuernummer beantragt.
Obwohl es in Deutschland in der Regel nicht möglich ist, einem Steuerpflichtigen rückwirkend eine USt-IdNr. zuzuteilen, tat dies 2006 der zuständige Sachbearbeiter der deutschen Finanzverwaltung rückwirkend zum 1. Mai 2004.
Das deutsche CLO teilt eine USt-IdNr. rückwirkend zu, wenn der deut sche Unternehmer die verzögerte Zuteilung nicht verursacht hat; z.B. wenn das CLO die Verzögerung nach Antrag des Steuerpflichtigen auf Erteilung einer USt-IdNr. verursachte.
In diesem Fall erteilt das CLO die USt-IdNr. rückwirkend, weil der Unternehmer nach § 6 UStG für eine innergemeinschaftliche Lieferung keine gültige USt-IdNr. benötigt.
Umgekehrt benötigt ein deutscher Erwerber nicht die (gültige) USt-IdNr. seines Lieferanten, um den innergemeinschaftlichen Erwerb zu erklären.
Wegen dieser Verfahrensunterschiede zwischen den Mitgliedstaaten erteilte das deutsche CLO die USt-IdNr. in diesem besonderen Falle rückwirkend.
2.
Im Rahmen des internationalen Auskunftsaustauschs ersuchte der tschechische Finanzbeamte um die Überprüfung der umsatzsteuerlichen Registrierung des deutschen Unternehmers.
Aus den MIAS-Daten ergibt sich als Tag der Registrierung der 5.  März 2005, während in einer vom örtlich zuständigen deutschen FA erteilten Registrierungsbescheinigung der 1. Oktober 2004 angegeben ist.
Die deutschen Behörden teilten in ihrer Antwort mit, der deutsche Steuerpflichtige sei am 1. Oktober 2004 registriert worden, habe jedoch aus organisatorischen Gründen erst am 16. Februar 2005 in Deutschland eine USt-IdNr. zugeteilt bekommen.


\pagebreak

\subsection*{Translation:}

By requesting the competent tax office in the Czech Republic, the NKÚ stated that the Czech taxpayers had submitted additional VAT returns in 2004 for the individual (sub-annual) tax periods in which they declared a higher value of acquisitions from other Member States and at the same time deducted the tax from input tax from purchases from other Member States.
On the basis of these additional sales tax returns, the responsible clerk of the Czech tax office was unable to determine unequivocally whether the declared purchases included the value of the deliveries of the abovementioned German supplier.
In response to a request for information from the NKÚ, the BRH found that the German supplier had been registered as a VAT payer in September 2003.
However, when registering, he had not applied for the allocation of a VAT number, but only for a tax number that applied exclusively to Germany.
Although it is generally not possible in Germany to pay a taxpayer retroactively a VAT ID number. In 2006, the responsible administrator of the German tax authorities did so with retroactive effect from 1 May 2004.
The German CLO shares a VAT number. retroactively if the German entrepreneur did not cause the delayed allocation; For example, if the CLO defers the delay after request of the taxpayer to issue a VAT ID. caused.
In this case, the CLO issues the VAT ID number. retroactive, because the entrepreneur does not have a valid VAT ID for intra-Community delivery under § 6 UStG. needed.
Conversely, a German purchaser does not need the (valid) VAT ID. his supplier to explain the intra-Community acquisition.
Because of these procedural differences between the Member States, the German CLO issued the VAT ID number. retroactive in this particular case.


Second

As part of the international exchange of information, the Czech tax office requested the verification of the VAT registration of the German entrepreneur.
From the VIES data, the date of registration is 5 March 2005, while in a registration certificate issued by the locally competent German FA the date is 1 October 2004.
The German authorities stated in their reply that the German taxpayer was registered on 1 October 2004, but for organizational reasons did not have a VAT ID number on 16 February 2005 in Germany. get assigned.


\end{document}
