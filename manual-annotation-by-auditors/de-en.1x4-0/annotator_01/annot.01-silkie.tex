\documentclass[10pt]{article}

%\usepackage{times}
\usepackage{fancyhdr}

\pagestyle{fancy}
\fancyhf{}
\rhead{\textbf{Code: silkie}}

\begin{document}

\subsection*{Source:}

Darüber hinaus sind die Rechnungshöfe der Auffassung, dass die Statistiken sowohl die jährliche Anzahl der Ersuchen als auch die Art der Forderungen (z. B. USt, Einkommensteuer) und die Höhe der pro Mitgliedstaat beigetriebenen Beträge ausweisen sollten.
Durch eine genauere Analyse der durch die Mitgliedstaaten vorgelegten Statistiken könnte die Kommission die Mitgliedstaaten dabei unterstützen, ihre Maßnahmen zur Bekämpfung des USt-Betrugs zu bewerten.
Da die Beitreibungsquote pro Mitgliedstaat berechnet werden kann, sprechen sich die Rechnungshöfe dafür aus, dass die Mitgliedstaaten mit niedriger Quote wenigstens die durchschnittliche Beitreibungsquote innerhalb der EU anstreben sollten.
Auf Wunsch eines Mitgliedstaats könnte die Kommission dabei eine aktive Rolle spielen.
Die beim Finanzministerium CZ eingegangenen ausländischen Beitreibungsersuchen wurden insbesondere in den Jahren 2006 und 2007 verzögert an die zuständigen FÄ weitergeleitet.
Gleiches gilt für die von den tschechischen FÄ gestellten Beitreibungsersuchen.
Die Prüfung ergab, dass das tschechische Finanzministerium insoweit gegen das Gesetz Nr. 191/2004 über die zwischenstaatliche Beitreibungshilfe verstieß, als es die zuständige Behörde des ersuchenden Mitgliedstaats nicht innerhalb der Sechsmonatsfrist ab Eingangsbestätigung des Ersuchens über den Fortschritt der Beitreibungsmaßnahmen unterrichtete.


4.5 Umsatzsteuer-Prüfung bei Großbetrieben



4.5.1 Umsatzsteuer-Prüfung bei Großbetrieben in der Tschechischen Republik



Allgemeines

Das Besteuerungsverfahren für Großbetriebe unterliegt denselben Regeln, die auch für andere Steuerpflichtige gelten.
Die Prüfung von Großbetrieben ist im tschechischen Gesetz Nr. 337/1992 geregelt, dem „Steuer- und Gebührenverwaltungsgesetz“.
Das tschechische Finanzministerium definiert Großbetriebe als natürliche Personen, die einen Geschäftsbetrieb errichtet haben bzw. juristische Personen, die zur Ausübung eines Geschäftsbetriebs gegründet worden sind und deren Nettoumsatz für den betreffenden Besteuerungszeitraum laut Einkommen- bzw. Körperschaftsteuererklärung mindestens 2 Mrd. CZK beträgt.
Der Steuerpflichtige wird als Großbetrieb anhand der Umsatzhöhe lt. Einkommen- oder Körperschaftsteuererklärung im jeweiligen Besteuerungszeitraum eingestuft; erstmals für das Jahr 2003.


\pagebreak

\subsection*{Translation:}

Moreover, the SAIs believe that the statistics should show both the annual number of requests and the nature of the claims (eg VAT, income tax) and the amounts recovered per Member State.
By further analyzing the statistics provided by Member States, the Commission could assist Member States in assessing their anti-VAT fraud policies.
As the recovery rate per Member State can be calculated, the SAIs support the idea that low quota Member States should at least strive for the average recovery rate within the EU.
At the request of a Member State, the Commission could play an active role.
The foreign claims for recovery received at the Ministry of Finance CZ were delayed, especially in the years 2006 and 2007, to the responsible FÄs.
The same applies to the requests for recovery made by the Czech FZ.
The audit revealed that the Czech Ministry of Finance was in breach of Law No 191/2004 on intergovernmental recovery aid in that it did not notify the competent authority of the requesting Member State within six months from the acknowledgment of receipt of the request for recovery.


4.5 VAT inspection at large companies



4.5.1 Sales tax audit of large companies in the Czech Republic



General

The taxation procedure for large companies is subject to the same rules as other taxable persons.
The audit of large enterprises is regulated by the Czech Law No. 337/1992, the "Tax and Charges Administration Act".
The Czech Ministry of Finance defines large enterprises as natural persons who have set up a business or legal persons established for the purpose of conducting business operations and whose net turnover for the tax period concerned is at least CZK 2 billion according to the income tax or corporation tax return.
The taxpayer is classified as a large enterprise on the basis of the amount of turnover according to income or corporate income tax return in the respective tax period; first time for the year 2003.


\end{document}
