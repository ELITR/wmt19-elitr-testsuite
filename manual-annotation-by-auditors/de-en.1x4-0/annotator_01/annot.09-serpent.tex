\documentclass[10pt]{article}

%\usepackage{times}
\usepackage{fancyhdr}

\pagestyle{fancy}
\fancyhf{}
\rhead{\textbf{Code: serpent}}

\begin{document}

\subsection*{Source:}



Zusammenfassende Evaluierung und Empfehlungen

Die parallelen Prüfungen des NKÚ und des BRH haben sich auf zwei Schwerpunkte gerichtet, die in der Tschechischen Republik und in Deutschland wesentlich sind für die Vergabe von Bauleistungen durch öffentliche Auftraggeber: Die Anwendung der EU-Vergabevorschriften und die Korruptionsvorbeugung.
Durch die Vorgaben des EU-Rechts wird ein grenzüberschreitender Wettbewerb zur Schaffung eines europäischen Binnenmarktes ermöglicht.
Korruptionsvorbeugende Maßnahmen dienen nicht nur dazu, dass Korruption vermieden oder zumindest vermindert wird. Sie sollen auch wirtschaftlichen Schaden und einen möglichen Vertrauensverlust in die Integrität des Staatswesens abwenden.
Die Prüfungserkenntnisse der beiden Rechnungshöfe zur Anwendung der vergaberechtlichen Regelungen in beiden Ländern sowie der Vergleich dieser Regelungen ergeben Aufschlüsse zu Gemeinsamkeiten und Unterschieden, die für beide Länder bei der Fortentwicklung ihres Vergaberechts und dessen Anwendung hilfreich sein können.


Anwendung der EU-Vergabevorschriften

Die Vorgaben der EU zur Vergabe von Bauleistungen haben beide Länder in nationales Recht umgesetzt.
Damit gibt es in der Tschechischen Republik und in Deutschland eine wichtige gemeinsame Grundlage für die Vergabe von Bauleistungen und die Berücksichtigung der hierfür wesentlichen Gebote des Wettbewerbs, der Gleichbehandlung der beteiligten Unternehmen und der Transparenz der Vergabeverfahren.
Die Bauverwaltungen beider Länder haben diese Gebote teilweise sehr umfangreich durch weitere Verwaltungsvorschriften untersetzt.
Die Tschechische Republik hat die Vorgaben zu den EUweit zu vergebenden Bauleistungen weitgehend auch auf ihre national regelbaren Vergabeverfahren unterhalb des EU-Schwellenwertes übertragen.
Die Untersuchungen der Rechnungshöfe haben aber ergeben, dass die Bauverwaltungen die Vorgaben des EU-Rechts nicht in allen Bereichen vollständig beachtet haben.
Die Bauverwaltung in Deutschland sollte die Qualität ihrer Vergabeverfahren weiter verbessern, um die aufgezeigten Beanstandungen beim Umsetzen der EU- Vorgaben beseitigen zu können.
Die Prüfungen in der Tschechischen Republik haben vor allem Gefahren aufgezeigt, die die Wirtschaftlichkeit der Vergaben beeinträchtigen.
So machten unvollständige Leistungsbeschreibungen zusätzliche Leistungen notwendig, die häufig unter Verstoß gegen Rechtsvorschriften vergeben wurden.  Zudem waren die Kostenermittlungen meist unzureichend erstellt.  Auch verringerten überzogene Qualifikationsanforderungen die Zahl möglicher Bieter und schränkten insbesondere bei den Verkehrsbauten den Wettbewerb ein.


\pagebreak

\subsection*{Translation:}



Summarised evaluation and recommendations

The parallel audits of the NKU and the BRH focused on two issues which are essential both in the Czech Republic and in Germany for the awarding of public works contracts: the application of EU procurement law and the prevention of corruption.
The provisions of EU law provide the opportunity for international competition within the EU in order to create a European Single Market.
Measures designed to prevent corruption do not only aim at avoiding or at least reducing corruption. They are also to prevent financial damage and potential loss of confidence in the integrity of government.
The audit findings generated by the two SAIs with respect to the application of procurement law in both countries and the comparison of the respective provisions of procurement law provide an insight into the common and differing features that may be helpful for both countries in further developing their procurement law and its application.


Application of EU procurement law

Both countries have transposed into national law the EU legislation concerning the awarding of public works contracts.
Thus, an important common basis exists in both countries for the awarding of public works contracts and the compliance with the essential requirements in this field: competition; equal treatment of the participating enterprises and transparency of the contract award procedure.
The construction administrations of both countries have underpinned these requirements by partly very extensive administrative regulations.
The Czech Republic has largely adopted the provisions on public works contracts requiring EU-wide tendering for those contract award procedures that are below the EU threshold and are therefore governed by national law.
However, the SAIs‘ audits revealed that the construction administrations did not always fully comply with all relevant provisions of EU-law.
The construction administration in Germany should continue to improve the quality of its contract award procedures in order to eliminate the deficiencies pointed out with respect to the implementation of EU requirements.
In the Czech Republic, the audits furthermore identified the risks impacting on the value for money achieved in contract award procedures.
For instance, incomplete specification of the public contract subject, resulting in need of extra works, often awarded in contradiction to the law. Furthermore, insufficient specification of the expected price was provided. Also, qualification requirements limited the possible number of bidders, thus restricting competition, namely in the area of transport constructions.


\end{document}
