\documentclass[10pt]{article}

%\usepackage{times}
\usepackage{fancyhdr}

\pagestyle{fancy}
\fancyhf{}
\rhead{\textbf{Code: dealer}}

\begin{document}

\subsection*{Source:}

In Einzelfällen wurden Bekanntmachungen auf nationaler Ebene früher veröffentlicht oder enthielten mehr Informationen – z. B. zu den Ausführungsfristen – als die Bekanntmachungen im Amtsblatt der Europäischen Union.


Ex-ante-Transparenz

Die Vergabestellen einer geprüften Bauverwaltung hatten ein Dienstleistungsunternehmen mit dem Veröffentlichen der nationalen Vergabeverfahren beauftragt.
Dieses veröffentlichte die Meldungen auf seiner eigenen kostenpflichtigen Vergabeplattform.
Die Vergabestellen überprüften aber nicht, ob das Unternehmen die Meldungen wie vorgeschrieben auch an das kostenfreie Internetportal des Bundes weitergeleitet hatte.
Stichproben ergaben, dass nur ein Drittel der Meldungen der geprüften Verwaltung auf diesem Portal veröffentlicht wurde.


NKÚ



Bekanntmachungen

Beim Bau einer Kläranlage gab die Vergabestelle in ihrer Bekanntmachung nicht die Kriterien an, aufgrund derer sie die Bieter auswählen wollte.
Die Vergabestelle konnte nicht nachweisen, wie die Bieter ausgewählt wurden.
Die Beschränkung der Zahl der Bieter war nach dem damals geltenden Recht zulässig.
Beim Bau eines Autobahnrings gab die Vergabestelle bei einem nicht offenen Verfahren weder in der Bekanntmachung noch in der Dokumentation des Vergabeverfahrens näher an, wonach sie die Qualifikation der Bewerber bewerten und wieviele Bewerber sie zum Wettbewerb zulassen wollte.
Die Vergabestelle kam bei ihrer Bewertung zum Ergebnis, dass alle sechs Bewerber den Qualifikationsanforderungen entsprachen, schloss den Bewerber mit den schlechtesten Bewertungen jedoch vom weiteren Verfahren aus.
Die Vorgehensweise der Vergabestelle war nicht transparent, weil sie zu keinem Zeitpunkt des Verfahrens Angaben zu den Qualifikationskriterien und zum Bewertungsverfahren bekannt gegeben hatte.


Veröffentlichung der Bekanntmachungen

Bei der Sanierung ehemaligen Militärgeländes veröffentlichte die Vergabestelle die Ausschreibung nicht.


Zusammenfassende Feststellungen

Das zweistufige Unterrichtungssystem mit Vorinformation und Bekanntmachung ist ein wichtiger Bestandteil, um das Transparenzgebot bei den europaweiten Vergaben umzusetzen, den Wettbewerb zu stärken und damit auch Korruption vorzubeugen.


Beide Länder haben die EU-Vorgaben umgesetzt und die Transparenz ihrer Vergabeverfahren auch im nationalen Vergabebereich erhöht:



\pagebreak

\subsection*{Translation:}

In isolated cases, the contract notices published at national level included more information – e.g. concerning performance deadlines – or were published earlier than the notices published in the European Union’s Official Journal.


Ex-ante transparency

The units of one construction administration had commissioned a service company with publishing the national notices.
The latter published the notices on its own procurement platform for the use of which fees are charged.
The construction authorities had however not checked whether this company had complied with the requirement to pass on the notices to the Federal Government’s free internet portal.
Sample audits showed that only one third of the notices had been published on the latter portal.


NKÚ



Contract notice

In case of the construction of a water treatment plant, the contracting authority in its publication of award procedure to suppliers did not indicate the criteria it intended to apply in selecting tenderers.
The contracting authority was not able to prove the way of bid selection.
Reduction of number of bidders was, according to previous legislation, duly in place.
In case of a road circle construction the contracting authority did not specify the figures for evaluation of compliance with qualification criteria and did not indicate the number of bidders to be invited nor in publication of restricted procedure opening neither in the contract award documentation.
The contracting authority evaluated the qualification requirements of all 6 bidders in the restricted procedure as being complied with, but one bidder was excluded because he met the qualification requirements on the last sixth position.
These steps of contracting authority were not transparent, because no information concerning the qualification particulars evaluation and the method of evaluation had ever been published.


Publication of notices

In case of ex-military areas reconstructions, the contracting authority did not publish the written invitation to tender.


Conclusions and recommendations / summarised findings

The two-stage information procedure with prior information and contract notice is an important instrument for implementing the transparency requirement that applies to EU-wide tendering procedures and thus for enhancing competition and preventing corruption.
By extending the obligation to publish prior information to national tendering procedures in the Czech Republic and by applying this obligation in Germany in cases of open and restricted tendering, both countries have implemented EU requirements and thus increased the transparency of national tendering procedures.


\end{document}
