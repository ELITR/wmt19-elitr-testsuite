\documentclass[10pt]{article}

%\usepackage{times}
\usepackage{fancyhdr}

\pagestyle{fancy}
\fancyhf{}
\rhead{\textbf{Kód: dotace}}

\begin{document}

\subsection*{Zdroj:}

In Germany, contracts below the EU threshold are awarded in accordance with the national procedures (public invitation to tender, restricted invitation to tender, negotiated contract), which, however, are aforementioned procedures. largely identical with the EU-wide contract award.
Public procurement law in Germany is characterised by the principle of preference for the open procedure (where tenders have to be invited EU-wide) and for the public invitation to tender (in case of merely national contract awards).
Under the open procedure and the public invitation to tender, every interested enterprise may submit a bid.
Another essential feature of German procurement law is the promotion of small and medium-sized enterprises (SMEs).
This is achieved mainly by dividing an overall public works project in trade-specific lots and part lots.
By awarding separate contracts for these lots, as many enterprises as possible, are to be given the chance to compete.
The two aforementioned characteristics of German procurement are time-tested tools for obtaining good value for money.
More than 330,000 construction firms with workforce of 1.9 mil. are registered in Germany.
In the Czech Republic, the EU-wide contract award procedures are also to be applied to public works contracts below the EU threshold.
An exception to this is the simplified procedure below threshold, where the contracting authorities may under certain conditions use considerably less strict procedure.
In agreement with EU law, Czech procurement law includes provisions according to which contracts are to be concluded, if competition is guaranteed and an advantageous price is obtained.
The present Czech legislation enables contracting authorities to choose between open and restricted procedure.
In the Czech Republic, there are about 85,000 registered building companies, out of it 1,900 building contractors with 20 or more workers each.


Aggregate volume of public works contracts awarded by contracting agencies and shares of contract award types in the two countries

In 2011, the total volume of construction works in Germany totalled € 307 billion.
Of these, € 45.5 billion (equivalent to 15 per cent) were accounted for by public contracts which, apart from contracts awarded by the Federal Government, included contracts awarded by the Federal States and municipalities.
The total volume of federal building and road construction projects (without railway construction and hydraulic engineering) amounted to € 8.5 billion.


\pagebreak

\subsection*{Překlad:}

V Německu jsou zakázky, které nedosahují prahové hodnoty EU, zadávány v souladu s vnitrostátními postupy (veřejné nabídkové řízení, omezené nabídkové řízení, sjednaná zakázka), které jsou však výše uvedenými postupy, které jsou do značné míry totožné s udělením zakázky v celé EU.
Zákon o zadávání veřejných zakázek v Německu je charakterizován zásadou upřednostňování otevřeného řízení (kde musí být nabídky pozvány v celé EU) a veřejného nabídkového řízení (v případě pouhého vnitrostátního zadání zakázky).
V rámci otevřeného řízení a veřejného nabídkového řízení může každý zúčastněný podnik podat nabídku.
Dalším zásadním rysem německého zákona o zadávání veřejných zakázek je podpora malých a středních podniků.
Toho se dosáhne zejména rozdělením celkového projektu veřejných prací na jednotlivé části a dílčí části a dílčí části části jednotlivých dílčích dílčích dílčích dílčích dílčích dílčích dílčích dílčích dílčích dílčích dílčích dílčích dílčích dílčích dílčích dílčích dílčích dílčích dílčích dílčích dílčích dílčích dílčích dílčích dílčích dílčích dílčích dílčích dílčích dílčích dílčích dílčích dílčích dílčích dílčích dílčích dílčích dílčích dílčích dílčích dílčích dílčích dílčích dílčích dílčích dílčích dílčích dílčích dílčích dílčích částí.
Zadáním samostatných zakázek na tyto části bude mít šanci soutěžit co nejvíce podniků.
Dvě výše uvedené charakteristiky německých veřejných zakázek jsou časově ověřenými nástroji pro získání dobré hodnoty za peníze.
V Německu je registrováno více než 330 000 stavebních podniků s 1,9 mil. zaměstnanců.
V České republice se zadávací řízení na veřejné zakázky na stavební práce, které nedosahují prahové hodnoty EU, uplatňují také v celé EU.
Výjimkou je zjednodušený postup pod prahovou hodnotou, kdy veřejní zadavatelé mohou za určitých podmínek použít podstatně méně přísný postup.
Ve shodě s právem EU obsahuje české právo zadávání veřejných zakázek ustanovení, podle nichž se uzavírají smlouvy, je-li zaručena hospodářská soutěž a je-li získána výhodná cena.
Stávající české právní předpisy umožňují veřejným zadavatelům vybrat si mezi otevřeným a omezeným řízením.
V České republice je přibližně 85 000 registrovaných stavebních firem, z toho 1 900 stavebních dodavatelů s 20 a více pracovníky.


Souhrnný objem veřejných zakázek na stavební práce zadávaných zadavateli a podílů na typech zakázek v obou zemích

V roce 2011 činil celkový objem stavebních prací v Německu 307 miliard EUR.
Z toho 45,5 miliardy EUR (tj. 15 \%) připadalo na veřejné zakázky, které kromě zakázek zadaných spolkovou vládou zahrnovaly zakázky zadané spolkovými státy a obcemi.
Celkový objem federálních stavebních a silničních stavebních projektů (bez železniční výstavby a hydraulického inženýrství) činil 8,5 miliardy EUR.


\end{document}
