\documentclass[10pt]{article}

%\usepackage{times}
\usepackage{fancyhdr}

\pagestyle{fancy}
\fancyhf{}
\rhead{\textbf{Kód: pavouk}}

\begin{document}

\subsection*{Zdroj:}

Besides pocket money, asylum seekers accommodated in private outside refugee facilities may receive an allowance up to the subsistence minimum.
Section 84 of Ac, no. 325/1999 Coll., on asylum, as amended, awards municipalities the entitlement to an allowance to cover costs associated with refugee facilities in their territory.
The size of the allowance is fixed every year by government resolution, and amounted to CZK 7 per person per day of stay in the period under scrutiny.
At the same time, municipalities are entitled to an allowance to cover non-investment costs of elementary schools attended by asylum seekers.
The size of the allowance in the period under scrutiny was CZK 120 for every school-going asylum seeker on the condition that the person in question attended school for at least 10 days in the given month.
In the Slovak Republic food allowance was SKK 70 per person per day in the period under scrutiny; pocket money was SKK 12 per person per day and SKK 8 for a child aged up to 15.
The average cost of housing one asylum seeker in an refugee facility per day was SKK 442 in 2001, SKK 447 in 2002 and SKK 445 in 2003.
Under Act no. 480/2002 Coll. (SR), on asylum, municipalities with refugee facilities in their territory receive an allowance to cover part of the municipality's expenditure associated with establishing and running an refugee facility.
So far municipalities have only made partial use of the possibility of applying for the allowance, which is fixed at SKK 1 500 per bed per calendar year.
The national legislation in both countries lays down a 90-day deadline, from the start of asylum proceedings, in which a decision must be issued. In warranted cases this deadline may be extended.
The most common reason for asylum proceedings overstepping this deadline, which has an influence on state budget expenditure, is appeals lodged against rulings not to award asylum.
In the Czech Republic most official decisions are issued after a period of 90 to 180 days (31.54\%).
The second greatest number of decisions is issued 90 days or less after the asylum application was lodged (17.61 \%).
In total, 73\% of asylum proceedings opened in 2001-2003 were officially ruled on a year or less after proceedings commenced.


\pagebreak

\subsection*{Překlad:}

Mimo kapesné může být žadatelům o azyl ubytovaným v pobytových střediscích se samostatným vařením a žadatelům ubytovaným v soukromí mimo azylová zařízení poskytován příspěvek až do výše životního minima.
Zákon č. 325/1999 Sb., o azylu, ve znění pozdějších předpisů mimo jiné přiznává v § 84 obcím nárok na příspěvek na úhradu nákladů vynaložených v souvislosti s azylovým zařízením na jejím území.
Výše příspěvku byla každoročně stanovena usneseními vlády a v kontrolovaném období činila 7 Kč za osobu a pobytový den.
Současně je obcím přiznán nárok na příspěvek na úhradu neinvestičních nákladů základních škol, které navštěvují žadatelé o udělení azylu.
Výše příspěvku byla v kontrolovaném období 120 Kč na jednoho žáka a měsíc za podmínky, že v daném měsíci dotyčný navštěvoval školu po dobu minimálně 10 dní.
Ve Slovenské republice bylo v kontrolovaném období stravné 70 Sk na osobu a den, kapesné 12 Sk na osobu a den a 8 Sk na dítě do 15 let věku.
Náklady na jednoho žadatele o azyl na jeden pobytový den byly v průměru 442 Sk v roce 2001, 447 Sk v roce 2002 a 445 Sk v roce 2003.
Obcím, na jejichž území se nachází azylové zařízení, je v souladu se zákonem č. 480/2002 Z. z., o azyle, poskytován příspěvek na částečnou úhradu výdajů, které obec vynaloží v souvislosti se zřízením a činností azylového zařízení.
Obce zatím využívaly pouze částečně možnost požádat o příspěvek, který je stanoven ve výši 1 500 Sk na lůžko a kalendářní rok.
Národní legislativa v obou zemích pro vydání rozhodnutí o azylu stanovuje 90denní lhůtu od zahájení řízení o udělení azylu a zároveň připouští v odůvodněných případech prodloužení této lhůty.
Nejčastějším důvodem prodloužení řízení o udělení azylu, což ovlivňuje výši výdajů státního rozpočtu, je podání opravných prostředků proti rozhodnutí o neudělení azylu.
V České republice byl nejvyšší počet pravomocných rozhodnutí vydán v intervalu mezi 90 až 180 dny (31,54\%).
Druhý nejvyšší počet byl v intervalu do 90 dnů ode dne podání žádosti o udělení azylu (17,61\%).
Celkem bylo 73\% azylových řízení zahájených v letech 2001-2003 pravomocně rozhodnuto do jednoho roku od jejich zahájení.


\end{document}
