\documentclass[10pt]{article}

%\usepackage{times}
\usepackage{fancyhdr}

\pagestyle{fancy}
\fancyhf{}
\rhead{\textbf{Kód: pahorek}}

\begin{document}

\subsection*{Zdroj:}



- 8525 20 99 Transmission apparatus incorporating reception apparatus: Others,



- 8529 90 40 Parts and components suitable for use solely or primarily with apparatus falling under headings 8525 to 8528

Others: Parts and components of apparatus falling under subheadings 8525 10 50, 8525 20 91, 8525 20 99, 8525 40 11 and 8527 90 92.
SAD items from the Customs Databases of the Czech and Slovak Republics were mutually assigned on the basis of matching contents of different columns of the SAD (e.g. invoiced price, quantity, weight, license plate number of the vehicle carrying the consignment, border crossing date, Customs Tariff heading etc.).
Results of the comparison are presented in the following tables and charts.
The process of matching individual export and import transactions was complicated, inter alia, by the fact that export and import transactions had not been declared by the parties in the same way.
So, for example, a Czech export transaction made use of a single SAD, while the corresponding import transaction on the Slovak side made use of multiple SADs, there were differences in the numbers of items between an export SAD and its matching import SAD etc.
Additional factors affecting the rate of success of the matching or assignment process was the structure of data in the Czech and Slovak Customs Databases, as well as the period of time elapsed between the registration of an export transaction in the Customs Database of one country and the registration of the corresponding import transaction in the Customs Database of the other country.
Results of the comparison were used to select check samples at audited Tax Offices.
Attention was focused on some taxpaying entities that had accomplished export or import transactions between the Czech Republic and the Slovak Republic, which were impossible to identify in both national Customs Databases.
Furthermore, some other taxpaying entities, in respect whereof data on export or import transactions between the Czech Republic and the Slovak Republic was found in both national Customs Databases, were selected for the purpose of verification of business operations shown in Value-Added Tax Return Forms.


4.3. Pilot Audit at the Stary Hrozenkov - Drietoma Border Checkpoint

The border checkpoint for the pilot audit was selected on the basis of a joint analysis of unassigned SAD items in the Czech and Slovak Customs Databases.


\pagebreak

\subsection*{Překlad:}



– Pro mobilní sítě (mobilní telefony), – 8525 20 99 Přenosové přístroje s přijímacím zařízením:



– 8529 90 40 Části, součásti a součásti vhodné pro použití výhradně nebo především s přístroji patřícími do čísel 8525 až 8528

Ostatní: Části a součásti přístrojů patřících do podpoložek 8525 10 50, 8525 20 91, 8525 20 99, 8525 40 11 a 8527 90 92.
Položky JSD z celní databáze České a Slovenské republiky byly vzájemně přiřazeny na základě odpovídajícího obsahu různých sloupců JSD (např. fakturovaná cena, množství, hmotnost, poznávací značka vozidla přepravujícího zásilku, datum překročení hranice, číslo celního sazebníku atd.).
Výsledky srovnání jsou uvedeny v následujících tabulkách a grafech.
Proces párování jednotlivých vývozních a dovozních transakcí byl komplikován mimo jiné tím, že vývozní a dovozní transakce nebyly stranami deklarovány stejným způsobem.
Například česká vývozní transakce využívala jeden JSD, zatímco odpovídající dovozní transakce na slovenské straně využívala více JSD, existovaly rozdíly v počtech položek mezi vývozním JSD a jeho odpovídajícím dovozním JSD atd.
Dalšími faktory ovlivňujícími míru úspěšnosti procesu srovnávání nebo přiřazování byla struktura údajů v české a slovenské celní databázi, jakož i období, které uplynulo mezi registrací vývozní transakce v celní databázi jedné země a registrací odpovídající dovozní transakce v celní databázi druhé země.
Výsledky srovnání byly použity k výběru kontrolních vzorků na auditovaných finančních úřadech.
Pozornost byla věnována některým subjektům povinným k dani, které uskutečnily vývozní nebo dovozní transakce mezi Českou republikou a Slovenskou republikou, které nebylo možné identifikovat v obou národních celních databázích.
Kromě toho byly některé jiné subjekty povinné k dani, pokud jde o údaje o vývozních nebo dovozních transakcích mezi Českou republikou a Slovenskou republikou, které byly zjištěny v obou národních celních databázích, vybrány za účelem ověření obchodních operací uvedených ve formulářích pro vrácení daně z přidané hodnoty.


4.3 Pilotní audit na hraničním kontrolním stanovišti Starý Hrozenkov – Drietoma

Hraniční kontrolní stanoviště pro pilotní audit bylo vybráno na základě společné analýzy nepřidělených položek JSD v české a slovenské celní databázi.


\end{document}
