\documentclass[10pt]{article}

%\usepackage{times}
\usepackage{fancyhdr}

\pagestyle{fancy}
\fancyhf{}
\rhead{\textbf{Kód: jahoda}}

\begin{document}

\subsection*{Zdroj:}



Report on audit results in refugee facilities in the Czech Republic and the Slovak Republic



State Property and Financial Means Allocated to Cover the Work of the Refugee Facilities Administration of the Ministry of Interior



Audit no. 04/23 „State Property and Financial Means Allocated to Cover the Work of the Refugee Facilities Administration of the Ministry of Interior"



Audit no. 15/2004 „Audit of the Management of State Property and Finances Earmarked for the Provision of Asylum and Temporary Refuge to Foreigners in the Slovak Republic



1. Introduction

Population migration is a worldwide trend that has become an integral part of the present development stage of the Czech Republic and the Slovak Republic.
It is mainly caused by citizens fleeing from their country of origin or last place of permanent residence to escape persecution, in particular racial, religious, national and political persecution, and to escape military conflicts and ethnic cleansing.
Its objective is to obtain asylum and to integrate into society, primarily in one of the advanced democratic countries of Europe.
However, another major group is formed of foreigners whose reason for migration is poor economic and living conditions.
Migration is not a new phenomenon.
Specific principles for assisting refugees were laid down by the 1951 Geneva Convention and the New York Protocol of 1967 regarding the legal status of refugees. Signatory countries bound themselves to apply these principles in their foreign policy.
The Czech and Slovak Federative Republic ratified the New York Protocol in 1991 and the Geneva Convention in 1992.
Following the split of Czechoslovakia in 1993, both the Czech Republic and the Slovak Republic declared their position on the refugee issue by continuing to comply with the international documents they had signed up to; both states confirmed them by enshrining them in national legislation.
Both countries have also enshrined the provision of asylum in their constitutions.
The asylum process and related matters came under scrutiny in parallel audits performed by the Supreme Audit Office, Czech Republic and the Supreme Audit Office of the Slovak Republic. The results of both are contained in the joint final report.
Cooperation between the two countries' supreme audit offices took place on the basis of the „Treaty on Cooperation in Audit Work" (hereinafter the „Treaty") signed in April 2004.


\pagebreak

\subsection*{Překlad:}



Zpráva o výsledcích auditu v uprchlických zařízeních v České republice a na Slovensku



Majetek státu a finanční prostředky určené na pokrytí práce Správy uprchlických zařízení Ministerstva vnitra



Audit č. 04/23 „Majetek státu a finanční prostředky určené na pokrytí práce správy uprchlických zařízení Ministerstva vnitra“



Audit č. 15/2004 „Audit hospodaření státního majetku a financí vyčleněných na poskytování azylu a dočasného uprchlíka cizincům v SR



1. Úvod

Migrace obyvatelstva je celosvětový trend, který se stal nedílnou součástí současné vývojové fáze České republiky a Slovenské republiky.
Je to způsobeno především občany, kteří prchají ze země původu nebo posledního místa trvalého pobytu, aby unikli pronásledování, zejména rasovému, náboženskému, národnímu a politickému pronásledování, a aby unikli vojenským konfliktům a etnickým čistkám.
Jeho cílem je získat azyl a začlenit se do společnosti, především v jedné z vyspělých demokratických zemí Evropy.
Další významnou skupinu však tvoří cizinci, jejichž důvodem pro migraci jsou špatné ekonomické a životní podmínky.
Migrace není nový fenomén.
Zvláštní zásady pro pomoc uprchlíkům byly stanoveny v Ženevské úmluvě z roku 1951 a v Newyorském protokolu z roku 1967 o právním postavení uprchlíků. Signatářské země se zavázaly tyto zásady uplatňovat ve své zahraniční politice.
Česká a Slovenská federativní republika ratifikovala Newyorský protokol v roce 1991 a Ženevskou úmluvu v roce 1992.
Po rozpadu Československa v roce 1993 vyhlásila Česká republika i Slovenská republika svůj postoj k problematice uprchlíků tím, že nadále dodržují mezinárodní dokumenty, které podepsaly; oba státy je potvrdily zakotvením do národní legislativy.
Obě země rovněž zakotvily poskytování azylu ve svých ústavách.
Proces azylu a souvisejících záležitostí byl prověřován souběžnými audity Nejvyššího kontrolního úřadu, České republiky a Nejvyššího kontrolního úřadu Slovenské republiky. Výsledky obou jsou obsaženy ve společné závěrečné zprávě.
Spolupráce mezi nejvyššími kontrolními úřady obou zemí proběhla na základě „Smlouvy o spolupráci při auditu“ (dále jen „Smlouva“) podepsané v dubnu 2004.


\end{document}
