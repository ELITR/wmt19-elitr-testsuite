\documentclass[10pt]{article}

%\usepackage{times}
\usepackage{fancyhdr}

\pagestyle{fancy}
\fancyhf{}
\rhead{\textbf{Kód: smaragd}}

\begin{document}

\subsection*{Zdroj:}



“The operator that dispatches selected products from the tax warehouse is obliged to notify the customs office with the appropriate territorial competence before the start of the movement of the selected product;

the movement of products can only start upon the approval of the customs office.


The customs office issues approval of the movement of selected products without delay upon a deposit of guarantee to secure the tax ...”

The consignor in the CR can only start the movement of any type of excisable products upon approval of the competent customs office.
The customs office grants such approval only when the guarantee to secure the tax has been deposited in the amount equal to the tax on transported products.
The consignor is obliged to issue the accompanying document in five copies before the start of the movement.
The customs office indicates its approval to start the movement on the accompanying document.
The copy 1 is kept by the consignor, the copies 2, 3, and 4 are given to the transporter and the copy 5 is kept by the customs office which approved the movement.
Under the Slovak legislation, the consignee in the SR keeps the copy 2 of the accompanying document, he confirms taking over the products on the copies 3 and 4 and presents both copies to the competent customs office.
The copy 3 validated by the customs office, is sent to the consignor no later than the 15th day of the subsequent calendar month from when the transfer began.
The copy 4 is kept by the customs office of the consignee.
The movement of products is ended when taken over by the consignee.
The customs office in the CR which approved of the movement to commence discharges the guarantee within 5 working days since the date of presentation of copy 3 of the accompanying document, which is validated by the consignee and the Slovak customs office.


Graph 2 - Circulation of accompanying document when the movement started in the CR and ended in the SR



Movement from the SR



Before 31 December 2005, the procedure at the movement of spirits and tobacco products from SR to another EU member state was regulated in the following manner:

Pursuant to the provision of § 26 (4) Act No 105/2004 Coll.


\pagebreak

\subsection*{Překlad:}



„Provozovatel, který zasílá vybrané produkty z daňového skladu, je povinen uvědomit celní úřad s příslušnou územní pravomocí před zahájením pohybu vybraného produktu;

Přeprava produktů může začít až po schválení celním úřadem.


Celní úřad neprodleně schválí přesun vybraných produktů po složení záruky za účelem zajištění daně...“

Odesílatel v ČR může zahájit přepravu jakéhokoli druhu výrobků podléhajících spotřební dani pouze po schválení příslušným celním úřadem.
Celní úřad takové schválení udělí pouze tehdy, pokud byla jistota na zajištění daně uložena ve výši rovnající se dani z přepravovaných produktů.
Odesílatel je povinen vydat průvodní doklad v pěti stejnopisech před zahájením přepravy.
Celní úřad vyjadřuje souhlas s tím, že zahájí přepravu průvodního dokladu.
Vyhotovení 1 si ponechá odesílatel, kopie 2,3 a 4 se předají dopravci a vyhotovení 5 si ponechá celní úřad, který přepravu schválil.
Podle slovenského právního předpisu si příjemce v SR uchovává kopii 2 průvodního dokladu, potvrzuje převzetí produktů na kopiemi 3 a 4 a předloží oba kopie příslušnému celnímu úřadu.
Vyhotovení 3 ověřené celním úřadem se zašle odesílateli nejpozději patnáctým dnem následujícího kalendářního měsíce od okamžiku, kdy byl převod zahájen.
Vyhotovení 4 si ponechá celní úřad příjemce.
Pohyb výrobků je ukončen poté, co jej převezme příjemce.
Celní úřad v ČR, který schválil přemístění za účelem zahájení vyřízení, vydá záruku do pěti pracovních dnů ode dne předložení kopie 3 průvodního dokladu, která je potvrzena příjemcem a slovenským celním úřadem.


Graf 2 – Obracení průvodních dokladů, když pohyb začal v ČR a skončil v SR



Pohyb z SR



Do 31. prosince 2005 byl postup při přemísťování lihovin a tabákových výrobků z SR do jiného členského státu EU upraven tímto způsobem:

Podle ustanovení § 26 odst. 4 zákona č. 105 / 2004 Sb.


\end{document}
