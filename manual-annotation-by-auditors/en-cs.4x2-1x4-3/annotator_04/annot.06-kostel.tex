\documentclass[10pt]{article}

%\usepackage{times}
\usepackage{fancyhdr}

\pagestyle{fancy}
\fancyhf{}
\rhead{\textbf{Kód: kostel}}

\begin{document}

\subsection*{Zdroj:}

The harmonisation of Slovak law with EU law on migration and the asylum process has been appraised in the European Commission's regular reports on readiness of the Slovak Republic to join the EU.
In 2002 the report drew attention to the MO's insufficient staffing in the light of the increased number of asylum seekers.


4. Asylum process

In the Czech Republic matters concerning refugees have come under the remit of the Mol since 1990.
With effect from 1.1.1996 the Mol set up the RFA in order to separate the exercise of stale administration in the asylum procedure, which is exercised via the Asylum and Migration Policy Department (AMPD) of the Mol, from the provision of services to asylum seekers.
The RFA's task is to handle, on behalf of the Mol, the provision of accommodation, catering, healthcare and other services to participants in asylum proceedings and those granted asylum.
In the Slovak Republic the key tasks in the asylum process are carried out by the MO with particular regard to the financing of and material provision for the asylum process in the scope set out by special regulations.
Priority tasks are matters related to decision-making on foreigners' asylum applications in the first stage of administrative proceedings and the comprehensive operation of refugee facilities in terms of accommodation, food, healthcare and financial and material provision for asylum seekers.
In addition, it prepares the concept of the state's migration policy.
The results achieved in the creation of an asylum system in the Slovak Republic in past years have been markedly influenced by its cooperation with and support from the office of the United Nations High Commissioner for Refugees („UNHCR").
One particular benefit was the office's financial involvement in renovating housing stock for asylum seekers.
The UNHCR has also co-financed social, cultural and sporting activities for asylum seekers.


4.1 Development of the number of asylum seekers and results of the asylum process

The following graph shows how the number of asylum seekers in the Czech Republic has developed.
There was a particularly pronounced increase in 2001 as a result of the Kosovo crisis; in 2003 there was an increased number of refugees from Chechnya.
Table 1 gives an overview of the results of asylum proceedings in the Czech Republic.


\pagebreak

\subsection*{Překlad:}

Harmonizace práva Slovenské republiky s právem EU v oblasti migrace a azylového procesu byla hodnocená v pravidelných zprávách Evropské komise k připravenosti na vstup Slovenské republiky do EU.
V roce 2002 bylo ve zprávě upozorněno na nedostatečné personální kapacity MÚ, které neodpovídaly zvýšenému počtu žadatelů o udělení azylu.


4. Azylový proces

V České republice spadají od roku 1990 záležitosti týkající se uprchlíků do kompetence MV.
S účinností od 1. 1. 1996 zřídilo MV organizační složku státu SUZ. Důvodem bylo oddělení výkonu státní správy v oblasti azylové procedury, která je vykonávána prostřednictvím odboru azylové a migrační politiky MV, od zabezpečování služeb žadatelům o udělení azylu.
Úkolem SUZ je zajišťovat pro MV poskytování ubytovacích, stravovacích, zdravotních a jiných služeb účastníkům řízení o udělení azylu a azylantům.
Ve Slovenské republice plní rozhodující úkoly v oblasti azylového procesu MÚ, a to zejména z hlediska finančního a materiálně-technického zabezpečení azylového procesu v rozsahu vymezeném zvláštními předpisy.
Prioritními úkoly jsou otázky související s rozhodováním o žádostech cizinců o udělení azylu v I. stupni správního řízení a komplexní provoz azylových zařízení v oblasti ubytování, stravování, zdravotního, finančního a materiálního zabezpečení žadatelů o udělení azylu.
Kromě toho připravuje koncepce k migrační politice státu.
Výsledky, dosažené v oblasti tvorby azylového systému ve Slovenské republice v uplynulých letech, významně ovlivnila spolupráce a podpora ze strany Úřadu Vysokého komisaře OSN pro uprchlíky (dále jen „UNHCR“).
Zvláštním přínosem byla jeho finanční spoluúčast na rekonstrukci bytového fondu pro azylanty.
UNHCR spolufinancovalo i sociální, kulturní a sportovní aktivity pro žadatele.


4.1 Vývoj počtu žadatelů o udělení azylu a výsledky azylového procesu

Vývoj počtu žadatelů o azyl v České republice je znázorněn v grafu.
K výraznému nárůstu došlo v roce 2001 z důvodu kosovské krize a v roce 2003 se projevil zvýšený počet uprchlíků z Čečenska.
Přehled o výsledcích azylového řízení v České republice je uveden v tabulce č.1.


\end{document}
