\documentclass[10pt]{article}

%\usepackage{times}
\usepackage{fancyhdr}

\pagestyle{fancy}
\fancyhf{}
\rhead{\textbf{Kód: zkratka}}

\begin{document}

\subsection*{Zdroj:}

Both the Czech Republic and the Federal Republic of Germany have transposed the European Union Directives into national law.


Corruption prevention



Significance of corruption prevention

In neither country is there a legal definition of what constitutes corruption.
In general, corruption is understood to mean the misuse of a position of confidence for private advantage.
As a rule, it is implied that such abuse is incited by those who benefit from it.
Corruption may hamper competition and may cause substantial economic damage.
According to estimates, global costs caused by corruption, amount to 5 \% of the global economic output.
For the European Union area, these costs are estimated at an amount equivalent of 1 \% of the area’s economic output.
If the contracting authority is involved, agreements restricting competition or unfair behaviour may undermine confidence in the integrity and the capability of the public administration to function.
Thus, preventing corruption is an essential task for awarding authorities.
Practices and procedures are to be reviewed to adress areas that are especially vulnerable to corruption so that appropriate anticorruption measures can be taken.
Taking preventive measures and carrying out effective controls may substantially reduce vulnerability of decision-making processes to corruption.
As a rule, the awarding of works contracts is considered to be particularly vulnerable to corruption.


Measures to prevent corruption

In Germany, corruption is prosecuted under penal and employment law, this especially applies to active and passive bribery and to the undue acceptance of benefits and undue granting of benefits.
In order to prevent corruption, the German Federal Government issued a guideline together with recommendations on corruption prevention in the federal administration , according to which, in addition to risk analyses, the following principles shall apply to contract award procedures: precedence of open tendering multiple control principle and transparent decision-making separation of planning, awarding of contracts and accounting precise and full documentation of procurement procedures.
In addition, further provisions having a preventive effect against corruption have become part of procurement law.
In 2011, the federal department responsible introduced an IT-supported system to monitor the awarding of federal road construction contracts.
By means of this system, procurement data, e.g. estimated costs, order total and private independent professionals involved, can be fully recorded and analysed with a view to finding irregularities.


\pagebreak

\subsection*{Překlad:}

Česká republika i Spolková republika Německo provedly směrnice Evropské unie do vnitrostátního práva.


Předcházení korupci



Význam předcházení korupci

Ani v jedné zemi neexistuje právní definice toho, co představuje korupci.
Korupce je obecně chápána jako zneužití postavení důvěry ve prospěch soukromého sektoru.
Zpravidla se předpokládá, že takové zneužití je podněcováno těmi, kdo z něj mají prospěch.
Korupce může narušit hospodářskou soutěž a může způsobit značné hospodářské škody.
Podle odhadů činí celkové náklady způsobené korupcí 5 \% celosvětového hospodářského výkonu.
Pro oblast Evropské unie se tyto náklady odhadují na částku rovnající se 1 \% hospodářského výkonu oblasti.
Pokud je zapojen veřejný zadavatel, mohou dohody omezující hospodářskou soutěž nebo nekalé jednání narušit důvěru v integritu a schopnost veřejné správy fungovat.
Zásadním úkolem zadavatelů je tedy předcházení korupci.
Praktiky a postupy je třeba přezkoumat v oblastech, které jsou obzvláště zranitelné korupcí, aby bylo možné přijmout vhodná protikorupční opatření.
Přijímání preventivních opatření a provádění účinných kontrol může podstatně snížit zranitelnost rozhodovacích procesů vůči korupci.
Zadávání zakázek na stavební práce je zpravidla považováno za obzvláště zranitelné vůči korupci.


Opatření k předcházení korupci

V Německu je korupce stíhána trestním a pracovním právem, to platí zejména pro aktivní a pasivní úplatkářství a pro neoprávněné přijímání dávek a neoprávněné poskytování dávek.
Aby se zabránilo korupci, vydala německá spolková vláda pokyny spolu s doporučeními pro předcházení korupci ve spolkové správě , podle nichž se kromě analýzy rizik použijí na postupy při zadávání zakázek tyto zásady: přednost otevřeného zadávání veřejných zakázek zásada vícenásobné kontroly a transparentní oddělení plánování rozhodování, zadávání zakázek a přesné a úplné dokumentace postupů při zadávání veřejných zakázek.
Součástí zákona o zadávání zakázek se navíc stala další ustanovení s preventivním účinkem proti korupci.
V roce 2011 zavedlo příslušné federální oddělení systém podporovaný informačními technologiemi pro sledování zadávání zakázek na výstavbu federálních silnic.
Prostřednictvím tohoto systému mohou být údaje o zadávání zakázek, např. odhadované náklady, celková objednávka a zapojení soukromých nezávislých odborníků, plně zaznamenány a analyzovány s cílem odhalit nesrovnalosti.


\end{document}
