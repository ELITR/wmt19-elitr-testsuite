\documentclass[10pt]{article}

%\usepackage{times}
\usepackage{fancyhdr}

\pagestyle{fancy}
\fancyhf{}
\rhead{\textbf{Kód: tlapka}}

\begin{document}

\subsection*{Zdroj:}

Because of the federal structure in Germany both the federal tax administration (in particular the CLO) and the tax administrations of the 16 federal states participate in the development and the operation of RMSs.
Although there are joint working groups for the development neither the CLO nor the tax administrations of the federal states share a completely consistent database.
At the level of the CLO unit ST I 2 (hereinafter the “KUSS“ – coordination unit for special VAT audits and tax investigations) collects audit findings on tax fraud in a separate database called OPUS.
KUSS established another database called ZAUBER in cooperation with the tax offices to communicate the names of discovered “missing traders” or VAT fraudsters.
KUSS coordinates both connected cases of national and cross-border tax fraud.
The experience gathered by KUSS is incorporated into the development of RMS in the working groups.
Furthermore, experts of several other units of the CLO and the MoF are entrusted with RMS duties and responsibilities.
At the level of the CLO, a formal control is done of the composition of VAT IDs of other Member States in submitted RS of German taxpayers.
This formal control is processed before RS data are submitted to other Member States to improve the clarity and use of the VIES data submitted (see 4.4.2).
The verification of VAT IDs mentioned in RS is proven on the basis of O\_MCTL-Codes submitted by other Member States (for the process of correcting an invalid VAT ID see also 4.4.2).
Furthermore, the RMS of the CLO is based on the VIES data transmitted by other Member States and it creates several types of inspection notes based on several characteristics usually found in carousel transactions to detect a potential “missing trader”.
The tax offices of the federal states are instructed to process these inspection notes and in some cases to report the result of their investigation to the CLO.
The audit findings of the BRH confirmed that it is necessary to include all available characteristics in order to be able to assume a case of tax fraud with a sufficient probability (instead of focussing on a company’s turnover only).
So the BRH suggested to the MoF to imply a mix of diverse characteristics for producing the reports.


\pagebreak

\subsection*{Překlad:}

Vzhledem k federativní (spolkové) struktuře SRN se federální daňová správa (zejména CLO) a daňové správy 16 spolkových států podílejí na vývoji a provozování RMS.
Přestože existují společné pracovní skupiny pro vývoj, ani CLO, ani daňové správy federálních států nesdílejí zcela konzistentní databázi.
Na úrovni CLO, shromažďuje útvar ST I 2 (dále jen „KUSS“ – koordinační útvar pro speciální audity DPH a daňová šetření) zjištění kontroly týkající se daňových podvodů ve zvláštní databázi zvané OPUS.
KUSS zavedl ve spolupráci s finančními úřady novou databázi zvanou ZAUBER určenou pro přenos názvů firem daňových podvodníků „missing traders“.
KUSS koordinuje související případy vnitrostátních i přeshraničních daňových podvodů.
Zkušenosti KUSS jsou promítány do vývoje RMS v pracovních skupinách.
Kromě toho jsou specialistům několika dalších útvarů CLO a MF svěřeny povinnosti a pravomoci spojené s RMS.
Na úrovni CLO je zavedena formální kontrola složení VAT ID jiných členských států v SH podaných německými plátci.
Tato formální kontrola je provedena před poskytnutím údajů z SH do jiných členských států, ke zlepšení srozumitelnosti a využití poskytnutých údajů VIES (viz 4.4.2).
Ověření platnosti VAT ID uvedených v SH je provedeno na základě kódů O\_MCTL poskytnutých jinými členskými státy (pro postup opravování neplatného VAT ID, viz. také 4.4.2).
Kromě toho je RMS na CLO založen na údajích systému VIES předávaných jinými členskými státy a vytváří několik typů kontrolních zpráv založených na několika charakteristikách, které obvykle figurují v tzv. karuselových obchodních operacích, za účelem zjišťování potenciálních „missing traders“.
Finanční úřady spolkových států mají instrukci, aby tyto kontrolní zprávy zpracovávaly a v některých případech podávaly zprávy o výsledcích svých zjištění CLO.
Zjištění auditu BRH potvrdila, že je nezbytné vzít v úvahu všechny dostupné charakteristiky, má-li být s dostatečnou pravděpodobností předpokládána existence případu daňového podvodu (místo výhradního zaměření na obrat společnosti).
BRH proto navrhl MF, aby při zpracování zpráv byly zohledněny různé charakteristiky.


\end{document}
