\documentclass[10pt]{article}

%\usepackage{times}
\usepackage{fancyhdr}

\pagestyle{fancy}
\fancyhf{}
\rhead{\textbf{Kód: podrost}}

\begin{document}

\subsection*{Zdroj:}

Two concrete bridges were identified on the A 17 and four on the D 8 whose price is markedly higher.
These bridges were identified partly by a method enabling the ascertainment of atypical values in a set of numbers and partly by regressive analysis taking into account the correlation between the bridge price and its size (distance above the regressive curve is a key to identifying atypically expensive bridges).
Information about the atypically high price of the said bridges should, however, be regarded merely as an indicator as to which bridges to focus on.
In order to be able to make a generally applicable conclusion the causes of the high price would have to be analysed in each identified case.
One reason may be inefficient administrative action but it may also have been caused by environmental protection requirements or any other acceptable reasons.
Given that the Czech concrete bridges are on average longer than the German ones (see table below), the proven negative correlation between the bridge price and length might lead one to expect that they would be cheaper.
The hypothesis about the lower cost of Czech concrete bridges is also supported by the lower cost of labour in the construction industry in the Czech Republic than in Germany.
The results of the analysis, however, indicate that the price of concrete bridges on the D 8 is higher than on the A 17.
Two possible reasons for this state of affairs were identified during joint meetings.
1.
In Germany most bridges are procured as independent projects.
That makes it possible for small and medium-sized regional firms to take part in the tender.
In the Czech Republic many bridges are procured as part of a contract for a part of motorway route that is several kilometres long.
Small and medium-sized firms do not take part in this kind of tender, as the job is too big for them.
2.
There is greater variety in bridge structures used in the Czech Republic than in Germany.
In this context the NKÚ found that the type of bridge structure is decided long before it is built.
The decision is made during the time when the RMD submits the zoning decision application.
At this time the RMD selects bridge types from several alternatives, but economy plays only a minor role in decision-making.


\pagebreak

\subsection*{Překlad:}

Dva betonové mosty byly identifikovány na A 17 a čtyři na D 8, jejichž cena je výrazně vyšší.
Tyto mosty byly identifikovány částečně metodou umožňující zjištění atypických hodnot v souboru čísel a částečně regresivní analýzou s přihlédnutím ke korelaci mezi cenou mostu a jeho velikostí (vzdálenost nad regresivní křivkou je klíčem k identifikaci atypicky drahých mostů).
Informace o atypicky vysoké ceně uvedených mostů by však měly být považovány pouze za ukazatel toho, na které mosty se zaměřit.
Aby bylo možné dospět k obecně platnému závěru, musely by být příčiny vysoké ceny analyzovány v každém zjištěném případě.
Jedním z důvodů může být neúčinné správní jednání, ale může být také způsobeno požadavky na ochranu životního prostředí nebo jinými přijatelnými důvody.
Vzhledem k tomu, že české betonové mosty jsou v průměru delší než německé (viz tabulka níže), prokazatelná negativní korelace mezi cenou mostů a jejich délkou by mohla vést k očekávání, že budou levnější.
Hypotézu o nižších nákladech českých betonových mostů podporuje také nižší cena práce ve stavebnictví v České republice než v Německu.
Výsledky analýzy však naznačují, že cena betonových mostů na dálnici D 8 je vyšší než na dálnici A 17.
Během společných jednání byly určeny dva možné důvody tohoto stavu.
1.
V Německu je většina mostů pořizována jako nezávislé projekty.
To umožňuje účast malých a středních regionálních firem v nabídkovém řízení.
V České republice je mnoho mostů pořizováno jako součást smlouvy na část dálnice, která je několik kilometrů dlouhá.
Malé a střední podniky se tohoto druhu výběrového řízení neúčastní, protože je pro ně příliš velké.
2.
V České republice jsou mostní konstrukce rozmanitější než v Německu.
V této souvislosti NKÚ zjistil, že o typu mostní konstrukce se rozhoduje dlouho před jejím postavením.
Rozhodnutí se přijímá v době, kdy ŘSD podává žádost o územní rozhodnutí.
V této době ŘSD vybírá typy mostů z několika alternativ, avšak hospodárnost hraje při rozhodování pouze malou roli.


\end{document}
