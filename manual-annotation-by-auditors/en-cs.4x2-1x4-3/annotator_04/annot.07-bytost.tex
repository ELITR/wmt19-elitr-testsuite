\documentclass[10pt]{article}

%\usepackage{times}
\usepackage{fancyhdr}

\pagestyle{fancy}
\fancyhf{}
\rhead{\textbf{Kód: bytost}}

\begin{document}

\subsection*{Zdroj:}



Basic audit findings



BRH



Documentation

A large number of the tendering procedures audited were not documented completly.
The contract award notices did not include all necessary information about e.g. reasons for product specifications and national regulations.
In the case of one tendering procedure, the contracting authority did not give any reason why it had awarded the contract to a bidder whose tender did not include obligatory proofs.
In a few cases, documentation of the contracting authorities did not include all records required for documenting the tendering procedure.


Publication of contract awards

In the case of some EU-wide tendering procedures, the contracting authorities failed to publish the contract award notices.
In the case of several tendering procedures, the proofs for the publication of contract award notices by the contracting authorities were missing.


Reporting duties

The statistical compilations reviewed included the required minimum information and had been produced and posted on a timely basis.


Ex-post transparency

In most cases, the construction administrations audited published the contract data of the completed tendering procedure.
In the case of one construction administration, however, these data were not published due to a communication error.
It took the construction administration half a year to notice the mistake.
As a result, contract data of the tendering procedures completed during that period were not published.


NKÚ



Documentation

In case of construction work connected with a large sports event considerable shortcomings in documentation in course of procurement procedure were found out, which concerned mainly incorrect data e.g. in evaluation committee records concerning the bid prices.
In adition, some of the minutes about the opening of envelopes were not signed.


Publication of contract awards

In case of repairs and maintenance of road construction projects, announcements of contracts awarded were distributed after the deadline, resp. they were not published.


Summarised findings

The provisions on documentation, reporting and notification implement the EU requirements and are largely identical in both countries.
This generally ensures the transparency of contract-award procedures and award decisions.
The Czech publication requirements are somewhat stricter than their German counterparts.
The examination revealed shortcomings substantially reducing the value of documentation and publication as a means of corruption prevention.


THISsegmentISintentionallyLEFTblank

The construction authorities should further enhance the transparency of their contract award procedures and should fully meet reporting and notification requirements.


\pagebreak

\subsection*{Překlad:}



Základní zjištění auditu



BRH



Dokumentace

Velký počet auditovaných nabídkových řízení nebyl kompletně zdokumentován.
Oznámení o zadání zakázky neobsahovala všechny nezbytné informace, např. O důvodech specifikace výrobku a vnitrostátních předpisů.
V případě jednoho nabídkového řízení zadavatel neuváděl důvod, proč zakázku zadal uchazeči, jehož nabídka neobsahovala povinné důkazy.
V několika případech dokumentace zadavatelů neobsahovala všechny záznamy potřebné pro doložení zadávacího řízení.


Zveřejňování zakázek

V případě některých nabídkových řízení na úrovni EU veřejní zadavatelé nezveřejnili oznámení o zadání zakázky.
V případě několika nabídkových řízení chyběly důkazy pro zveřejnění oznámení o zadání zakázky zadavateli.


Oznamovací povinnosti

Zkoumané statistické kompilace zahrnovaly požadované minimální informace a byly připraveny a zveřejněny včas.


Průhlednost ex post

Ve většině případů auditované stavební úřady zveřejnily údaje o zakázce dokončeného nabídkového řízení.
V případě jedné stavební správy však tyto údaje nebyly z důvodu chyby komunikace zveřejněny.
Správci stavby trvalo půl roku, než zjistil chybu.
V důsledku toho nebyly zveřejněny smluvní údaje o nabídkových řízeních dokončených v tomto období.


NKÚ



Dokumentace

V případě stavebních prací spojených s velkou sportovní událostí byly zjištěny značné nedostatky v dokumentaci v průběhu zadávacího řízení, které se týkaly především nesprávných údajů např. V záznamech hodnotící komise o nabídkových cenách.
Kromě toho některé zápisy o otevření obálek nebyly podepsány.


Zveřejňování zakázek

V případě oprav a údržby staveb pozemních komunikací byly po uzávěrce, resp. nebyly zveřejněny.


Shrnutí zjištění

Ustanovení o dokumentaci, podávání zpráv a oznámení plní požadavky EU a jsou v obou zemích do značné míry totožné.
To obecně zajišťuje transparentnost postupů zadávání zakázek a rozhodnutí o přidělení zakázky.
Česká publikační požadavky jsou poněkud přísnější než jejich německé protějšky.
Zkoumání odhalilo nedostatky, které podstatně snižují hodnotu dokumentace a publikace jako prostředku prevence korupce.


THISsegmentISintentionallyLEFTblank

Stavební úřady by měly dále zvyšovat transparentnost svých postupů zadávání zakázek a měly by plně splňovat požadavky na podávání zpráv a oznámení.


\end{document}
