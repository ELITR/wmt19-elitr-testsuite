\documentclass[10pt]{article}

%\usepackage{times}
\usepackage{fancyhdr}

\pagestyle{fancy}
\fancyhf{}
\rhead{\textbf{Kód: vidina}}

\begin{document}

\subsection*{Zdroj:}



JOINT FINAL REPORT on the Results of Parallel Audits of Excise Duty Administration in the Slovak Republic and in the Czech Republic



Joint Report of the SAO of the Czech Republic and the SAO of the Slovak Republic



May 2017



Supreme Audit Office, Slovak Republic



Supreme Audit Office, Czech Republic



Introduction

This joint report provides information about the course and outcome of international cooperation on parallel audits conducted by the Supreme Audit Office of the Slovak Republic (“Slovak SAO”) and the Supreme Audit Office of the Czech Republic (“Czech SAO”) in the field of excise duty administration in 2012-2015.
The cooperation between the Supreme Audit Institutions of both countries was carried out based on an agreement on cooperation on parallel audits of excise duty administration concluded between them in August 2015.
The topic of the parallel audits was chosen because both EU Member States follow common European legislation on this matter, thereby providing an opportunity to ascertain how European legislation is applied on a national level, how the excise duty administration system is set up, and how the system in Slovakia and the system in the Czech Republic differ.
The greatest benefits of these audits consisted in the exchange of experience, identification of differences and good practice.
Last, but not least, finding out that international comparability of data on efficiency and effectiveness of excise duty administration is extremely limited without performing parallel audits was also beneficial.


Miloslav Kala



Karol Mitrík



president



president



Supreme Audit Office Czech Republic



Supreme Audit Office Slovak Republic



Results of Parallel Audits of Excise Duty Administration



Summary and assessment

Legislation in the Czech Republic and in Slovakia on excise duty and its administration complies with EU legislation and exceeds EU requirements substantially.
Differences have been discovered between Slovak and Czech excise duty regulations: in the Czech Republic, verification of the economic stability of applicants for permits and mandatory labelling and dyeing of several mineral and some other oils contribute to effective excise duty administration; in Slovakia, the distribution of excise stamps is simpler than in the Czech Republic and less of a burden on excise duty administrators.
In the fight against excise duty evasion, supervising the movement of raw tobacco or tobacco materials, the implementation of the institute of a transport fuel distributor, and keeping a registry of merchants of consumer packaged alcohol appear to be a good practice in the fight against excise duty evasion.


\pagebreak

\subsection*{Překlad:}



Společná závěrečná zpráva o výsledcích paralelních auditů správy spotřebních daní ve Slovenské republice a v České republice



Společná zpráva NKÚ České republiky a NKÚ Slovenské republiky



Květen 2017



Nejvyšší kontrolní úřad, Slovenská republika



Nejvyšší kontrolní úřad, Česká republika



Úvod

Tato společná zpráva poskytuje informace o průběhu a výsledcích mezinárodní spolupráce v oblasti paralelních auditů prováděných Nejvyšší kontrolní kanceláří Slovenské republiky („Slovak Sao“) a Nejvyšší kontrolní kanceláří České republiky („Czech Sao“) v oblasti správy spotřebních daní v letech 2012 – 2015.
Spolupráce mezi nejvyšším kontrolním orgánem obou zemí byla vedena na základě dohody o spolupráci v oblasti paralelních auditů správy spotřební daně, která byla uzavřena mezi těmito orgány v srpnu 2015.
Téma paralelních auditů bylo vybráno proto, že oba členské státy EU dodržují společné evropské právní předpisy v této oblasti, čímž poskytují příležitost zjistit, jak se evropské právní předpisy uplatňují na vnitrostátní úrovni, jak je zaveden systém správy spotřební daně a jak se systém na Slovensku a systém v České republice liší.
Největší přínos těchto auditů spočíval v výměně zkušeností, zjišťování rozdílů a osvědčených postupů.
V neposlední řadě bylo přínosné zjistit, že mezinárodní srovnatelnost údajů o účinnosti a účelnosti správy spotřební daně je mimořádně omezená bez provádění paralelních auditů.


Miloslav Kala



Karol Mitrík



Předseda



Předseda



Nejvyšší kontrolní úřad Česká republika



Nejvyšší kontrolní úřad Slovenská republika



Výsledky paralelních kontrol správy spotřebních daní



Shrnutí a posouzení

Právní předpisy České republiky a Slovenska týkající se spotřební daně a její správy jsou v souladu s právními předpisy EU a výrazně překračují požadavky EU.
Rozdíly byly zjištěny mezi slovenskými a českými předpisy o spotřební dani: v České republice je kontrola hospodářské stability žadatelů o povolení a povinné označování a barvení několika minerálních a některých jiných olejů přínosem pro účinnou správu spotřební daně; na Slovensku je distribuce spotřebních známek jednodušší než v České republice a méně než v České republice. zatížení správců spotřební daně.
V boji proti daňovým únikům se zdá, že dohled nad pohybem surového tabáku nebo tabákových materiálů, zavedení institutu distributora pohonných hmot a vedení registru obchodníků s alkoholem zabaleným spotřebitelům jsou dobrou praxí v boji proti daňovým únikům.


\end{document}
