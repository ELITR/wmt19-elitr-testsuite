\documentclass[10pt]{article}

%\usepackage{times}
\usepackage{fancyhdr}

\pagestyle{fancy}
\fancyhf{}
\rhead{\textbf{Kód: pustina}}

\begin{document}

\subsection*{Zdroj:}

The shifting of asylum from the 3rd to the 1 st pillar was an important change for asylum policy in the EU - asylum law became part of community law on 1.5.1999 when the Treaty of Amsterdam entered into force.
This shift made it necessary to create suitable legislation, i.e. to create legal instruments of suitable binding force.
The compatibility of Czech asylum law with EU law has been covered by many assessment reports of the European Commission.
Key community regulations governing asylum were adopted shortly before or even after the Czech Republic joined the EU and the deadline for their transposition had not yet elapsed at the time of the audit by the Supreme Audit Office, Czech Republic.
Once they are incorporated into Czech law, the Czech Republic will be obliged to report to the European Commission on how it effected the transposition.
In the Slovak Republic the procedure to be followed by the state authorities in proceedings to award refugee status for and determine the duties of foreigners who have applied for refugee status or who have been provided with temporary refuge in the territory of the Slovak Republic was laid down, with effect from 1.1.1996, by Act of the National Assembly of the Slovak Republic no. 283/1005 Coll., on refugees.
Since 2003 the act on refugees has been superseded by Act no. 480/2002 Coll., on asylum and amending certain acts, in which Slovak law was harmonised with Council Directive 2001/55/EC of 20.7.2001, on minimum standards for giving temporary protection in the event of a mass influx of displaced persons and on promoting a balance of efforts between Member States in receiving such persons and bearing the consequences thereof.
After two amendments of Act no. 480/2002 Coll., on asylum, in 2003 and 2004, a third amendment was passed at the beginning of 2005. This third amendment represents further harmonisation of Slovak law with EU legislation on migration and asylum.
The duty for asylum seekers to visit a Slovak language introductory course during their stay in an integration centre, the appointment of court guardians for foreign minors staying in the territory of the Slovak Republic without a legal representative, making it possible for applicants to enter into employment under certain defined circumstances are all significant features of the amendment.


\pagebreak

\subsection*{Překlad:}

Významnou změnou pro azylovou politiku v rámci EU byl přesun problematiky azylu z 3. do 1. pilíře, kdy se právo azylu stalo součástí komunitárního práva od 1.5. 1999 na základě Amsterodamské smlouvy.
Tento posun vyvolal potřebu vytvořit i odpovídající legislativu, tzn. vytvořit právní nástroje v odpovídající právní závaznosti.
Slučitelnost českého azylového práva s dokumenty EU byla předmětem pravidelných hodnotících zpráv Evropské komise.
Klíčové komunitární předpisy v oblasti azylu byly přijaty krátce před vstupem či až po vstupu České republiky do EU a jejich transpoziční lhůta v době kontroly Nejvyššího kontrolního úřadu, Česká republika ještě neuplynula.
Poté, co dojde k jejich implementaci do českého právního řádu, bude Česká republika povinna oznámit Evropské komisi, jakým způsobem transpozici provedla.
Ve Slovenské republice postup státních orgánů v řízení o přiznání postavení uprchlíka a určení povinností cizinců, kteří požádali o přiznání postavení uprchlíka, nebo kterým bylo poskytnuto dočasné útočiště na území Slovenské republiky, stanovil s účinností od 1.1.1996 zákon Národní rady Slovenské republiky č. 283/1995 Z. z., o utečencoch.
Od roku 2003 byl zákon o uprchlících nahrazen zákonem č. 480/2002 Z. z„ o azyle a o změně a doplnění některých zákonů, ve kterém byla zapracována harmonizace právní úpravy Slovenské republiky se směrnicí Rady Evropy 2001/55/ES z 20. 7. 2001 o minimálních standardech na poskytování dočasné ochrany v případě hromadného přílivu vysídlených osob a o opatřeních na podporu rovnováhy úsilí mezi členskými zeměmi při přijímání těchto osob a snášení z toho vyplývajících důsledků.
Po dvou novelách zákona č. 480/2002 Z z., o azyle, v letech 2003 a 2004 byla na začátku roku 2005 schválena třetí novela, která znamená další harmonizaci práva Slovenské republiky s legislativou EU v oblasti migrace a azylu.
Jejími významnými prvky jsou například stanovení povinnosti pro azylanty navštěvovat po dobu pobytu v integračním středisku kurz základů slovenského jazyka, určení opatrovníka soudy u neplnoletých cizinců, pokud se zdržují na území Slovenské republiky bez zákonného zástupce, umožnění žadatelům vstupovat do pracovněprávního vztahu za určitých okolností, uvedených v novele zákona.


\end{document}
