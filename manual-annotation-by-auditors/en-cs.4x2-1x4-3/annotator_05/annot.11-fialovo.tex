\documentclass[10pt]{article}

%\usepackage{times}
\usepackage{fancyhdr}

\pagestyle{fancy}
\fancyhf{}
\rhead{\textbf{Kód: fialovo}}

\begin{document}

\subsection*{Zdroj:}

54 such cases were found at the audited tax offices.
E.g., during examination of a review instruction, it was ascertained that a difference of CZK 10 974 thousand was caused by the fact that the Slovak taxpayer included in the RS for the 1st quarter of 2005 the provision of services (construction work and supply of goods including assembly in the territory of the CR).
Another reason for the existence of differences consisted in inclusion of values of accepted advance payments in the RS; e.g., 2 Slovak taxpayers included in their RS advance payments accepted from Czech taxpayers, resulting in generating a review instruction concerning a difference of CZK 3 580 thousand.
Review instructions concerning acquired goods are generated only once per quarter.
In case of a subsequent change in the data declared in the RS in the EU member countries, such as a change in the buyer of goods in the CR or an increase in the value of goods supplied to the CR, no new review instructions are generated.
Given the fact that the VIES system does not provide automatic notification of these changes, without further information the locally competent tax administrator for the acquirer of the goods may not learn of the facts decisive for correct assessment of the tax obligation concerning the acquired goods and the probability of VAT fraud is very high.


The audit performed by the SAO, CR indicated that this situation occurred, e.g. in the following cases:

A Slovak taxpayer stated in the RS for the 2nd quarter of 2004 the value of goods supplied to a Czech taxpayer equal to SKK 1 504 thousand.
On February 1, 2005, the locally competent tax office for the Czech taxpayer sent a request for information to the SR, as the Czech taxpayer accounted for acquisition of goods only in the amount of CZK 425 thousand.
On the basis of tax proceedings, on March 1, 2005, the Slovak taxpayer submitted a supplementary RS, in which he corrected data on supply of goods to the CR, namely by decreasing the value of the supply to the original Czech taxpayer to SKK 538 thousand and increasing the value of goods supplied to another Czech taxpayer from SKK 194 thousand to the value of SKK 1 160 thousand.


\pagebreak

\subsection*{Překlad:}

54 takových případů bylo zjištěno u auditovaných finančních úřadů.
Například při přezkumu pokynů k přezkumu bylo zjištěno, že rozdíl ve výši 10 974 000 Kč byl způsoben skutečností, že slovenský daňový poplatník zahrnul do daňového přiznání za 1. čtvrtletí roku 2005 poskytování služeb (stavební práce a dodání zboží včetně montáže na území ČR).
Další důvod existence rozdílů spočíval v zahrnutí hodnot přijatých zálohových plateb do RS, např. 2 slovenští daňoví poplatníci zahrnutí do svých zálohových plateb do RS přijatých od českých daňových poplatníků, což vedlo k vydání pokynu k přezkumu, který se týkal rozdílu ve výši 3 580 tisíc korun.
Přezkum pokynů týkajících se pořízeného zboží se provádí pouze jednou za čtvrtletí.
V případě následné změny údajů uvedených v prohlášení o platební neschopnosti v členských zemích EU, jako je změna kupujícího zboží v ČR nebo zvýšení hodnoty zboží dodaného do ČR, nevznikají žádné nové pokyny k přezkumu.
Vzhledem ke skutečnosti, že systém VIES neposkytuje automatické oznamování těchto změn, bez dalších informací se místně příslušný správce daně pro nabyvatele zboží nemusí dozvědět o skutečnostech rozhodujících pro správné posouzení daňové povinnosti týkající se pořízeného zboží a pravděpodobnost podvodu s DPH je velmi vysoká.


Kontrola, kterou provedl NKÚ, ukázala, že k této situaci došlo, např. v těchto případech:

Slovenský daňový poplatník uvedl v RS za druhé čtvrtletí roku 2004 hodnotu zboží dodaného českému daňovému poplatníkovi ve výši 1 504 tisíc SKK.
Dne 1. února 2005 zaslal místně příslušný daňový úřad českému daňovému poplatníkovi žádost o informace, protože český daňový poplatník zaúčtoval pořízení zboží pouze ve výši 425 tisíc korun.
Na základě daňového řízení předložil slovenský daňový poplatník dne 1. března 2005 doplňující daňový přiznání, v němž opravil údaje o dodání zboží do ČR, a to snížením hodnoty dodání původnímu českému daňovému poplatníkovi na 538 tisíc SKK a zvýšením hodnoty zboží dodaného jinému českému daňovému poplatníkovi ze 194 tisíc SKK na hodnotu 1 160 tisíc SKK.


\end{document}
