\documentclass[10pt]{article}

%\usepackage{times}
\usepackage{fancyhdr}

\pagestyle{fancy}
\fancyhf{}
\rhead{\textbf{Kód: porucha}}

\begin{document}

\subsection*{Zdroj:}

An invoice containing the motor fuel distributor’s or registered person’s number allocated to it by the excise duty administrator is also sufficient proof of the method of acquisition of motor fuels.
During their supervision over mineral oils, excise duty administrators in the Slovak Republic registers duty evasion related to the movement of these commodities.


Small independent brewery

A small independent brewery operator is a person who brews beer as a business and fulfils certain statutory conditions.
Assuming the fulfilment of certain statutory conditions, a small independent brewery can apply a reduced duty rate of €2.652 per hl/percent of actual ethyl alcohol by volume.
Annual beer production is limited to 200,000 hl of beer.
As beer production is outside the duty suspension regime, an accompanying document is not required during movement because the beer is in free circulation.
A small independent brewery cannot be a tax warehouse.
In practice, such breweries are small family enterprises that also operate a restaurant or pub where they offer beer that they have brewed themselves.


Administrative tasks in Slovakia

Excise duty supervision over certain tax entities is divided up under Slovak law as follows.
The greatest increase in administrative tasks, i.e., by 5.3\%, was registered in 2012 and 2013. This was caused by the greater number of movements executed in EMCS.
Conversely, the greatest year-on-year drop was registered in 2013 and 2014, specifically by 4.3\%, which was caused by a smaller number of completed tax audits as well as a smaller number of submitted tax declarations.
The drop was due in part to changes in legislation.


Tax audits

Tax audits performed by the customs office aim to discover or examine facts required for the proper assessment of duties.
When performing tax audits, customs office employees present a tax audit authorisation to identify themselves. This authorisation contains information about the purpose and subject of the audit and periods covered by the audit.
In practice, this means that the audited period can be as short as one month or even as long as a number of taxation periods.
The scope of the audited periods is determined by the customs office.
A tax audit begins with a tax audit commencement report being drawn up.
Documents received or taken by the customs office are confirmed by a receipt.


\pagebreak

\subsection*{Překlad:}

Faktura obsahující číslo distributora pohonných hmot nebo registrované osoby, které mu přidělí správce spotřební daně, je rovněž dostatečným důkazem o způsobu pořízení motorových paliv.
Správci spotřební daně ve Slovenské republice během svého dohledu nad minerálními oleji zaznamenávají daňové úniky související s pohybem těchto komodit.


Malé nezávislé pivovarnictví

Malý nezávislý provozovatel pivovaru je osoba, která vaří pivo jako podnik a splňuje určité zákonné podmínky.
Za předpokladu splnění určitých zákonných podmínek může malý nezávislý pivovar uplatňovat sníženou celní sazbu ve výši 2,652 EUR na hl /\% skutečného lihu na základě objemu.
Roční produkce piva je omezena na 200 000 hl piva.
Vzhledem k tomu, že výroba piva není v režimu s podmíněným osvobozením od cla, nevyžaduje se během pohybu průvodní doklad, protože pivo je ve volném oběhu.
Malý nezávislý pivovar nemůže být daňovým skladem.
V praxi jsou takové pivovaru malými rodinnými podniky, které rovněž provozují restauraci nebo hospodu, kde nabízejí pivo, které si sami vychovali.


Administrativní úkoly na Slovensku

Dohled nad spotřební daní nad některými daňovými subjekty je podle slovenského práva rozdělen takto.
Největší nárůst administrativních úkolů, tj. o 5,3\%, byl zaznamenán v letech 2012 a 2013, což bylo způsobeno vyšším počtem pohybů prováděných v systému EMCS.
Naopak největší pokles oproti roku byl zaznamenán v letech 2013 a 2014, konkrétně o 4,3\%, což bylo způsobeno menším počtem dokončených daňových auditů a menším počtem předložených daňových přiznání.
Pokles byl částečně způsoben změnami právních předpisů.


Daňové audity

Cílem daňových auditů prováděných celním úřadem je zjistit nebo přezkoumat skutečnosti nezbytné pro řádné posouzení cel.
Při provádění daňových auditů předkládají zaměstnanci celního úřadu povolení k daňovému auditu, aby se identifikovali. Toto povolení obsahuje informace o účelu a předmětu auditu a obdobích, na něž se audit vztahuje.
V praxi to znamená, že auditované období může být stejně krátké jako jeden měsíc nebo dokonce dlouhé jako několik daňových období.
Rozsah kontrolovaných období určuje celní úřad.
Daňový audit začíná tím, že se připravuje zpráva o zahájení daňového auditu.
Doklady obdržené nebo přijaté celním úřadem se potvrzují potvrzením o přijetí.


\end{document}
