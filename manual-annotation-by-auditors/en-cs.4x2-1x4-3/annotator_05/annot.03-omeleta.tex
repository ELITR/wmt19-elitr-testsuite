\documentclass[10pt]{article}

%\usepackage{times}
\usepackage{fancyhdr}

\pagestyle{fancy}
\fancyhf{}
\rhead{\textbf{Kód: omeleta}}

\begin{document}

\subsection*{Zdroj:}

The audit was expanded by including October and November 2005 and by the movement of tobacco products under the duty-suspension arrangement in these months.
Tables 1 and 2 indicate the total amount of products subject to excise duty, which were transported under the duty-suspension arrangement between the Czech and Slovak tax entities in the audited period.


4. Comparison of registers of the movement of products under the duty-suspension arrangement

The SAO, CR handed over the database of movements ending in the Czech territory to the Slovak Party. It contained a total of 5,876 items from the accompanying documents, of which 5,738 items referred to mineral oils, 134 items referred to spirits and 4 items to tobacco products.
The SAO SR handed over the database of movements ending in the SR to the Czech Party. The database contained a total of 3,319 items from the accompanying documents, of which 696 items referred to mineral oil, 2,066 items referred to spirits and 557 items to tobacco products.
The SAO, CR made a comparison of the data in the Register of the movements started in the CR with the data on movements ended in the SR.


The results of this comparison are indicated in the Table 3:

The SAO, CR compared data on movements in November 2004 and September 2005 with the Database of the Slovak Customs submitted by the SAO SR at the beginning of the audit.
The database did not contain accurate data making it difficult to rely upon the confirmation of individual movements and thus only 63 \% of started movements were confirmed.
Individual movements were confirmed on the basis of various data, e.g. consignor, consignee, date of transportation, amount of transported products.
The SAO, CR compared data on movements in October and November 2005 started in the CR and ended in the SR with the Slovak Customs Database. Since all the data in the Database were in accordance with the accompanying documents, 93 \% of started movements were confirmed outright.
In the Czech Customs Database the volume of mineral oil was indicated in litres or tonnes, depending on the type of mineral oil.
The volume of spirits was in the majority of cases indicated in litres, which could be converted to litres of 100 \% alcohol by the alcohol content percentage indicated in the Database.


\pagebreak

\subsection*{Překlad:}

V průběhu kontrol byla prověřovaná období rozšířena o měsíce říjen a listopad 2005 a kontrole byly v těchto měsících podrobeny i dopravy tabákových výrobků v režimu podmíněného osvobození od daně.
V tabulkách č. 1 a č. 2 je uvedeno celkové množství výrobků podléhajících spotřební dani, které byly dopravovány v režimu podmíněného osvobození od daně mezi českými a slovenskými daňovými subjekty v kontrolovaném období.


4. Porovnání evidencí doprav zboží v režimu podmíněného osvobození od daně

NKÚ-ČR předal slovenské straně databázi doprav ukončených v ČR, která obsahovala celkem 5 876 položek uvedených na průvodních dokladech, z toho 5 738 položek se týkalo minerálních olejů, 134 položek lihu a 4 položky tabákových výrobků.
NKÚ SR předal české straně databázi doprav ukončených v SR, která obsahovala celkem 3 319 položek uvedených na průvodních dokladech, z toho 696 položek se týkalo minerálního oleje, 2 066 položek lihu a 557 položek tabákových výrobků.
NKÚ-ČR porovnal údaje o dopravách uvedených v evidenci zahájených doprav v ČR s údaji o ukončených dopravách evidovaných v SR.
Výsledky porovnání jsou uvedeny v tabulce č. 3.
Údaje o dopravách v měsících listopad 2004 a září 2005 porovnával NKÚ-ČR s databází slovenské celní správy, kterou mu NKÚ SR předal na začátku kontroly.
Databáze neobsahovala údaje umožňující spolehlivě odsouhlasit jednotlivé dopravy, a proto se podařilo přiřadit pouze 63 \% zahájených doprav.
Jednotlivé dopravy byly přiřazovány podle více údajů, např. odesílatel, příjemce, datum dopravy, množství dopravovaných výrobků.
Údaje o dopravách zahájených v ČR a ukončených v SR v měsících říjen a listopad 2005 porovnával NKÚ-ČR s databází slovenské celní správy, která obsahovala veškeré údaje podle průvodních dokladů, na jejichž základě se podařilo jednoznačně přiřadit 93 \% zahájených doprav.
V databázi Celní správy ČR byly údaje o množství minerálních olejů uvedeny v litrech nebo v tunách v závislosti na druhu minerálního oleje.
Množství lihu bylo ve většině případů uvedeno v litrech, které však bylo možné na základě v databázi uvedeného procentuálního obsahu alkoholu přepočítat na litry absolutního alkoholu.


\end{document}
