\documentclass[10pt]{article}

%\usepackage{times}
\usepackage{fancyhdr}

\pagestyle{fancy}
\fancyhf{}
\rhead{\textbf{Kód: obluda}}

\begin{document}

\subsection*{Zdroj:}



In the following cases taxpayers declared intra-Community transactions, which were probably not realised in fact:

18.
The Czech tax administrator obtained a request from the German tax administration for information on whether the Czech taxpayer acquired a passenger car at a value of EUR 47,400 from the German taxpayer in February 2005.
The request included an annex consisting in a copy of the VAT registration certificate of the Czech taxpayer.
It also included a copy of the passport of the person who was said to have performed the transport of this automobile.
The Czech tax administrator found in the investigation that this was a stolen passport, that the person mentioned in it did not import any vehicle into the CR and he stated that he did not have any contacts with a German tax entity.
The Czech tax administrator never issued the certificate of VAT registration, whose copy was sent by the German party.
The Czech tax administrator entered these data in the answer to the RFI together with the fact that he cannot confirm the acquisition of the goods – the automobile in question.
Through a question to the competent local tax administrator in the CR, the SAO found that the Czech taxpayer does not have his seat at the address listed in the Commercial Register and the executive of the company could not be contacted.
After performance of a tax objection proceeding for the taxation period of February 2005, the Czech tax administrator assessed the Czech taxpayer tax obligation in the amount of CZK 0.
In a further investigation, the Czech tax administrator found that the relevant automobile was sold to a further Czech entity, who purchased it through an Internet advertisement.
BRH found that, on the basis of the Czech answer to the RFI, the competent local German tax administrator did not recognise the claim of the German taxpayer to VAT exemption for this supply.
19. During an audit a German tax office was not able to clarify a difference of EUR 3,000 between the amount of acquired goods declared in the taxpayer’s tax return and the amount submitted from other Member States.
A Czech taxpayer was said to have supplied goods at a value of EUR 89,424 to a German taxpayer during the 4th quarter of 2004 and 1st quarter of 2005.


\pagebreak

\subsection*{Překlad:}



V následujících případech daňoví poplatníci nahlásili plnění uvnitř Společenství, která pravděpodobně nebyla uskutečněna:

18.
Český daňový správce obdržel od německého daňového úřadu žádost o informace o tom, zda český daňový poplatník získal od německého daňového poplatníka v únoru 2005 osobní automobil v hodnotě 47 400 EUR.
Žádost obsahovala přílohu obsahující kopii osvědčení o registraci českého daňového poplatníka pro účely DPH.
Obsahovala také kopii pasu osoby, která údajně provedla přepravu tohoto automobilu.
Český daňový správce při šetření zjistil, že se jedná o odcizený pas, že osoba v něm uvedená nedovážela do ČR žádné vozidlo a uvedl, že neměl žádné kontakty s německým daňovým subjektem.
Český daňový správce nikdy nevydal osvědčení o registraci k DPH, jehož kopii zaslala německá strana.
Český daňový správce vložil tyto údaje do odpovědi na finanční správu spolu se skutečností, že nemůže potvrdit pořízení zboží – dotčeného automobilu.
NKÚ prostřednictvím dotazu na příslušného místního daňového správce v ČR zjistil, že český daňový poplatník nemá své sídlo na adrese uvedené v obchodním rejstříku a nelze kontaktovat výkonného ředitele společnosti.
Po podání daňové námitky za zdaňovací období února 2005 vyměřil český daňový správce daňovou povinnost českého daňového poplatníka ve výši 0 Kč.
V dalším šetření český daňový správce zjistil, že příslušný automobil byl prodán dalšímu českému subjektu, který jej zakoupil přes internetovou reklamu.
Společnost BRH zjistila, že příslušný německý daňový správce na základě české odpovědi na adresu RFI neuznal nárok německého daňového poplatníka na osvobození od DPH za toto dodání.
19. Při auditu nebyl německý daňový úřad schopen objasnit rozdíl ve výši 3 000 EUR mezi částkou pořízeného zboží uvedenou v daňovém přiznání daňového poplatníka a částkou předloženou z jiných členských států.
Český daňový poplatník údajně dodal zboží v hodnotě 89 424 EUR německému daňovému poplatníkovi během čtvrtého čtvrtletí roku 2004 a prvního čtvrtletí roku 2005.


\end{document}
