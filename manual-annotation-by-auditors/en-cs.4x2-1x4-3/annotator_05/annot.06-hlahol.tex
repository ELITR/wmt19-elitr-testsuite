\documentclass[10pt]{article}

%\usepackage{times}
\usepackage{fancyhdr}

\pagestyle{fancy}
\fancyhf{}
\rhead{\textbf{Kód: hlahol}}

\begin{document}

\subsection*{Zdroj:}

When the tax administrator has verified that the tax entity meets the conditions for registration, he shall issue a decision in the form of a certificate stating that the tax entity is registered and that it has been allocated a tax identification number (hereinafter the “ID”).
In the certificate the tax administrator shall state the date when the registration for VAT became effective and the periodicity of the tax period for VAT.
The taxable period is one calendar quarter unless the taxpayer’s turnover for the previous calendar year reached the amount of CZK 10,000,000 (EUR 335,750).
If the taxpayer’s turnover for the previous calendar year reached the amount of CZK 10,000,000 (EUR 335,750), the taxable period is one calendar month.
A taxpayer, whose turnover for the previous calendar year attained an amount of CZK 2,000,000 (EUR 67,150), is entitled to choose a calendar month as the tax period and shall notify the tax administrator of this change.
The assignment of an ID in the certificate of registration is always constitutive in character.
The tax entity is obliged to state it in all communications (including payments) with the tax administrator, and in those cases stipulated by a special law (e.g. Act No. 235/2004 Coll., in drawing up tax documents).
The ID contains the code “CZ” and a basic part, consisting of a general identifier.
For natural persons, the general identifier is the birth certificate number or some other general identifier if so stipulated by a special law, and for a legal person, the general identifier is the business identification number (provisions of § 33 (12) of Act No. 337/1992 Coll.).
If the tax entity has not fulfilled its obligation in spite of being requested to do so by the tax administrator, the competent local tax administrator shall register it at his own initiative.
The registration proceeding results in the issuing a Certificate of Registration ex officio, which shall be delivered to the tax entity.
In the CR, a tax entity that has acquired authorisation for a business activity is assigned an ID at the moment of his first registration for any kind of tax.
The tax entity also uses the same ID when he is registered as a VAT payer.


\pagebreak

\subsection*{Překlad:}

Pokud správce daně ověří, že daňový subjekt splňuje podmínky pro registraci, vydá rozhodnutí ve formě potvrzení, které uvádí, že daňový subjekt je registrován a že mu bylo přiděleno daňové identifikační číslo (dále jen ‚identifikační číslo‘).
V osvědčení správce daně uvede datum, kdy registrace pro účely DPH nabyla účinnosti, a periodicitu daňového období pro účely DPH.
Zdanitelné období je jedno kalendářní čtvrtletí, pokud obrat daňového poplatníka za předchozí kalendářní rok nedosáhl částky 10 000 000 CZK (335 750 EUR).
Pokud obrat daňového poplatníka za předchozí kalendářní rok dosáhl výše 10 000 000 CZK (335 750 EUR), zdaňovací období je jeden kalendářní měsíc.
Daňový poplatník, jehož obrat za předchozí kalendářní rok dosáhl částky 2 000 000 CZK (67 150 EUR), má právo zvolit jako zdaňovací období kalendářní měsíc a tuto změnu oznámí správci daně.
Přiřazení průkazu totožnosti v osvědčení o registraci má vždy konstitutivní charakter.
Daňový subjekt je povinen ji uvést ve všech sděleních (včetně plateb) správci daně a v případech stanovených zvláštním zákonem (např. zákon č. 235/2004 Sb., při sestavování daňových dokladů).
Průkaz totožnosti obsahuje kód ‚CZ‘ a jeho základní část sestávající z obecného identifikátoru.
U fyzických osob je obecným identifikátorem rodný list nebo jiný obecný identifikátor, pokud tak stanoví zvláštní zákon, a u právnických osob je obecným identifikátorem identifikační číslo podniku (ustanovení § 33 odst. 12 zákona č. 337/1992 Sb.).
Pokud daňový subjekt nesplnil svou povinnost, přestože o to byl požádán správcem daně, příslušný místní správce daně jej zaregistruje z vlastního podnětu.
Výsledkem registračního řízení je vydání osvědčení o registraci z moci úřední, které se doručí daňovému subjektu.
Daňovému subjektu, který získal povolení k podnikatelské činnosti, je v ČR v okamžiku jeho první registrace přidělen průkaz totožnosti pro jakýkoli druh daně.
Daňový subjekt rovněž používá stejné ID, je-li registrován jako plátce DPH.


\end{document}
