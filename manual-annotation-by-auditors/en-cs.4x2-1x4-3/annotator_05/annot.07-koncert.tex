\documentclass[10pt]{article}

%\usepackage{times}
\usepackage{fancyhdr}

\pagestyle{fancy}
\fancyhf{}
\rhead{\textbf{Kód: koncert}}

\begin{document}

\subsection*{Zdroj:}

If the importer of consumer packaged alcohol is at the same time a person authorised to distribute ethyl alcohol, he is obliged to inform the customs office electronically about the name of the consumer packaged alcohol, amount and concentration of ethyl alcohol by volume in the consumer packaged alcohol, and the EAN bar code for each new type of consumer packaged alcohol that he is placing into free tax circulation no later than on the day of placing the packaged goods into free circulation.


Merchant of consumer packaged alcohol in free circulation

For the purpose of monitoring ethyl alcohol in free circulation and eliminating tax evasion, legislators have introduced the concept of merchant of consumer packaged alcohol.
A person who wishes to be a holder of a sales permit has to request the customs office for the permit and satisfy statutory conditions (trade licence to sell goods, no record of intentional economic crimes, no revocation of his trade licence in the last ten years, being not in liquidation or in bankruptcy).
As consumer packaged alcohol is in free circulation, no accompanying document is required during the movement of this commodity.
An invoice containing the number of the distribution licence holder is sufficient to demonstrate how the consumer packaged alcohol was acquired.
For each establishment, the authorisation holder is obliged to keep for each calendar month: a record of the number of pieces of consumer packaged alcohol received, the number of pieces of consumer packaged alcohol dispatched, the state of inventory of consumer packaged alcohol as at the last day of the month, and a record of shortage or surpluses in the number of pieces of received and dispatched consumer packaged goods.
The authorisation holder also has the obligation to purchase or otherwise take off consumer packaged alcohol only from a person authorised to distribute it.


Distributor of consumer packaged alcohol

For the purpose of monitoring the movement of ethyl alcohol in free tax circulation and eliminating tax evasion, legislators have introduced the concept of distributor of consumer packaged alcohol.
A person who wishes to become a distribution licence holder has to apply to the customs office for a distribution licence.


\pagebreak

\subsection*{Překlad:}

Je-li dovozce spotřebitelského baleného alkoholu současně osobou oprávněnou k distribuci lihu, je povinen elektronicky informovat celní úřad o názvu spotřebitele baleného alkoholu, množství a koncentraci lihu v množství zabaleného ve spotřebitelském alkoholu a čárový kód EAN pro každý nový typ spotřebitele baleného alkoholu, který: uvede do volného oběhu daně nejpozději v den propuštění zabaleného zboží do volného oběhu.


Obchodník s alkoholem zabaleným ve volném oběhu

Za účelem sledování lihu ve volném oběhu a odstranění daňových úniků zavedly zákonodárci pojem obchodníka s alkoholem zabaleným ve spotřebitelských obalech.
Osoba, která si přeje být držitelem povolení k prodeji, musí požádat celní úřad o povolení a splňovat zákonné podmínky (obchodní licence k prodeji zboží, žádný záznam o úmyslných hospodářských trestných činech, žádné zrušení obchodní licence v posledních deseti letech, která není v likvidaci ani v úpadku).
Vzhledem k tomu, že alkohol zabalený ve spotřebitelských obalech je ve volném oběhu, není během pohybu této komodity vyžadován průvodní doklad.
Faktura obsahující počet držitelů licence k distribuci postačuje k prokázání toho, jak byl alkohol zabalený ve spotřebitelích získán.
Pro každé zařízení je držitel povolení povinen vést pro každý kalendářní měsíc záznamy o počtu kusů alkoholu v balení pro spotřebitele, o počtu kusů alkoholu v balení pro spotřebitele, o stavu zásob alkoholu v balení pro spotřebitele v poslední den měsíce a o záznamu nedostatku. nebo přebytky v počtu kusů zboží v balení pro spotřebitele, které bylo dodáno a odesláno.
Držitel povolení má rovněž povinnost zakoupit nebo jinak odebrat alkohol zabalený ve spotřebitelských obalech pouze od osoby oprávněné jej distribuovat.


Distributor alkoholu v balení pro spotřebitele

Za účelem sledování pohybu lihu ve volném oběhu daní a odstranění daňových úniků zavedly zákonodárci pojem „distributor alkoholu v balení pro spotřebitele“.
Osoba, která se chce stát držitelem licence k distribuci, musí požádat celní úřad o licenci k distribuci.


\end{document}
