\documentclass[10pt]{article}

%\usepackage{times}
\usepackage{fancyhdr}

\pagestyle{fancy}
\fancyhf{}
\rhead{\textbf{Kód: divadlo}}

\begin{document}

\subsection*{Zdroj:}

The taxpayer did not submit any further VAT return.
For the purpose of non-fulfilment of obligations following from the Act, its VAT registration was cancelled as of January 4, 2005.
According to information from the VIES system, this entity acquired goods in the value of SKK 7 846 thousand from a Slovak taxpayer in the 3rd and 4th quarters of 2004.
For the purpose of reviewing the performance of these transactions, the Czech tax administrator sent two requests for information to the SR.
By inquiry at the locally competent tax office, the SAO SR ascertained that the executive of the Czech company denied, in his testimony, that the reviewed transactions actually took place and contested the authenticity of his signature on invoices and other documents.
The intra-Community supply to the CR, as reported by the Slovak taxpayer, was challenged on the basis of tax proceedings performed in the SR.
The taxpayer’s claim for VAT exemption will not be recognized and VAT will be assessed on the aforementioned amount.


o

According to information from the VIES system, a Czech taxpayer acquired goods (working clothes) from the SR in the total value of SKK 39 970 thousand in the 1st quarter of 2005.
In that quarter, the Czech taxpayer accounted for acquisition of goods from another EU member country only in the VAT return for the month of March 2005, in the value of CZK 11 292 thousand.
The locally competent tax administrator performed local investigation of the relevant entity, aimed at reviewing data stated in the VAT return.
As the company failed to submit any documents or otherwise prove the taxable fulfilments, on June 9, 2005, the tax administrator assessed VAT in the amount of CZK 0.
Given the doubts regarding the actual acquisition of goods, in May 2005, the tax administrator sent two requests for information to the SR.
According to the response to the requests for information, the tax administrator in the SR had doubts as regards the performance of the transactions, as well as the payment.
The SAO SR stated that the Slovak tax administrator sent to the TD CLO instigation to perform simultaneous audits, together with the tax administration of the CR, which will be carried out at the participating entities.


\pagebreak

\subsection*{Překlad:}

Daňový poplatník nepředložil žádné další přiznání k DPH.
Za účelem nesplnění povinností vyplývajících ze zákona byla jeho registrace k DPH od 4. 1. 2005 zrušena.
Podle informací ze systému VIES tento subjekt ve 3. a 4. čtvrtletí 2004 získal od slovenského daňového poplatníka zboží v hodnotě 7 846 tis.
Za účelem přezkoumání plnění těchto transakcí zaslal český správce daně do SR dvě žádosti o informace.
Pátráním na místně příslušném daňovém úřadu NKÚ SR zjistil, že exekutiva české společnosti ve svém svědectví popřela, že provedené prověrky skutečně proběhly a zpochybnily pravost jeho podpisu na fakturách a jiných dokladech.
Dodávka uvnitř Společenství do ČR, jak uvádí slovenský daňový poplatník, byla zpochybněna na základě daňových řízení vedených v SR.
Žádost daňového poplatníka o osvobození od DPH nebude uznána a DPH bude vyčíslena z výše uvedené částky.


Ó

Podle informací ze systému VIES získal český poplatník v 1. čtvrtletí 2005 zboží (pracovní oděvy) ze SR v celkové hodnotě 39 970 tis.
V tomto čtvrtletí se český plátce podílel na pořízení zboží z jiné členské země EU pouze v přiznání k dani za měsíc březen 2005 v hodnotě 11 292 tis. Kč.
Místně příslušný správce daně provedl místní šetření příslušného subjektu, jehož cílem bylo přezkoumat údaje uvedené v přiznání k dani z přidané hodnoty.
Vzhledem k tomu, že společnost nepředložila žádné doklady ani neprokázala zdanitelné plnění, dne 9. června 2005 správce daně vyměřil DPH ve výši 0 Kč.
Vzhledem k pochybnostem o skutečném pořízení zboží zaslal správce daně v květnu 2005 dvě žádosti o informace do SR.
Dle odpovědi na žádosti o informace měl správce daně v SR pochybnosti, pokud jde o plnění transakcí i platbu.
NKÚ SR uvedl, že slovenský správce daně zaslal podnětu TD CLO k provedení souběžných auditů spolu s daňovou správou ČR, která bude provedena u zúčastněných subjektů.


\end{document}
