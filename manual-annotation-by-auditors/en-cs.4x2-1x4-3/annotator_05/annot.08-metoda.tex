\documentclass[10pt]{article}

%\usepackage{times}
\usepackage{fancyhdr}

\pagestyle{fancy}
\fancyhf{}
\rhead{\textbf{Kód: metoda}}

\begin{document}

\subsection*{Zdroj:}



Summarised evaluation and recommendations

The parallel audits of the NKU and the BRH focused on two issues which are essential both in the Czech Republic and in Germany for the awarding of public works contracts: the application of EU procurement law and the prevention of corruption.
The provisions of EU law provide the opportunity for international competition within the EU in order to create a European Single Market.
Measures designed to prevent corruption do not only aim at avoiding or at least reducing corruption. They are also to prevent financial damage and potential loss of confidence in the integrity of government.
The audit findings generated by the two SAIs with respect to the application of procurement law in both countries and the comparison of the respective provisions of procurement law provide an insight into the common and differing features that may be helpful for both countries in further developing their procurement law and its application.


Application of EU procurement law

Both countries have transposed into national law the EU legislation concerning the awarding of public works contracts.
Thus, an important common basis exists in both countries for the awarding of public works contracts and the compliance with the essential requirements in this field: competition; equal treatment of the participating enterprises and transparency of the contract award procedure.
The construction administrations of both countries have underpinned these requirements by partly very extensive administrative regulations.
The Czech Republic has largely adopted the provisions on public works contracts requiring EU-wide tendering for those contract award procedures that are below the EU threshold and are therefore governed by national law.
However, the SAIs‘ audits revealed that the construction administrations did not always fully comply with all relevant provisions of EU-law.
The construction administration in Germany should continue to improve the quality of its contract award procedures in order to eliminate the deficiencies pointed out with respect to the implementation of EU requirements.
In the Czech Republic, the audits furthermore identified the risks impacting on the value for money achieved in contract award procedures.
For instance, incomplete specification of the public contract subject, resulting in need of extra works, often awarded in contradiction to the law. Furthermore, insufficient specification of the expected price was provided. Also, qualification requirements limited the possible number of bidders, thus restricting competition, namely in the area of transport constructions.


\pagebreak

\subsection*{Překlad:}



Souhrnné hodnocení a doporučení

Paralelní kontroly NKÚ a BRH byly zaměřeny na dvě oblasti, které jsou pro proces zadávání veřejných zakázek podstatné jak v České republice, tak v Německu, a to aplikace zadávacího práva EU a prevence korupce.
Legislativa EU dává příležitost mezinárodní konkurenci v rámci EU s cílem vytvoření jednotného evropského trhu.
Cílem opatření směřujících k prevenci korupce není pouze vyhnout se jí nebo ji alespoň omezit, ale zároveň zabránit finančním škodám a možným ztrátám důvěry v integritu vlády.
Shledaná kontrolní zjištění obou kontrolních institucí – NKÚ a BRH – s ohledem na aplikaci zadávacího práva obou zemí a srovnání jeho příslušných ustanovení dávají vnitřní pohled na společné a odlišné prvky, které mohou být oběma zemím nápomocny v dalším vývoji a aplikaci tohoto práva.


Aplikace zadávacího práva EU

Obě země transponovaly právo EU týkající se zadávání veřejných zakázek, do svých národních legislativ.
V obou zemích tak existuje důležitá společná základna pro zadávání veřejných zakázek a naplnění základních požadavků v oblasti: konkurence; rovného zacházení se zájemci; transparentnosti zadávacích řízení.
Zadavatelé posilují tyto požadavky částečně velmi rozsáhlými opatřeními.
Česká republika vztáhla ustanovení o veřejných zakázkách vyžadující evropské řízení i na řízení, která jsou pod limity EU, a řídí se tudíž národní legislativou.
Kontroly nicméně prokázaly, že zadavatelé nejsou vždy plně v souladu se všemi relevantními ustanoveními požadavků EU.
Němečtí zadavatelé by měli pokračovat ve zlepšování kvality svých zadávacích řízení, aby eliminovali zmiňované nesrovnalosti s přihlédnutím k implementaci požadavků EU.
V České republice byla kontrolami identifikována rizika ovlivňující hospodárnost a dosažení hodnoty za peníze.
Jedná se zejména o neúplné vymezení předmětu veřejné zakázky, které mělo za následek dodatečné práce, které byly často zadávány v rozporu se zákonem, nedostatečné vymezení předpokládané ceny a dále o požadované kvalifikační předpoklady zužující možný okruh uchazečů a tím snižující konkurenci zejména v oblasti dopravních staveb.


\end{document}
