\documentclass[10pt]{article}

%\usepackage{times}
\usepackage{fancyhdr}

\pagestyle{fancy}
\fancyhf{}
\rhead{\textbf{Kód: ilustr}}

\begin{document}

\subsection*{Zdroj:}

— As to SAD items, in respect whereof differences between the Czech Customs Database and the databases of Tax Offices were found to exist, printouts of appropriate SADs, including invoices attached thereto, were requested from General Directorate of Customs of the Czech Republic.
The audit established, inter alia, that the SAD databases of Tax Offices did not contain, for example, records on import- and export-related related SADs corresponding to aggregate VAT values of CZK 3,622,000.- and CZK 9,359,000.-, respectively.


5.2.1.1. Differences Established by the Audit



A comparison of SAD databases at Tax Offices and cumulative tax liabilities in 2002 and 2003 according to VAT Return Forms of a selected sample of taxpaying entities identified the following differences:



The differences are attributable especially to:

- The time difference between the physical completion of the import transaction and the date of the refund claim with respect to the VAT assessed on the imported goods.
The taxpaying entity could lodge the VAT refund claim with respect to imported goods within 3 years from the end of the fiscal period in which the goods had been placed under the appropriate procedure; insofar as the temporary use procedure is concerned, within 3 years from the end of the fiscal period in which the procedure was terminated or ended.
- The time difference between the physical completion of the export transaction and the date on which the taxpaying entity received the relevant SAD.
The taxpaying entity was obliged to present data on accomplished export transactions exempted from VAT upon leaving the country in the VAT Return Form, at the earliest in the fiscal period in which the goods in question had left the country and subject to the taxpaying entity having received a tax document, i.e. the SAD.
- Non-compliance of taxpaying entities with their duties and obligations concerning reporting of import and export transactions in their VAT Return Forms.
In some cases, taxpaying entities had failed to report correct data on import and export transactions in their VAT Return Forms.
As taxpaying entities did not have to attach SAD lists to their VAT Return Forms, the possibility of checking and verifying VAT refund or VAT exemption claims by Tax Offices was rendered more difficult.


5.2.2. Cooperation of Tax and Customs Authorities



\pagebreak

\subsection*{Překlad:}

— Pokud jde o položky JSD, u nichž byly zjištěny rozdíly mezi českou celní databází a databázemi daňových úřadů, byly od Generálního ředitelství cel České republiky vyžádány výtisky příslušných JSD, včetně k nim přiložených faktur.
Audit mimo jiné zjistil, že databáze JSD daňových úřadů neobsahovaly například záznamy o JSD souvisejících s dovozem a vývozem, které by odpovídaly úhrnným hodnotám DPH ve výši 3 622 000,- resp. 9 359 000,- Kč.


5.2.1.1 Rozdíly zjištěné auditem



Srovnání databází JSD u daňových úřadů a kumulativních daňových závazků v letech 2002 a 2003 podle formuláře pro vrácení DPH u vybraného vzorku subjektů povinných k dani určilo tyto rozdíly:



Rozdíly lze přičíst zejména:

- Časový rozdíl mezi fyzickým dokončením dovozního plnění a datem žádosti o vrácení DPH vyměřené za dovezené zboží.
Subjekt povinný k dani mohl podat žádost o vrácení DPH u dovezeného zboží do tří let od konce daňového období, v němž bylo zboží propuštěno do příslušného režimu; pokud jde o postup dočasného použití, do tří let od konce daňového období, v němž byl režim ukončen nebo ukončen.
- Časový rozdíl mezi fyzickým dokončením vývozní transakce a datem, kdy daňový subjekt obdržel příslušný JSD.
Subjekt povinný k dani byl povinen předložit údaje o uskutečněných vývozních plněních osvobozených od DPH po opuštění země ve formuláři pro vrácení DPH, a to nejdříve v daňovém období, v němž dotyčné zboží opustilo zemi, a podléhat tomu, že subjekt povinný k dani obdržel daňový doklad, tj. JSD.
- Nedodržení povinností a povinností subjektů povinných k dani, pokud jde o vykazování dovozních a vývozních plnění ve formulářích pro vrácení DPH.
V některých případech subjekty povinné k dani nevykazovaly ve svých formulářích pro vrácení DPH správné údaje o dovozních a vývozních plněních.
Vzhledem k tomu, že subjekty povinné k dani nemusely přikládat seznamy JSD ke svým formulářům pro vrácení DPH, ztížila se možnost kontroly a ověřování vrácení DPH nebo žádostí o osvobození od DPH ze strany daňových úřadů.


5.2.2 Spolupráce daňových a celních orgánů



\end{document}
