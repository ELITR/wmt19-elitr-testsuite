\documentclass[10pt]{article}

%\usepackage{times}
\usepackage{fancyhdr}

\pagestyle{fancy}
\fancyhf{}
\rhead{\textbf{Kód: leopard}}

\begin{document}

\subsection*{Zdroj:}

The objective of the RFI is, e.g. to demonstrate performance of a given transaction in the framework of intra-Community performances, verification of a certain VAT ID, etc.
SIE involves provision of information that is important for a tax administrator in another EU member country.
Cooperation in the area of information exchange is governed by Council Regulation (EC) No. 1798/2003.
Pursuant to Art. 8 of the Regulation, the limit for response to a RFI equals 3 months.
The conditions for admissibility of requests are regulated in the Commission Decision of November 15, 2004 which stipulates that, unless the countries involved agree otherwise, the RFI must concern transactions with a value of at least EUR 1 500 excl. VAT.
If a RFI concerns a transaction with a value of less than EUR 15 000 excl. VAT, the requested authority may refuse to respond with respect to its workload.
The procedure in dealing with received RFI by the tax administration of the CR, including specification of internal time limits stipulated for processing at individual organizational levels, is depicted in the following chart.
In the SR, the time limit for responding to a RFI by the tax offices was internally set at 2 months so that the three-month limit pursuant to Council Regulation (EC) No. 1798/2003 can be complied with.
The MF CLO received a total of 29 RFI and 3 SIEs from the SR in the period from May 1, 2004 to June 30, 2005.
Of 29 RFI received, 26 (i.e. 89.7 \%) were dealt with within the set deadline of 3 months and 3 (i.e. 10.3 \%) were dealt with after the deadline.
During the same period, the TD CLO received a total of 71 RFI and 7 SIE from the CR.
Of 71 RFI received, 64 (i.e. 90.1 \%) were dealt with within the set deadline of 3 months and 5 (i.e. 7.0 \%) were dealt with after the deadline (in 2 remaining cases, it was not possible to determine the deadline on the basis of the submitted materials).
More detailed information on the duration of addressing RFI accepted by the MF CLO or TD CLO, as appropriate, is given in Graph No. 1.
The reasons for sending RFI, as stated in the SCAC forms, are described in the following table.


\pagebreak

\subsection*{Překlad:}

Cílem RFI je např. prokázat výkonnost dané transakce v rámci výkonů uvnitř Společenství, ověření určitého DIČ atd.
SIE zahrnuje poskytování informací, které jsou důležité pro správce daně v jiné členské zemi EU.
Spolupráce v oblasti výměny informací se řídí nařízením Rady (ES) č. 1798/2003.
Podle článku 8 nařízení se limit pro odezvu na RFI rovná třem měsícům.
Podmínky přípustnosti žádostí jsou upraveny v rozhodnutí Komise ze dne 15. listopadu 2004, které stanoví, že pokud se zúčastněné země nedohodnou jinak, musí se finanční instituce týkat transakcí v hodnotě nejméně 1 500 EUR bez DPH.
Pokud se finanční instituce týká transakce v hodnotě nižší než 15 000 EUR bez DPH, může dožádaný orgán odmítnout odpovědět na její pracovní zátěž.
Postup při vyřizování obdržených finančních institucí daňovou správou ČR, včetně upřesnění vnitřních lhůt stanovených pro zpracování na individuální organizační úrovni, je znázorněn v následujícím grafu.
V SR byla lhůta pro odpověď finančních institucí daňovými úřady interně stanovena na 2 měsíce, aby bylo možné dodržet tříměsíční lhůtu podle nařízení Rady (ES) č. 1798/2003.
V období od 1. května 2004 do 30. června 2005 obdržel CLO pro MF od SR celkem 29 finančních nástrojů a 3 SIE.
Z 29 obdržených finančních nástrojů bylo 26 (tj. 89,7 \%) vyřízeno ve stanovené lhůtě tří měsíců a 3 (tj. 10,3 \%) byly vyřízeny po stanovené lhůtě.
Ve stejném období obdržel CLO pro TD celkem 71 finančních nástrojů a 7 SIE od ČR.
Ze 71 obdržených RFI bylo 64 (tj. 90,1 \%) vyřízeno ve stanovené lhůtě tří měsíců a pět (tj. 7,0 \%) bylo vyřízeno po stanovené lhůtě (ve dvou zbývajících případech nebylo možné určit lhůtu na základě předložených materiálů).
Podrobnější informace o době trvání vyřízení RFI, kterou přijal subjekt kolektivního investování nebo případně subjekt kolektivního investování, jsou uvedeny v grafu č. 1.
Důvody pro zaslání RFI, jak jsou uvedeny ve formulářích SCAC, jsou popsány v následující tabulce.


\end{document}
