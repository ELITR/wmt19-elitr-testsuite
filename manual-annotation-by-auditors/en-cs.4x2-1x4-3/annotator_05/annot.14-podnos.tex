\documentclass[10pt]{article}

%\usepackage{times}
\usepackage{fancyhdr}

\pagestyle{fancy}
\fancyhf{}
\rhead{\textbf{Kód: podnos}}

\begin{document}

\subsection*{Zdroj:}

The applicant has to fulfil the statutory conditions (trade licence to sell goods, storage area of more than 200m2, annual turnover from the sale of consumer packaged alcohol of at least € 100,000 or paid up registered capital of at least € 100,000, at least 10 employees, no arrears payable to the customs office, no arrears on mandatory payments, no record of economic crime, no revocation of his licence in the last ten years, and being not in liquidation or bankruptcy).
As the consumer packaged alcohol is in free circulation, no accompanying document is required during movement of this commodity.
An invoice containing the number of the distribution licence holder is sufficient to demonstrate how the consumer packaged alcohol was acquired.
For each establishment, the authorisation holder is obliged to keep for each calendar year: a record of the number of pieces of consumer packaged alcohol received, the number of pieces of consumer packaged alcohol dispatched, the inventory of consumer packaged alcohol as at the last day of the month, and a record of shortages or surpluses in the number of pieces of received and dispatched consumer packaged goods.
The distribution licence holder also has the obligation to purchase or otherwise take off consumer packaged alcohol only from another person authorised to distribute it.
A distribution licence holder reports data for each calendar month electronically by the 25th day of the following calendar month.


Motor fuel merchant

To monitor motor fuels in free tax circulation and to eliminate tax evasion, a new type of tax entity was introduced: motor fuel merchant.
The Slovak Republic has no a special regulation on service stations.
For this reason, service stations are treated as motor fuel merchants under Slovak legislation.
A person who wishes to sell petrol, diesel, or LPG to end users as part of his business activities in the respective fiscal territory has to obtain a sales licence from the customs office first and present a document showing that the storage facility is certified and equipped with a suitable, certified metering device for measuring the amount of mineral oil received and dispatched. The storage tank also has to meet technical standards.
If the motor fuel merchant is also a warehouse keeper, authorised consignee, or authorised consigner, the customs office automatically also issues him a licence to sell motor fuels.


\pagebreak

\subsection*{Překlad:}

Žadatel musí splnit zákonné podmínky (živnostenské oprávnění k prodeji zboží, skladovací plochu větší než 200 m2, roční obrat z prodeje konzumního baleného alkoholu ve výši nejméně 100 000 € nebo zaplacený základní kapitál ve výši nejméně 100 000 €, minimálně 10 000 Kč). Zaměstnanci, žádné nedoplatky splatné celnímu úřadu, nedoplatky na povinných platbách, žádný záznam o hospodářské kriminalitě, žádné zrušení jeho licence v posledních deseti letech, a není v likvidaci nebo v konkursu).
Vzhledem k tomu, že spotřebitelský balený alkohol je ve volném oběhu, není při přepravě této komodity vyžadován žádný průvodní doklad.
Faktura obsahující číslo držitele distribuční licence postačuje k prokázání, jakým způsobem byl spotřebitel spotřebiteli zabalen.
Pro každé zařízení je držitel povolení povinen uchovávat pro každý kalendářní rok: počet přijatých kusů spotřebního baleného alkoholu, počet kusů spotřebovaného baleného alkoholu, inventář spotřebního baleného alkoholu k poslednímu dni a záznam o nedostatcích nebo přebytcích v počtu přijatých a odeslaných spotřebitelských balených výrobků.
Držitel licence na distribuci je rovněž povinen nakupovat nebo jinak odebírat spotřebitelský balený alkohol pouze od jiné osoby, která je oprávněna distribuovat tento alkohol.
Držitel distribuční licence podává údaje za každý kalendářní měsíc elektronicky do 25. dne následujícího kalendářního měsíce.


Obchodník s motorovým palivem

Pro sledování motorového paliva ve volném daňovém oběhu a odstranění daňových úniků byl zaveden nový typ daňového subjektu: obchodník s pohonnými hmotami.
Slovenská republika nemá zvláštní nařízení o čerpacích stanicích.
Z tohoto důvodu jsou čerpací stanice podle slovenské legislativy považovány za obchodníky s motorovými palivy.
Osoba, která si přeje prodávat benzín, naftu nebo LPG koncovým uživatelům v rámci své podnikatelské činnosti na příslušném daňovém území, musí nejprve získat od celního úřadu prodejní licenci a předložit doklad, že sklad je certifikován a vybaven s vhodným certifikovaným měřicím zařízením pro měření množství přijatého a odeslaného minerálního oleje. Zásobník musí také splňovat technické normy.
Pokud je obchodníkem s pohonnými hmotami také skladovatel, oprávněný příjemce nebo oprávněný odesílatel, celní úřad mu automaticky vydá také povolení k prodeji motorových paliv.


\end{document}
