\documentclass[10pt]{article}

%\usepackage{times}
\usepackage{fancyhdr}

\pagestyle{fancy}
\fancyhf{}
\rhead{\textbf{Kód: kaktus}}

\begin{document}

\subsection*{Zdroj:}

Act No 307/2013 Coll., on mandatory labelling of spirits, and the related implementing regulations set out the mandatory labelling of consumer packaged alcohol, the conditions for using alcohol excise stamps, and the conditions for handling ethyl alcohol (i.e., production, processing, sale or other transfer, carriage, movement, storage, receipt, use, holding, import, or export).
The procedures in duty proceedings that are not set out in special regulations are regulated by Act No 280/2009 Coll., the Tax Code.
Penalties are imposed according to Act No 500/2004 Coll., the Code of Administrative Procedure.
The conditions for issuing trade licences (concessions) for carrying out the business of “production and processing of fuel and lubricants and distribution of fuels” and “production, modification, and sale of fermented ethyl alcohol, potable ethyl alcohol, and alcoholic beverages” are regulated by Act No 455/1991 Coll., on trades (the Trade Licensing Act).
The Czech Republic made use of its right to introduce control procedures and concepts for checking the movement of specific products placed in free circulation and overseeing the observance of legal regulations on handling various products on which excise duty has not been paid and on handling various products on which excise duty has been paid.
This applies to the handling of special mineral oils and raw tobacco, marking certain other mineral oils, and labelling tobacco products and consumer packaged alcohol.
The Czech Republic made use of its rights to impose excise duty on other products as well and introduced excise duty on raw tobacco as one of the measures against tax fraud.


Structure of amendments to legislation on each type of excise duty in relation to the subject of the audit in the Czech Republic

Based on Resolution No 735/2012 of the Government of the Czech Republic of 3 October 2012 implementing the “plan for zero tolerance of the alcohol black market”, fundamental legislative changes took place in the Czech Republic between 2012 and 2015 leading to the implementation of measures against excise duty fraud on ethyl alcohol.
In the same period, several amendments were made to legislation regulating duty on mineral oil, duty on tobacco products, and the criminal offence of planning tax evasion.


Criminal law protection of taxable revenues



\pagebreak

\subsection*{Překlad:}

Zákon č. 307/2013 Sb., O povinném označování lihovin, a související prováděcí vyhlášky stanovily povinné označování spotřebitelského baleného alkoholu, podmínky pro používání alkoholických známek a podmínky pro nakládání s alkoholem (tj. Výroba, zpracování , prodej nebo jiný převod, přeprava, pohyb, skladování, příjem, použití, držení, dovoz nebo vývoz).
Postupy ve služebním řízení, které nejsou upraveny zvláštními předpisy, upravuje zákon č. 280/2009 Sb., Daňový řád.
Sankce jsou ukládány podle zákona č. 500/2004 Sb., Správního řádu.
Podmínky pro vydávání živnostenských oprávnění (koncesí) k podnikání „výroby a zpracování pohonných hmot a maziv a distribuce pohonných hmot“ a „výroba, úprava a prodej kvašeného lihu, konzumního ethanolu a alkoholických nápojů“ upraveno zákonem č. 455/1991 Sb., o živnostenském podnikání (živnostenský zákon).
Česká republika využila svého práva zavést kontrolní postupy a koncepty pro kontrolu pohybu konkrétních výrobků uváděných do volného oběhu a dohlížet na dodržování právních předpisů o manipulaci s různými výrobky, na které nebyla spotřební daň zaplacena, ao nakládání s různými výrobky na spotřební daně.
To platí pro manipulaci se speciálními minerálními oleji a surovým tabákem, označování některých dalších minerálních olejů a označování tabákových výrobků a spotřebního baleného alkoholu.
Česká republika využila svých práv na uložení spotřební daně z ostatních výrobků a zavedla spotřební daň ze surového tabáku jako jedno z opatření proti daňovým podvodům.


Struktura novely zákona o každém typu spotřební daně ve vztahu k předmětu auditu v České republice

Na základě usnesení vlády České republiky č. 735/2012 ze dne 3. října 2012, kterým se provádí „plán nulové tolerance trhu s alkoholickým černým alkoholem“, došlo v letech 2012 až 2015 k zásadním legislativním změnám, které vedly k realizaci opatření proti podvodům v oblasti spotřební daně z etylalkoholu.
Ve stejném období bylo provedeno několik úprav legislativy upravující clo na minerální olej, clo na tabákové výrobky a trestný čin plánování daňových úniků.


Trestněprávní ochrana zdanitelných příjmů



\end{document}
