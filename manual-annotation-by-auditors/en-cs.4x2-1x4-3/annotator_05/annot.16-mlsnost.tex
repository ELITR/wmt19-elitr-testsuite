\documentclass[10pt]{article}

%\usepackage{times}
\usepackage{fancyhdr}

\pagestyle{fancy}
\fancyhf{}
\rhead{\textbf{Kód: mlsnost}}

\begin{document}

\subsection*{Zdroj:}

We are convinced that the results of our audit may help not only to improve tax administrators´ work, but also serve as a basis for improving an international information exchange system.


Report on the Results of the Parallel Audit of the Administration of Value Added Tax in the Czech Republic and in the Federal Republic of Germany



Introduction



Audit number: (CZ) 06/27 (D) VIII 2 – 2005 – 1215



Subject of audit:



(CZ) Administration of value added tax following accession of the Czech Republic to the European Union



(D) Review of the administrative cooperation in the field of VAT according to the Council Regulation (EC) No 1798/2003 and the Commission Regulation (EC) No 1925/2004



Objective of audit:

(CZ) The objective of the audit in the Czech Republic was to review the procedure used by financial authorities in administrating value added tax following integration of the Czech Republic into the common internal market of the European Community (hereinafter “EC”), connected with free movement of goods and services, and to review the use of the VIES, particularly monitoring the exercise of the right to exempt intraCommunity deliveries from value added tax.
(D) The objective of the audit was to review the system of intraCommunity VAT control with a special focus on administrative cooperation in the field of VAT according to the above mentioned regulations.
As a result of this, weaknesses should be reported and recommendations be developed to address the problems stated.
Value Added Tax Information Exchange System The audits were performed by the Supreme Audit Office, Czech Republic (hereinafter the “SAO”) and by the German Bundesrechnungshof (hereinafter “BRH”) in mutual cooperation on the basis of an agreement concluded between the two audit institutions on June 8, 2006.
The cooperation primarily concerned the review of selected commercial transactions between taxpayers from the Czech Republic (hereinafter “CR”) and from the Federal Republic of Germany (hereinafter “Germany”), use of information obtained via international cooperation between tax administrations and comparison of the VAT administration systems in the two countries.
A joint final report on the results of the audits was processed in the sense of European Directive No. 31 for implementing the INTOSAI audit standards.
The audited period covered the period from May 1, 2004 to December 31, 2005.


Summary of results and evaluation of parallel audits



\pagebreak

\subsection*{Překlad:}

Jsme přesvědčeni, že výsledky našeho auditu mohou pomoci nejen zlepšit práci správců daní, ale také sloužit jako základ pro zlepšení mezinárodního systému výměny informací.


Zpráva o výsledcích paralelního auditu správy daně z přidané hodnoty v České republice a ve Spolkové republice Německo



Úvod



Číslo auditu: (CZ) 06/27 (D) VIII 2 – 2005 – 1215



Předmět auditu:



(CZ) Správa daně z přidané hodnoty po přistoupení České republiky k Evropské unii



D) Přezkum správní spolupráce v oblasti DPH podle nařízení Rady (ES) č. 1798/2003 a nařízení Komise (ES) č. 1925/2004



Cíl auditu:

Cílem auditu v České republice bylo přezkoumat postup používaný finančními orgány při správě daně z přidané hodnoty po začlenění České republiky do společného vnitřního trhu Evropského společenství (dále jen „ES“), který souvisí s volným pohybem zboží a služeb, a přezkoumat využívání systému VIES, zejména kontrolu výkonu práva na osvobození dodávek uvnitř Společenství od daně z přidané hodnoty.
D) Cílem auditu bylo přezkoumat systém kontroly DPH uvnitř Společenství se zvláštním zaměřením na správní spolupráci v oblasti DPH v souladu s výše uvedenými nařízeními.
V důsledku toho by měly být oznámeny nedostatky a měla by být vypracována doporučení k řešení uvedených problémů.
Systém výměny informací o dani z přidané hodnoty Kontroly prováděl Nejvyšší kontrolní úřad České republiky (dále jen „NKÚ“) a německý Bundesrechnungshof (dále jen „BRH“) ve vzájemné spolupráci na základě dohody uzavřené mezi oběma kontrolními orgány dne 8. června 2006.
Spolupráce se týkala především přezkumu vybraných obchodních transakcí mezi daňovými poplatníky z České republiky (dále jen „ČR“) a ze Spolkové republiky Německo (dále jen „Německo“), využití informací získaných prostřednictvím mezinárodní spolupráce mezi daňovými správami a srovnání systémů správy DPH v obou zemích.
Společná závěrečná zpráva o výsledcích auditů byla zpracována ve smyslu evropské směrnice č. 31 pro provádění auditních standardů INTOSAI.
Kontrolované období zahrnovalo období od 1. května 2004 do 31. prosince 2005.


Souhrn výsledků a hodnocení paralelních auditů



\end{document}
