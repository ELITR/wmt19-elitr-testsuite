\documentclass[10pt]{article}

%\usepackage{times}
\usepackage{fancyhdr}

\pagestyle{fancy}
\fancyhf{}
\rhead{\textbf{Kód: sillos}}

\begin{document}

\subsection*{Zdroj:}

As part of international exchange of information, the German tax administrator requested confirmation of whether the Czech taxpayer actually existed, because the alleged shareholder of the German taxpayer established a number of fictitious companies in Germany.
During reviewing of the business transaction by the tax office, the Czech taxpayer confirmed the acquisition of the goods.
However, according to a statement by the Czech taxpayer, the goods were not transported from Germany to the CR and were further sold in Germany during 2005.
The Czech taxpayer submitted documents on sale of the goods in January 2005.
The Czech taxpayer was not registered in Germany as a VAT payer and did not declare or pay taxes for the sale of the relevant goods neither in the CR nor Germany.
All these facts indicated tax evasion in the territory of Germany and the Czech tax administrator communicated these facts to the German tax administration in the answer to the RFI.
In an inquiry to the competent local tax administrator in the CR, the SAO found that, on the basis of a performed investigation, the Czech taxpayer submitted an additional VAT return for the 4th quarter of 2004, in which he included the value of goods acquired, i.a. from a German taxpayer and simultaneously exercised the right to tax deduction in acquisition of goods from other Member States.
He did not enter VAT on sale of goods in Germany to the Czech VAT return.
The Czech tax administrator assessed an additional VAT return following the local investigation without any changes.
According to the Czech legislation, the case in question did not consist in acquisition of goods from other Member States, as the goods were not sent or transported from another Member State to the CR.
Thus, the Czech tax administrator drew an erroneous conclusion in the local investigation.
The BRH found that, immediately after receiving the answer to the RFI, stating that the goods did not leave Germany, the German tax administrator corrected the tax return of the German taxpayer and requested payment of VAT of EUR 1,600.
Furthermore, the competent tax office sent a hint to the tax office of the real recipient of the goods to ensure that the further sale of these goods is properly taxed.


\pagebreak

\subsection*{Překlad:}

V rámci mezinárodní výměny informací požádal německý správce daně o potvrzení, zda český daňový poplatník skutečně existoval, protože údajný akcionář německého poplatníka založil v Německu řadu fiktivních společností.
Český daňový poplatník potvrdil při revizi obchodní transakce daňovým úřadem nabytí zboží.
Podle vyjádření českého daňového poplatníka však nebylo zboží přepravováno z Německa do ČR a v roce 2005 bylo dále prodáváno v Německu.
Český daňový poplatník předložil doklady o prodeji zboží v lednu 2005.
Český daňový poplatník nebyl v Německu registrován jako plátce DPH a nehlásil ani nehradil daně za prodej příslušného zboží ani v ČR, ani v Německu.
Všechny tyto skutečnosti naznačují daňové úniky na území Německa a český správce daně tyto skutečnosti sdělil německé daňové správě v odpovědi na RFI.
NKÚ zjistil v šetření na příslušném místním správci daně v ČR, že na základě provedeného šetření předložil český plátce dodatečné přiznání k DPH za 4. čtvrtletí 2004, ve kterém zahrnoval hodnotu nabytého zboží. , mimo jiné od německého poplatníka a současně uplatnila právo na odpočet daně při pořízení zboží z jiných členských států.
DPH z prodeje zboží v Německu nezadal do českého přiznání k DPH.
Český správce daně posoudil dodatečné přiznání k DPH po místním šetření bez jakýchkoliv změn.
Podle českého právního řádu se nejednalo o pořízení zboží z jiných členských států, neboť zboží nebylo odesláno ani přepraveno z jiného členského státu do ČR.
Český správce daně tak učinil chybný závěr v místním vyšetřování.
BRH zjistila, že bezprostředně po obdržení odpovědi na RFI s uvedením, že zboží neopustilo Německo, německý správce daně opravil daňové přiznání německého poplatníka a požádal o zaplacení DPH ve výši 1 600 EUR.
Příslušný daňový úřad dále zaslal finančnímu úřadu skutečného příjemce zboží nápovědu, aby zajistil, že další prodej tohoto zboží bude řádně zdaněn.


\end{document}
