\documentclass[10pt]{article}

%\usepackage{times}
\usepackage{fancyhdr}

\pagestyle{fancy}
\fancyhf{}
\rhead{\textbf{Kód: narcis}}

\begin{document}

\subsection*{Zdroj:}

The RMS should include other facts like the taxpayer’s payment behaviour, the (economic) branch or the shareholders.
A second level of RMS is processed by the tax administrations of the federal states and focuses on processing of VAT returns.
RMS for example generates a list of those taxpayers who declared intra-Community supplies in their VAT returns.
This list is sent to the CLO to monitor the complete submission of RS.
Furthermore, this type of RMS compares the intra-Community acquisitions declared in the taxpayer’s annual VAT return with the data of the VIES-system.


The tax administrations of the federal states avoid a comparison of the quarterly data because this comparison would disclose too many mismatches which may attributed to other reasons than tax evasion or tax fraud; for example:



– differing times when the supplier and the acquirer thought the intra-Community transaction was carried out;

– differing accounting and declaring of sales discounts and incentives.
In addition to comparing the VAT return data and VIES data many federal states implemented components of RMS based on the data of VAT returns, and some other facts (like the composition of the firm‘s name, the age of the shareholders, the number of employees and the taxpayer’s payment behaviour).
Bavaria implemented a rule-based RMS, permitting a team of tax fraud specialists (hereinafter the risk managers) to program their own rules to identify potential tax fraudsters.
This RMS is applied when the tax offices are processing the monthly VAT return.
So the tax office in charge receives suspected cases of the RMS before the VAT is fixed.
Since this RMS is not able to learn the risk managers have to adjust the rules on the basis of the feedback of the tax offices.
Mecklenburg-Western Pomerania has programmed an adaptive neural network with the data of tax files in which an audit disclosed tax evasions.
Based on these data the RMS is able to learn the characteristics of tax evasion.
The RMS produces a list of tax files monthly, which are suspected of being involved in tax evasion.
On the basis of this list the tax offices in charge decide to audit the proposed taxpayers.
After the audit the auditors have to report their findings, so the RMS can learn from the results of its list.


\pagebreak

\subsection*{Překlad:}

RMS by měl zahrnovat další skutečnosti, jako je platební chování daňového poplatníka, (ekonomická) pobočka nebo akcionáři.
Druhá úroveň RMS zpracovává daňové správy spolkových zemí a zaměřuje se na zpracování přiznání k DPH.
RMS například vytváří seznam těch daňových poplatníků, kteří ve svých přiznáních k DPH ohlašovali dodání zboží uvnitř Společenství.
Tento seznam je zaslán k rozhodnutí o registraci, aby bylo možné sledovat úplné podání RS.
Tento typ RMS navíc srovnává akvizice uvnitř Společenství uvedené v ročním přiznání k DPH daňového poplatníka s údaji systému VIES.


Daňové správy spolkových zemí se vyhýbají porovnání čtvrtletních údajů, protože toto srovnání by odhalilo příliš mnoho nesouladu, které mohou být způsobeny jinými důvody než daňovými úniky nebo daňovými podvody; například:



– Různé doby, kdy si dodavatel a nabyvatel mysleli, že transakce uvnitř Společenství byla provedena;

– Rozdílné účtování a vykazování prodejních slev a pobídek.
Kromě porovnání údajů o přiznání k DPH a údajů VIES zavedly mnohé spolkové země komponenty RMS na základě údajů o přiznání k DPH a některé další skutečnosti (jako je složení názvu firmy, věk akcionářů, počet zaměstnanců a platební chování daňového poplatníka).
Bavorsko zavedlo systém RMS založený na pravidlech, který umožňuje týmu odborníků na daňové podvody (dále jen „správci rizik“) naprogramovat svá vlastní pravidla pro identifikaci potenciálních daňových podvodníků.
Tento RMS se používá při zpracování měsíčního přiznání k DPH ze strany daňových úřadů.
Příslušný daňový úřad tak před stanovením DPH obdrží podezření na případy RMS.
Vzhledem k tomu, že tento RMS se nemůže naučit, musí správci rizik upravit pravidla na základě zpětné vazby daňových úřadů.
Mecklenburg-Vorpommersko naprogramovalo adaptivní neurální síť s údaji o daňových souborech, v nichž audit odhalil daňové úniky.
Na základě těchto údajů se RMS může seznámit s vlastnostmi daňových úniků.
RMS každý měsíc sestavuje seznam daňových spisů, u nichž je podezření, že jsou zapojeny do daňových úniků.
Na základě tohoto seznamu se příslušné daňové úřady rozhodnou provést audit navrhovaných daňových poplatníků.
Po provedení auditu musí auditoři oznámit svá zjištění, aby se RMS mohl poučit z výsledků svého seznamu.


\end{document}
