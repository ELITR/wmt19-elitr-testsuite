\documentclass[10pt]{article}

%\usepackage{times}
\usepackage{fancyhdr}

\pagestyle{fancy}
\fancyhf{}
\rhead{\textbf{Kód: povlak}}

\begin{document}

\subsection*{Zdroj:}



Selection of candidates in case of restricted tendering procedures

In contrast to public tendering procedures, restricted tendering procedures do not permit all potential contractors who are interested in winning the contract to participate in the competition.
Under restricted procedures, the contracting authority must previously select potential contractors that will be asked to submit a tender.


Requirements of EU law and of corruption prevention

The EU provisions permit a limitation of the number of candidates in EU-wide restricted tendering procedures.
However, restricted procedure requires at least five candidates to be asked to submit tenders and in the case of negotiated procedures with previous publication of a contract notice at least three potential contractors must be invited.
In any case, the number of invited candidates must be sufficiently large to guarantee genuine competition.
The selection criteria must be objective and nondiscriminatory and must be published in advance.
The selection of potential contractors to be invited is a stage within the tendering procedure which is highly vulnerable to manipulation, e.g. undue preference being given to certain contractors.
Moreover, the restriction of competition to few contractors may facilitate collusive tendering.
In both case, such practices may considerably deteriorate the result of thetendering procedure to the disadvantage of the contracting authority.
Therefore, the latter, must make special arrangements to prevent manipulations, to ensure that adequate controls are in place and that this stage in the procedure is transparent and plausible.


Provisions of German and Czech procurement law



Identical or similar provisions

The provisions on the selection of candidates are similar in the two countries.
In addition to EU requirements, the contracting authorities are to alternate among the contractors to be invited.
It is also not permitted to restrict competition to contractors domiciled in particular regions.
The contracting authorities have to document and justify the selection of bidders and the course of the selection procedure in the contract award memos.


Differing provisions

In Germany, it is required to invite at least three candidates in the case of restricted tendering and several suitable candidates in the case of awards by negotiated contract.
Except in two specific cases, the new Czech regulations do not allow limiting the number of invited tenderers in case of restricted and negotiated procedure.
In Germany, it is required to observe the crosscheck principle when selecting the candidates.


\pagebreak

\subsection*{Překlad:}



Výběr uchazečů v případě omezených nabídkových řízení

Na rozdíl od veřejných nabídkových řízení neumožňují omezené nabídkové řízení všem potenciálním dodavatelům, kteří mají zájem o získání zakázky, účastnit se výběrového řízení.
V rámci omezených řízení musí zadavatel předem vybrat potenciální dodavatele, kteří budou vyzváni k podání nabídky.


Požadavky právních předpisů EU a předcházení korupci

Ustanovení EU umožňují omezit počet uchazečů v rámci omezených nabídkových řízení na úrovni EU.
Omezené řízení však vyžaduje, aby nejméně pět uchazečů bylo požádáno o předložení nabídek, a v případě vyjednávacího řízení s předchozím zveřejněním oznámení o zahájení zadávacího řízení musí být pozváni nejméně tři potenciální dodavatelé.
V každém případě musí být počet pozvaných uchazečů dostatečně vysoký, aby byla zaručena skutečná hospodářská soutěž.
Kritéria výběru musí být objektivní a nediskriminační a musí být zveřejněna předem.
Výběr potenciálních dodavatelů, kteří mají být pozváni, je fází nabídkového řízení, která je velmi citlivá na manipulaci, např. nepatřičným upřednostňováním některých dodavatelů.
Omezení hospodářské soutěže na několik málo dodavatelů může navíc usnadnit tajné nabídkové řízení.
V obou případech mohou tyto praktiky výrazně zhoršit výsledek řízení o nesplnění povinnosti ve prospěch veřejného zadavatele.
Tyto orgány proto musí přijmout zvláštní opatření, aby zabránily manipulaci, zajistily, že budou zavedeny přiměřené kontroly a že tato fáze postupu bude transparentní a věrohodná.


Ustanovení německého a českého zákona o zadávání veřejných zakázek



Totožná nebo podobná ustanovení

Ustanovení o výběru kandidátů jsou v obou zemích podobná.
Kromě požadavků EU musí veřejní zadavatelé střídat dodavatele, kteří mají být pozváni.
Rovněž není dovoleno omezovat hospodářskou soutěž na dodavatele s bydlištěm v určitých regionech.
Veřejní zadavatelé musí v zadávací dokumentaci doložit a odůvodnit výběr uchazečů a průběh výběrového řízení.


Různá ustanovení

V Německu musí přizvat alespoň tři uchazeče v případě omezeného výběrového řízení a několik vhodných uchazečů v případě udělení na základě vyjednané smlouvy.
S výjimkou dvou zvláštních případů nové české právní předpisy neumožňují omezit počet pozvaných uchazečů v případě omezeného a vyjednávacího řízení.
V Německu musí při výběru uchazečů dodržovat zásadu „crosscheck“.


\end{document}
