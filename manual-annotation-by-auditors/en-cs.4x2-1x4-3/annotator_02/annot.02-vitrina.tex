\documentclass[10pt]{article}

%\usepackage{times}
\usepackage{fancyhdr}

\pagestyle{fancy}
\fancyhf{}
\rhead{\textbf{Kód: vitrina}}

\begin{document}

\subsection*{Zdroj:}

Through a question sent to the competent local tax administrator, the SAO found that the Czech tax administrator performed a VAT audit at the Czech taxpayer for the taxation periods of the 3rd quarter of 2004 to the 2nd quarter of 2005.
In this audit, the tax office found that the Czech taxpayer included in his VAT return services at a value of CZK 9,897 (EUR 314), provided to the German taxpayer prior to his VAT registration (i.e. prior to 1 October 2004).
The tax office subsequently assessed VAT of CZK 1,880 (EUR 60) for the tax entity for the taxation period of the 3rd quarter of 2004.
On the basis of a request from the SAO, the BRH found that the VAT ID was allocated to the German taxpayer only on 16 February 2005, when he informed the tax office about his business which was started in 2004.
Furthermore the tax office registered him retroactively for VAT as 1 October 200415 to monitor the submission of the annual VAT return of the year 2004.
In this VAT return the taxpayer declared intra-Community acquisitions from CR in the amount of EUR 160 (» CZK 5,000).
3. In the VAT return for September 2004 and RS for the 3rd quarter of 2004, the Czech taxpayer declared the supply of goods at a value of CZK 114,615 (EUR 3,642) to a German taxpayer.
In reviewing the registration of the German taxpayer in the VIES system, the Czech tax administrator found that his registration was valid from 11 November 2004, i.e. after supply of the goods to Germany by the Czech taxpayer.
The Czech taxpayer submitted an additional VAT return, in which he declared a tax base for performed domestic supplies at a value of CZK 114,615 (EUR 3,642) and VAT of CZK 21,777 (EUR 692), which he paid.
Through a question to the competent local tax office in the CR, the SAO found that the Czech taxpayer subsequently submitted a request for renewal of the proceedings in the matter of the additional VAT return and, as evidence, submitted the statement of the German tax administration that the German taxpayer had been registered for taxes since 1 January 2000.
Consequently, the Czech tax administrator reviewed the validity of registration of the German taxpayer as part of international exchange of information.


\pagebreak

\subsection*{Překlad:}

NKÚ dotazem u místně příslušného správce daně zjistil, že český správce daně provedl u českého plátce kontrolu DPH za zdaňovací období III. čtvrtletí 2004 až II. čtvrtletí 2005.
Touto kontrolou finanční úřad zjistil, že český plátce zahrnul do daňového přiznání k DPH služby v hodnotě 9 897 Kč (314 EUR), poskytnuté německému plátci v době před jeho registrací k DPH (tj. před 1. 10. 2004).
Za zdaňovací období III. čtvrtletí 2004 finanční úřad daňovému subjektu doměřil DPH ve výši 1 880 Kč (60 EUR).
BRH na základě požadavku NKÚ zjistil, německému plátci bylo VAT ID přiděleno až 16. 2. 2005, když informoval finanční úřad o svém podnikání, které zahájil v roce 2004.
Mimoto jej finanční úřad zaregistroval k DPH retrospektivně k 1. říjnu 200415, aby mohl sledovat podání ročního přiznání k DPH za rok 2004.
V tomto daňovém přiznání plátce vykázal intrakomunitární pořízení z ČR v hodnotě 160 EUR (cca 5000 Kč).
3. Český plátce vykázal v daňovém přiznání k DPH za září 2004 a SH za III. čtvrtletí 2004 dodání zboží v hodnotě 114 615 Kč (3 642 EUR) německému plátci.
Český správce daně při ověřování registrace německého plátce v systému VIES zjistil, že jeho registrace byla platná od 11. 11. 2004, tj. až po uskutečnění dodání zboží českým plátcem do SRN.
Český plátce podal dodatečné DAP k DPH, ve kterém vykázal základ daně z uskutečněných plnění v tuzemsku v hodnotě 114 615 Kč (3 642 EUR) a DPH ve výši 21 777 Kč (692 EUR), kterou uhradil.
NKÚ dotazem u místně příslušného finančního úřadu v ČR zjistil, že český plátce následně podal žádost o obnovu řízení ve věci dodatečného daňového přiznání k DPH a jako důkazní prostředek předložil potvrzení německé daňové správy o tom, že německý plátce byl daňově veden od 1. 1. 2000.
Proto český správce daně ověřoval platnost registrace německého plátce v rámci mezinárodní výměny informací.


\end{document}
