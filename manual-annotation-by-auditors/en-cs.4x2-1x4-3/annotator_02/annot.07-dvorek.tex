\documentclass[10pt]{article}

%\usepackage{times}
\usepackage{fancyhdr}

\pagestyle{fancy}
\fancyhf{}
\rhead{\textbf{Kód: dvorek}}

\begin{document}

\subsection*{Zdroj:}

Such notification should be sent no later than within three months of the date of acknowledgement of the receipt of the request.
Upon receipt of the request, the Czech Ministry of Finance shall verify if the request meets all legal requirements. After that, the Czech Ministry of Finance shall transfer the request through a financial directorate to the competent local tax office for processing.
In the event the request for recovery is accepted, the tax administrator shall i. a. identify the debtor's assets, e. g. through the extracts from the Cadastre of Real Estate, identification of bank accounts, registrations of motor vehicles with the Traffic Inspectorate etc.
When six months have elapsed from the date of the request receipt acknowledgement, the Czech Ministry of Finance should notify the applicant authority of the outcome of the investigation that has been undertaken in order to obtain the requested information.
The applicant authority should also be informed in the event that the tax entities in question cannot be contacted and the claim can therefore not be recovered.
Any amount successfully recovered by the tax administrator shall be transferred to the applicant authority within one month from the date on which such claim was recovered.
The Czech tax administrator making efforts to recover foreign claims may use an option to allow a moratorium on the payment of the claim or its payment by instalments, providing the applicant state gives its consent to that.
The interest or penalty charge is assessed to the debtor for the recovery period in accordance with Act No. 337/1992 Coll., on administration of taxes and fees.
Such interest or penal charges shall pertain to the state that has requested the recovery.
Concerning received requests for recovery, the Czech Ministry of Finance did not submit the requests to the competent tax administrators in time, particularly during 2006 and 2007.
A similar situation was identified with respect to the requests for international assistance for the recovery of the claims received from the tax administrators in the Czech Republic.
The audit revealed that the Czech Ministry of Finance did not proceed in accordance with the Act on the international assistance for the recovery of claims because it failed to notify a competent authority of the Member State of the progress made with the recovery within the six months time period from the day when the receipt of the request was acknowledged.


\pagebreak

\subsection*{Překlad:}

Toto sdělení musí být zasláno nejpozději do tří měsíců od data potvrzení obdržení žádosti.
Po přijetí žádosti české ministerstvo financí ověří, zda má žádost všechny náležitosti podle zákona, a mělo by žádost neprodleně předat prostřednictvím finančního ředitelství místně příslušnému finančnímu úřadu k vyřízení.
Jestliže je žádost o vymožení akceptována, správce daně mj. zjišťuje majetek dlužníka, např. pořízením výpisů z katastru nemovitostí, zjišťováním účtů v bankách, ověřením registrace vozidel na dopravním inspektorátu apod.
Po uplynutí šesti měsíců ode dne, kdy bylo potvrzeno přijetí žádosti, by mělo české ministerstvo financí dožadujícímu orgánu sdělit výsledek šetření, které bylo provedeno za účelem získání požadovaných informací.
V případě nekontaktních daňových subjektů, kdy není možné pohledávku vymoci, by měl být dožadující orgán o této skutečnosti též informován.
Každá částka vymožená správcem daně se převede dožadujícímu orgánu v CZK ve lhůtě jednoho měsíce ode dne, kdy byla pohledávka vymožena.
Není vyloučeno, aby český správce daně vymáhající zahraniční pohledávky povolil posečkání s jejím zaplacením nebo zaplacení ve splátkách, pokud s tím vysloví souhlas stát, který o vymáhání požádal.
Po dobu vymáhání je dlužníkovi předepisován úrok nebo penále podle zákona č. 337/1992 Sb., o správě daní a poplatků.
Tento úrok či penále náleží státu, který o vymáhání požádal.
V případě přijatých žádostí o vymáhání české ministerstvo financí nepostoupilo místně příslušným správcům daně žádosti k vyřízení včas, a to zejména v letech 2006 a 2007.
Podobná situace byla zjištěna v rámci mezinárodní pomoci při vymáhání pohledávek i v případě žádostí obdržených od správců daně z České republiky.
Kontrolní akce odhalila, že české ministerstvo financí nepostupovalo v souladu se zákonem č. 191/2004 Sb., protože neuvědomilo místně příslušný úřad členského státu o dosaženém pokroku při vymáhání do šesti měsíců od data, kdy bylo potvrzeno obdržení žádosti.


\end{document}
