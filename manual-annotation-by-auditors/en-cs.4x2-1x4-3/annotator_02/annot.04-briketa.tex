\documentclass[10pt]{article}

%\usepackage{times}
\usepackage{fancyhdr}

\pagestyle{fancy}
\fancyhf{}
\rhead{\textbf{Kód: briketa}}

\begin{document}

\subsection*{Zdroj:}



Procurement of public building and corruption prevention



Joint report on parallel audit submitted by the Czech Supreme Audit Ofice (NKÚ) and the German Bundesrechnungshof (BRH)



Introduction and overview

The procurement of goods and services by public-sector contracting agencies is of great importance both for the government and for the business community.
The government’s task is to use public resources as efficiently and economically as possible and to ensure fair and regulated competition.
In the European Union (EU), public contract awarding is key to the success of the single economic area.
In order to ensure a largely unrestricted competition in the field of crossborder procurement, awarding authorities also have to meet certain requirements imposed by the EU in awarding contracts where the contract value exceeds certain thresholds.
To this end, the EU Member States have to transpose EU procurement law into their national law and apply it.
This has already happened both in the Czech Republic and in Germany.
For contracts worth less than the EU thresholds, regulating public contract awarding procedures remains a matter for the individual countries.
Moreover, the contract award procedure is, as a rule, vulnerable to corruption.
In view of the generally high damage potential, awarding authorities are obliged to take reasonable action to fight and to rule out corruption and abusive favouritism in their contract award procedures (corruption prevention).
In January 2011, the Supreme Audit Office of the Czech Republic (Czech SAI – NKÚ) and the Bundesrechnungshof of Germany (German SAI – BRH) agreed to conduct parallel audits both of the EU-wide awarding of building contracts and of corruption prevention.
The audit focused on the application of EU procurement law as transposed into national law and corruption prevention of contracts for building construction and road construction and/or transport infrastructure.
The audit also covered contract awards below the EU thresholds with a view to corruption prevention.
The two SAIs’ audit findings are summarised in the joint report, which is also to be addressed to international institutions such as the International Organisation of Supreme Audit Institutions (INTOSAI).


Fields of audit and audited bodies

The assessments performed by the German SAI are for the most part based on horizontal audits of the above-mentioned topics, which we carried out at German building authorities responsible for federal road and building construction in the years 2011–2012.


\pagebreak

\subsection*{Překlad:}



Zadávání veřejných zakázek na budování a prevenci korupce



Společná zpráva o paralelním auditu předložená Českým nejvyšším kontrolním úřadem (NKÚ) a německým Bundesrechnungshof (BRH)



Úvod a přehled

Zadávání zboží a služeb veřejnými zadavateli má velký význam jak pro vládu, tak pro podnikatelskou sféru.
Úkolem vlády je využívat veřejné zdroje co nejefektivněji a hospodárně a zajišťovat spravedlivou a regulovanou hospodářskou soutěž.
V Evropské unii (EU) je zadávání veřejných zakázek klíčem k úspěchu jednotné hospodářské oblasti.
Aby byla zajištěna do značné míry neomezená hospodářská soutěž v oblasti přeshraničního zadávání zakázek, musí zadavatelé také splňovat určité požadavky, které EU ukládá při zadávání zakázek, u nichž hodnota zakázky přesahuje určité prahové hodnoty.
Za tímto účelem musí členské státy EU provést právo EU v oblasti zadávání veřejných zakázek do svého vnitrostátního práva a uplatňovat ho.
To se již stalo jak v České republice, tak v Německu.
U zakázek s nižší hodnotou, než je prahová hodnota EU, zůstává regulace postupů zadávání veřejných zakázek záležitostí jednotlivých zemí.
Postup při zadávání zakázek je navíc zpravidla náchylný ke korupci.
Vzhledem k obecně vysokému potenciálu škod jsou zadavatelé povinni přijmout přiměřená opatření k boji proti korupci a zneužívání ve svých postupech při zadávání zakázek (předcházení korupci) a k vyloučení tohoto zneužívání.
V lednu 2011 se Nejvyšší kontrolní úřad České republiky (NKÚ) a Bundesrechnungshof Německo (NKÚ) dohodly, že provedou souběžné audity jak zadávání zakázek na stavební práce v celé EU, tak předcházení korupci.
Audit se zaměřil na uplatňování právních předpisů EU o zadávání zakázek, jak byly provedeny do vnitrostátního práva, a na předcházení korupci u zakázek na stavební práce a výstavbu silnic a/nebo dopravní infrastrukturu.
Audit se rovněž zabýval zadáváním zakázek pod prahovými hodnotami EU s cílem předcházet korupci.
Obě auditní zjištění NKÚ jsou shrnuta ve společné zprávě, která je určena také mezinárodním institucím, jako je Mezinárodní organizace nejvyšších kontrolních institucí (INTOSAI).


Oblasti auditu a auditované subjekty

Posouzení provedená německým NKÚ jsou z větší části založena na horizontálních auditech výše uvedených témat, které jsme provedli u německých stavebních úřadů odpovědných za výstavbu federálních silnic a budov v letech 2011–2012.


\end{document}
