\documentclass[10pt]{article}

%\usepackage{times}
\usepackage{fancyhdr}

\pagestyle{fancy}
\fancyhf{}
\rhead{\textbf{Kód: pomyje}}

\begin{document}

\subsection*{Zdroj:}

On the basis of the request from the SAO, the BRH found that the audit performed by the German tax office at the acquirer in Germany had not yet been completed.
The German acquirer declared the acquisition of goods at a value of EUR 16,335 twice, in the tax returns for the months of August and September 2004.
10.
In 2005, through a SCAC form, the Czech tax administrator requested review of supplies of automobiles to a German taxpayer at a value of EUR 198,167, performed in the 4th quarter of 2004 and 1st quarter of 2005, showing the characteristics of carousel fraud.
The German tax administration sent a partial answer to the RFI to the CR, in which it confirmed that the German taxpayer acquired several automobiles and further stated that the investigation in Germany had not yet been completed.
The SAO found that the competent local tax office performed several local investigations at the Czech supplier for the purpose of reviewing the claim of excessive VAT deductions, in which it did not find unauthorised claiming.
The BRH found that in this period the German taxpayer declared in his VAT returns the acquisition of goods from other Member States with a total value of EUR 579,114 and that the audit performed by the German tax office at the acquirer in Germany has still not yet been completed.
In chapter 4.4.1 the reasons of mismatches between the declaration of intra-Community acquisitions in the recipient’s VAT return and the data of the RS of the supplier are described.


The audited cases showed that temporary mismatches or the improper declaration of a transaction as an intra-Community supply of goods in RS were not the most frequent causes for the deviations:

11.
According to data from the VIES system, in the 1st quarter of 2005, the Czech taxpayer allegedly acquired goods at a value of EUR 35,716 from the German taxpayer; however, the Czech taxpayer did not declare this acquisition.
Since the Czech taxpayer failed to cooperate with the tax administrator and no activities were performed in the registered seat of the company, the Czech tax administrator requested, as part of international exchange of information, review of the way in which the German taxpayer communicated with the Czech taxpayer.


\pagebreak

\subsection*{Překlad:}

Na základě žádosti NKÚ BK zjistil, že kontrola provedená německým daňovým úřadem u nabyvatele v Německu nebyla dosud dokončena.
Německý nabyvatel vykázal v daňových přiznáních za měsíce srpen a září 2004 pořízení zboží v hodnotě 16 335 EUR dvakrát.
10.
V roce 2005 požádal český daňový správce prostřednictvím formuláře SCAC o přezkum dodávek automobilů německému daňovému poplatníkovi v hodnotě 198 167 EUR, který byl proveden ve čtvrtém čtvrtletí roku 2004 a v prvním čtvrtletí roku 2005 a který vykazoval znaky kolotočového podvodu.
Německá daňová správa zaslala RFI částečnou odpověď CR, v níž potvrdila, že německý daňový poplatník získal několik automobilů, a dále uvedla, že šetření v Německu nebylo dosud ukončeno.
NKÚ zjistil, že příslušný místní daňový úřad provedl u českého dodavatele několik místních šetření za účelem prověření nároku na nadměrné odpočty DPH, u něhož nezjistil neoprávněné uplatnění nároku.
BRH zjistil, že v tomto období německý daňový poplatník přiznal ve svém přiznání k DPH pořízení zboží z jiných členských států v celkové hodnotě 579,114 EUR a že kontrola provedená německým daňovým úřadem u nabyvatele v Německu dosud nebyla dokončena.
V kapitole 4.4.1 jsou popsány důvody nesouladu mezi prohlášením o pořízení zboží uvnitř Společenství v přiznání k DPH příjemce a údaji o daňovém přiznání dodavatele.


Auditované případy ukázaly, že dočasné neshody nebo nesprávné prohlášení o transakci jako dodání zboží v RS uvnitř Společenství nebyly nejčastějšími příčinami odchylek:

11.
Podle údajů ze systému VIES nabyl český daňový poplatník v 1. čtvrtletí roku 2005 údajně od německého daňového poplatníka zboží v hodnotě 35 716 EUR; český daňový poplatník však tuto akvizici neuvedl.
Vzhledem k tomu, že český daňový poplatník nespolupracoval s daňovým správcem a v sídle společnosti nebyla vykonávána žádná činnost, požádal český daňový správce v rámci mezinárodní výměny informací o přezkum způsobu, jakým německý daňový poplatník komunikoval s českým daňovým poplatníkem.


\end{document}
