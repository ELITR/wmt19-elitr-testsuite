\documentclass[10pt]{article}

%\usepackage{times}
\usepackage{fancyhdr}

\pagestyle{fancy}
\fancyhf{}
\rhead{\textbf{Kód: rozpor}}

\begin{document}

\subsection*{Zdroj:}

- Another taxpaying entity (a limited liability company) had declared an accepted taxable goods valued at SK 77,167,576 and exports of goods valued at SK 18,006,041, and claimed an excessive VAT refund of SK 17,157,101.-.
The 2003 VAT Return Forms indicate that the difference between the value of the purchased goods and the value of the exported goods, i.e. SK 59,161,535.-, which had been created in January 2003, was not made up for throughout the year.
In the cases described above, the Slovak Supreme Audit Office requested Tax Offices of jurisdiction to carry out a tax audit aimed at the excessive VAT refund claims.
- In yet another case, a taxpaying entity (a limited liability company) had stated exports of goods valued at SK 87,199.- in its January to June 2002 VAT Return Forms.
A check in the Slovak Customs Database showed that the total exports of the company during the period in question had amounted to SK 296,747.-.
The Tax Offices of jurisdiction carried out multiple local investigations in the seat of the company, establishing that its executive secretary (a British national) had left and that the company had vacated the leased premises without terminating the Lease Contract in August 2002.
During the audit conducted by the Slovak Supreme Audit Office, the tax administrator notified authorities involved in criminal investigations of a suspicion of a criminal act of tax evasion.


Based on results of the audit at selected Tax Offices, the Slovak Supreme Audit Office concludes that major problems in the exercise of administration and control of VAT and ExT occur predominantly in the following areas:

- The possibility to carry out a tax audit of problematic taxpayers, particularly those claiming excessive VAT refunds, has not been used often enough.


- Tax audit protocols did not contain a list of audited documents

- export and import SADs.


- Inadequate verification of export and import transactions in tax return forms,

- A lengthy decision-making process of tax administrators, in particular with respect to tax assessment proceedings, including forfeiture of the right to impose sanctions.
- Failure to comply with legal requirements and conditions when granting tax payment deferments.
- Failure to set off outstanding VAT payments against VAT refunds within the prescribed period of time.


\pagebreak

\subsection*{Překlad:}

- Jiný subjekt povinný k dani (společnost s ručením omezeným) přiznal přijaté zboží podléhající dani v hodnotě 77,167,576 SK a vývoz zboží v hodnotě 18,006,041 SK a požadoval nadměrné vrácení DPH ve výši 17,157,101 SK.
Formuláře pro vrácení DPH za rok 2003 uvádějí, že rozdíl mezi hodnotou nakoupeného zboží a hodnotou vyváženého zboží, tj. SK 59,161,535.-, který byl vytvořen v lednu 2003, nebyl vyrovnán po celý rok.
V případech popsaných výše požádal Nejvyšší kontrolní úřad Slovenské republiky daňové úřady o provedení daňové kontroly zaměřené na nadměrné nároky na vrácení DPH.
- V dalším případě daňový subjekt (společnost s ručením omezeným) uvedl ve svých formulářích pro vrácení DPH za leden až červen 2002 vývoz zboží v hodnotě 87,199 SK.
Kontrola ve slovenské celní databázi prokázala, že celkový vývoz společnosti v dotyčném období činil 296 747 SK.
Daňové úřady v sídle společnosti provedly několik místních šetření, v nichž prokázaly, že její výkonný tajemník (britský státní příslušník) odešel a že společnost opustila pronajaté prostory, aniž by v srpnu 2002 vypověděla smlouvu o pronájmu.
Během auditu prováděného slovenským nejvyšším kontrolním úřadem oznámil správce daně orgánům zapojeným do trestního vyšetřování podezření z trestného činu daňového úniku.


Na základě výsledků auditu ve vybraných daňových úřadech dospěl Nejvyšší kontrolní úřad Slovenské republiky k závěru, že hlavní problémy při výkonu správy a kontroly DPH a DPH a DPH se vyskytují převážně v těchto oblastech:

- Možnost provést daňovou kontrolu problematických daňových poplatníků, zejména těch, kteří požadují nadměrné vrácení DPH, nebyla dostatečně často využita.


- Protokoly daňového auditu neobsahovaly seznam auditovaných dokumentů

- exportovat a importovat SAD.


- Nedostatečné ověřování vývozních a dovozních transakcí ve formulářích daňového přiznání,

- zdlouhavý rozhodovací proces správců daně, zejména pokud jde o řízení o daňovém výměru, včetně pozbytí práva ukládat sankce.
- Nedodržení právních požadavků a podmínek při poskytování odkladů platby daně.
- nezapočítání nezaplacených plateb DPH proti vrácení DPH ve stanovené lhůtě.


\end{document}
