\documentclass[10pt]{article}

%\usepackage{times}
\usepackage{fancyhdr}

\pagestyle{fancy}
\fancyhf{}
\rhead{\textbf{Kód: vojsko}}

\begin{document}

\subsection*{Zdroj:}

Restrictions on package volume, material of consumer packaged goods, and on the number of packages with a damaged excise stamp at the point of final sale or dispatch to consumers.
Consumer packaging means a container or other package holding a maximum of one litre or, if a container is made of glass, three litres.
Ethyl alcohol in bigger packaging is considered as being unmarked and may not be handled.
At the point of final sale to individuals for consumption, there can only be one consumer packaged item with a damaged excise stamp or, if the packaging is equipped with a one-way valve, three consumer packaged items with a damaged excise stamp.
Mandatory registration of persons who place consumer packaged alcohol with at least 15\% ethyl alcohol per volume into free circulation and who are ethyl alcohol importers or producers in the Czech Republic or operators of warehouses under duty suspension, provided personal and technical conditions are met.
The personal conditions of credibility, integrity, and no record of being banned from doing business have to be met by the applicant and the person acting in the capacity of the statutory body or a responsible person.
Depending on the number of alcohol excise stamps ordered, the applicant has to pay a guarantee of CZK 100,000-5,000,000 (i.e., approx. EUR 3,700-185,000).
The technical conditions for registration include suitability of the alcohol labelling site and a camera security system that is located in the labelling facilities or warehouse, operates continuously and allows remote access to the duty administrator.
Registration markings of the excise stamp are used for labelling alcohol with a volume greater than 0.06 litres.
During movement of the alcohol excise stamps, the registration markings of the alcohol excise stamps must be indicated in the transport documents or, in the case of export, in the customs declaration.
Until 30 November 2013, alcohol excise stamps contained a 10-digit registration code that included a 4-digit offtake registration number, excise stamp application code, and ethanol concentration level classification code.
Effective of 1 December 2013, excise stamps contain two types of identifiers and the amount of ethyl alcohol in the consumer packaged alcohol expressed in millilitres of ethanol at 20°C, rounded down to whole millilitres.


\pagebreak

\subsection*{Překlad:}

Omezení týkající se objemu balení, materiálu spotřebitelského baleného zboží a počtu balení s poškozeným razítkem spotřební daně v místě konečného prodeje nebo expedice spotřebitelům.
Spotřebitelským balením se rozumí nádoba nebo jiné balení obsahující nejvýše jeden litr nebo, pokud je nádoba ze skla, tři litry.
Ethylalkohol ve větším balení se považuje za neoznačený a nesmí se s ním manipulovat.
V místě konečného prodeje fyzickým osobám ke spotřebě může být pouze jedna spotřebitelská balená věc s poškozeným razítkem spotřební daně nebo, pokud je balení vybaveno jednosměrným ventilem, tři spotřebitelské balené věci s poškozeným razítkem spotřební daně.
Povinná registrace osob, které uvádějí do volného oběhu balený alkohol s nejméně 15 \% objemových ethylalkoholu a které jsou dovozci nebo výrobci ethylalkoholu v České republice nebo provozovateli skladů s podmíněným osvobozením od cla, pokud jsou splněny osobní a technické podmínky.
Osobní podmínky důvěryhodnosti, bezúhonnosti a zákazu podnikání musí splňovat žadatel a osoba jednající z titulu statutárního orgánu nebo odpovědné osoby.
V závislosti na počtu objednaných kolků na spotřební daň z alkoholu musí žadatel zaplatit záruku ve výši 100 000 až 5 000 000 Kč (tj. přibližně 3 700 až 185 000 EUR).
Technické podmínky pro registraci zahrnují vhodnost místa pro označování alkoholu a kamerový bezpečnostní systém, který je umístěn v prostorách pro označování nebo skladu, pracuje nepřetržitě a umožňuje vzdálený přístup správci daně.
Registrační označení kolku pro spotřební daň se používá pro označování alkoholu o objemu větším než 0,06 litru.
Během přepravy kolků spotřební daně z alkoholu musí být registrační značky kolků spotřební daně z alkoholu uvedeny v přepravních dokladech nebo, v případě vývozu, v celním prohlášení.
Do 30. listopadu 2013 obsahovaly kolky spotřební daně z alkoholu desetimístný registrační kód, který obsahoval čtyřmístné registrační číslo offtake, kód žádosti o vydání kolku spotřební daně a klasifikační kód úrovně koncentrace ethanolu.
Od 1. prosince 2013 obsahují kolky pro spotřební daně dva druhy identifikátorů a množství ethylalkoholu ve spotřebitelském baleném alkoholu vyjádřené v mililitrech ethanolu při 20 °C, zaokrouhlené dolů na celé mililitry.


\end{document}
