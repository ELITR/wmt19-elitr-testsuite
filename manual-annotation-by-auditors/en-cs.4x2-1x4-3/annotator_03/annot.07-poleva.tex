\documentclass[10pt]{article}

%\usepackage{times}
\usepackage{fancyhdr}

\pagestyle{fancy}
\fancyhf{}
\rhead{\textbf{Kód: poleva}}

\begin{document}

\subsection*{Zdroj:}

Cases of risky intra-Community transactions carried out by taxpayers from EU presented in EUROCANET were selected for the audit (see chapter 4.3).
The objectives of improved administrative cooperation and enhanced effectiveness of combating VAT fraud should be achieved also through establishing the EUROFISC network, meant to assist improving cooperation of the responsible authorities to a level enabling disclosure of fraud already at an early stage.
Later on, the network should become a resource for assessing risk exposures in intra-Community transactions.
ECOFIN approved establishment of the EUROFISC network already on 7 October 20089 and stipulated the key principles to govern the network at the same time.
A model for EUROFISC can be seen in EUROCANET, the system already operative for the swift and selective exchange of information developed by the Belgian tax authorities in cooperation with the other Member States and with support from the European Commission and OLAF.
The objective of the system is to detect risky business transactions and entities involved in organised VAT fraud.


4.1.2 Evaluation of the recommendations in Germany



The evaluation of the recommendations of the first report resulted in the following findings:



Recommendation 1



Conditions for registration of taxpayers should be harmonised within the EU

The tax administration in Germany established a well functioning system for the registration of taxpayers that is based on a detailed questionnaire.
The Federal Ministry of Finance brought this system to the attention of the Commission in October 2007.
The system itself is being improved continuously.
The data of the questionnaire are now being recorded electronically and will subsequently be available for research so that the data are part of risk management during registration.
The Commission made a proposal to amend Regulation 1798/2003. This proposal included common standards for the registration and deregistration of taxpayers.
The Member States needed further explanation and detailed elaboration of these issues.
The German SAI requested the Federal Ministry of Finance to make sure that Germany will not go below the standards achieved.


Recommendation 2



Monthly submission of recapitulative statements

On April 8, 2010 a law was adopted for the implementation of EU directives into national law.
The law includes provisions regarding the monthly submission of recapitulative statements for intra-Community supplies.
The provisions take effect as of 1 July 2010.


Recommendation 3



\pagebreak

\subsection*{Překlad:}

Pro audit byly vybrány případy riskantních transakcí uvnitř Společenství prováděných daňovými poplatníky z EU, které byly uvedeny v programu EUROCANET (viz kapitola 4.3).
Cílů zlepšené správní spolupráce a zvýšené účinnosti boje proti podvodům v oblasti DPH by mělo být dosaženo rovněž zřízením sítě EUROFISC, která by měla pomoci zlepšit spolupráci odpovědných orgánů na úrovni umožňující zveřejňování podvodů již v rané fázi.
Později by se tato síť měla stát zdrojem pro posuzování rizikových expozic při transakcích uvnitř Společenství.
Rada ECOFIN schválila zřízení sítě EUROFISC již dne 7. října 20089 a stanovila hlavní zásady pro řízení této sítě ve stejnou dobu.
Model pro EUROFISC lze vidět v systému EUROCANET, který již funguje pro rychlou a selektivní výměnu informací vyvinutou belgickými daňovými orgány ve spolupráci s ostatními členskými státy a s podporou Evropské komise a úřadu OLAF.
Cílem systému je odhalovat rizikové obchodní transakce a subjekty zapojené do organizovaných podvodů v oblasti DPH.


4.1.2 Hodnocení doporučení v Německu



Hodnocení doporučení obsažených v první zprávě vedlo k následujícím zjištěním:



Doporučení 1



Podmínky pro registraci daňových poplatníků by měly být v rámci EU harmonizovány

Daňová správa v Německu zavedla dobře fungující systém registrace daňových poplatníků, který je založen na podrobném dotazníku.
Spolkové ministerstvo financí na tento systém upozornilo Komisi v říjnu 2007.
Samotný systém je neustále zdokonalován.
Údaje z dotazníku se nyní zaznamenávají elektronicky a následně budou k dispozici pro výzkum, aby byly údaje během registrace součástí řízení rizik.
Komise předložila návrh na změnu nařízení č. 1798/2003. Tento návrh obsahoval společné normy pro registraci a zrušení registrace daňových poplatníků.
Členské státy potřebovaly další vysvětlení a podrobné rozpracování těchto otázek.
Německý úřad pro kontrolu potravin požádal spolkové ministerstvo financí, aby se ujistilo, že Německo nepřekročí dosažené standardy.


Doporučení 2



Měsíční předkládání souhrnných hlášení

Dne 8. dubna 2010 byl přijat zákon o provádění směrnic EU do vnitrostátního práva.
Zákon obsahuje ustanovení týkající se měsíčního předkládání souhrnných hlášení pro dodávky uvnitř Společenství.
Ustanovení nabývají účinku dnem 1. července 2010.


Doporučení 3



\end{document}
