\documentclass[10pt]{article}

%\usepackage{times}
\usepackage{fancyhdr}

\pagestyle{fancy}
\fancyhf{}
\rhead{\textbf{Kód: kohout}}

\begin{document}

\subsection*{Zdroj:}

Because of some errors in entering data into the customs database, it was not possible to identify some registered import transactions, e.g. when the importer's name was missing or if there was a mistake in the Tax Identification Number (hereinafter „Tax ID").
Consequently, it is not possible to say with certainty that the import transactions had not actually taken place.
- For 4 export SADs from the Czech Customs Database, the aggregate statistical value of which was CZK 399,776., no matching imports were found and proven (see Paragraph 5 above).
In this case, there exists a suspicion of a fictitious export of goods for the purpose of placing an unjustifiable and illegal VAT refund claim.


5. Audit of Exports and Imports of Goods and Services at Selected Tax offices



5.1. Audit Sample



Folders of taxpaying entities were selected at Tax Offices being audited for the purpose of verifying whether taxpaying entities comply with their obligations with respect to exports and imports of goods and services, and assessing the cooperation between financial and customs authorities. The selection was based on the following criteria:



1. The criteria used to select taxpayers from the Czech and Slovak Customs Databases were as follows: a) Highest volume of imports, b) Highest assessed VAT, c) Highest volume of exports,



2. The criteria used to select taxpayers from records of Tax Offices were as follows: a) Amount of VAT refund claims related to imports, b) Excessive VAT refund claims, c) ExT refund claims, d) VAT exemption on exports,

3. Taxpaying entities that had undergone a tax audit.
The sample selection did not take into account the country of destination and was focused on risky commodities.


5.2. Audit Findings in the Czech Republic



A list of imports and exports of risky commodities to and from the Czech Republic in 2002 and 2003 is shown in the following table:



5.2.1. Audit of the Selected Sample at Tax Offices



In its audit of Tax Offices, the Czech Supreme Audit Office focused particularly on the following issues:



- Procedures employed by Tax Offices to check VAT values shown in VAT Return Forms,



- Cooperation between Tax Offices and customs authorities,



\pagebreak

\subsection*{Překlad:}

Z důvodu některých chyb při vkládání údajů do celní databáze nebylo možné identifikovat některé zapsané dovozní transakce, např. když chybělo jméno dovozce nebo když byla chyba v daňovém identifikačním čísle (dále jen „daňové ID“).
Proto nelze s jistotou říci, že se dovozní transakce ve skutečnosti neuskutečnily.
– U čtyř vývozních JSD z české celní databáze, jejichž souhrnná statistická hodnota činila 399 776 CZK, nebyly nalezeny a prokázány žádné odpovídající dovozy (viz odstavec 5 výše).
V tomto případě existuje podezření na fiktivní vývoz zboží za účelem neoprávněného a protiprávního uplatnění nároku na vrácení DPH.


5. Audit vývozu a dovozu zboží a služeb na vybraných finančních úřadech



5.1 Vzorek auditu

Na finančních úřadech, které byly kontrolovány za účelem ověření, zda subjekty povinné k dani plní své povinnosti, pokud jde o vývoz a dovoz zboží a služeb, a posouzení spolupráce mezi finančními a celními orgány byly vybrány složky subjektů povinných k dani, které byly vybrány v rámci auditu.


Kritéria použitá pro výběr daňových poplatníků z české a slovenské celní databáze byla následující: a) nejvyšší objem dovozu, b) nejvyšší posouzená DPH, c) nejvyšší objem vývozu,



2. Kritéria použitá pro výběr daňových poplatníků ze záznamů daňových úřadů byla následující: a) výše nároků na vrácení DPH v souvislosti s dovozem, b) nadměrné nároky na vrácení DPH, c) nároky na vrácení DPH, d) osvobození od DPH u vývozů,

3. subjekty, které platí daně a které prošly daňovou kontrolou.
Výběr vzorku nezohlednil zemi určení a zaměřil se na rizikové komodity.


5.2 Zjištění auditu v České republice



Seznam dovozů a vývozů rizikových komodit do České republiky a z České republiky v letech 2002 a 2003 je uveden v následující tabulce:



5.2.1 Audit vybraného vzorku u daňových úřadů



Při auditu daňových úřadů se Nejvyšší kontrolní úřad České republiky zaměřil zejména na následující otázky:



– postupy používané daňovými úřady ke kontrole hodnot DPH uvedených ve formulářích pro vrácení DPH,



– spolupráce mezi daňovými úřady a celními úřady,



\end{document}
