\documentclass[10pt]{article}

%\usepackage{times}
\usepackage{fancyhdr}

\pagestyle{fancy}
\fancyhf{}
\rhead{\textbf{Kód: krajan}}

\begin{document}

\subsection*{Zdroj:}

Inspections carried out by the tax offices of the respective countries had not been finished at the conclusion of our previous joint audit.
The follow-up audit verified that most of the tax proceedings relating to the selected cases of business transactions between the taxpayers into/from the Czech Republic were concluded and adequate measures were taken by the tax administrations in the Czech Republic or in Germany (i. a. additional VAT assessments or prevention of reimbursement of input tax).


3. Audit of selected cases of high-risk intra-Community transactions

We selected high-risk cases of intra-Community transactions made by EU taxpayers presented in EUROCANET.
The EUROCANET system is a tool for the detection of high-risk taxpayers within EU.
The Member States can participate in the system on a facultative basis.
The Czech Republic is an active participant of EUROCANET, whereas Germany has a passive status.
From the point of view of an active participant, it is unsatisfactory that other Member States are opposed to active participation.
Given the fact that the German tax administration took no part in data exchange in the EUROCANET system, no information could be obtained what suppliers with German VAT ID were assessed as high-risk cases.
The Czech SAI found that cases including normal business transactions were identified during the audit as well as cases of suspected involvement in carousel trading and fraudulent activities related to VAT.
The tax administrators have limited possibilities for inspecting high-risk transactions, concerning in particular “missing traders”, while a physical review of the documents alone can hardly serve to prove any fraudulent activity to the tax entity.
The same applies to tax administrators, when import from and export to third countries are additionally combined with intra-Community carousel trading.
In these cases, it would be recommendable to provide for increased mutual cooperation between the customs and tax authorities aimed at identifying goods and irregularities with transportation documents as well as monitoring their physical movements within the EU in direction to the final consumer.
Germany as well as a number of other countries were reluctant to actively join the EUROCANET system due to the nonexistence of a sufficient legal basis.
In Germany, other tools for the detection of risk taxpayers are used that have proven to be suitable for the German tax administration so far.


\pagebreak

\subsection*{Překlad:}

Kontroly prováděné daňovými úřady příslušných zemí nebyly ukončeny na závěr naší předchozí společné kontroly.
Následná kontrola ověřila, že většina daňových řízení týkajících se vybraných případů obchodních transakcí mezi daňovými poplatníky do/z České republiky byla uzavřena a daňová správa v České republice nebo v Německu přijala odpovídající opatření (např. dodatečné vyměření DPH nebo zabránění vrácení daně na vstupu).


3. Audit vybraných případů vysoce rizikových transakcí uvnitř Společenství

Vybrali jsme vysoce rizikové případy transakcí uvnitř Společenství uskutečněných daňovými poplatníky EU prezentované v systému EUROCANET.
Systém EUROCANET je nástrojem pro odhalování vysoce rizikových daňových poplatníků v rámci EU.
Členské státy se mohou systému účastnit na základě fakultace.
Česká republika je aktivním účastníkem systému EUROCANET, zatímco Německo má pasivní status.
Z hlediska aktivního účastníka je neuspokojivé, že ostatní členské státy jsou proti aktivní účasti.
Vzhledem k tomu, že se německá daňová správa na výměně údajů v systému EUROCANET nepodílela, nebylo možné získat žádné informace o tom, kteří dodavatelé s německým průkazem totožnosti pro účely DPH byli vyhodnoceni jako vysoce rizikové případy.
Český kontrolní úřad zjistil, že při auditu byly zjištěny případy včetně běžných obchodních transakcí, jakož i případy podezření z účasti na kolotočových obchodech a podvodných činnostech souvisejících s DPH.
Správci daně mají omezené možnosti kontroly vysoce rizikových transakcí týkajících se zejména „chybějících obchodníků“, zatímco samotný fyzický přezkum dokladů může jen stěží posloužit k prokázání jakékoli podvodné činnosti daňovému subjektu.
Totéž platí pro správce daně, pokud je dovoz z třetích zemí a vývoz do třetích zemí navíc kombinován s kolotočovým obchodováním uvnitř Společenství.
V těchto případech by bylo vhodné zajistit zvýšenou vzájemnou spolupráci mezi celními a daňovými orgány s cílem identifikovat zboží a nesrovnalosti s přepravními doklady a sledovat jejich fyzický pohyb v rámci EU směrem ke konečnému spotřebiteli.
Německo a řada dalších zemí se zdráhaly aktivně se připojit k systému EUROCANET z důvodu neexistence dostatečného právního základu.
V Německu se používají jiné nástroje pro odhalování rizikových daňových poplatníků, které se dosud ukázaly jako vhodné pro německou daňovou správu.


\end{document}
