\documentclass[10pt]{article}

%\usepackage{times}
\usepackage{fancyhdr}

\pagestyle{fancy}
\fancyhf{}
\rhead{\textbf{Kód: bolest}}

\begin{document}

\subsection*{Zdroj:}

At the end of 2014, national legislation introduced a new excise stamp information system.
The objective of this new system was to facilitate and computerise the system for ordering, assessing, printing, distributing, and using excise stamps intended for labelling consumer packaged tobacco, cigars, cigarillos, and cigarettes.
Excise stamp users who wish to take off excise stamps according to the “new system” have to file a request with the customs office to have excise stamps issued electronically.
Excise stamp users have to use the electronic excise stamp system not only to request that excise stamps be issued electronically, but also to report data required under Slovak national legislation and to keep a record of excise stamps.
Excise stamps may only be printed by the state enterprise Mincovňa Kremnica, štátny podnik (the Kremnica Mint).
The FD SR performs fiscal supervision over the handing of excise stamps and their destruction at the printing house.
The FD SR is obliged to arrange for the presence of one of its employee at the printing house during handling of the excise stamps.
The FD SR arranges the sale of excise stamps to excise stamp users at the printing house.
It is possible to take off excise stamps only from the Financial Directorate employee at the printing house.
Tax warehouse keepers or tobacco product importers can affix excise stamps to consumer packaged tobacco.
Tax warehouse keepers, tobacco product importers, or authorised consignees can affix excise stamps to consumer packaged cigars or cigarillos.
Only an excise stamp with the symbol indicating the valid excise duty rate can be used to label consumer packaged tobacco, cigars, and cigarillos.
Labelling of consumer packaged cigarettes with excise stamps may only be carried out by the company that manufactured the cigarettes.
No fewer than 500 excise stamps, or a multiple thereof, may be ordered.
The customs office will notify the excise stamp user electronically about the number of excise stamps that he may collect, the price of the excise stamps, and the bank account number of the FD SR no later than within three working days.
The customs office may reduce the number of ordered excise stamps if circumstances arise based on which it can be reasonable assumed that a breach of legal regulations might occur.


\pagebreak

\subsection*{Překlad:}

Na konci roku 2014 zavedly vnitrostátní právní předpisy nový informační systém Briefmarken pro spotřební daně.
Cílem tohoto nového systému bylo usnadnit a EDVovat systém objednávání, posuzování, tisku, distribuce a používání spotřebních známek určených pro označování tabákových výrobků, doutníků, doutníčků a cigaret pro spotřebitele.
Uživatelé známek pro spotřební daně, kteří si přejí odstranit známky pro spotřební daně podle „nového systému“, musí podat u celního úřadu žádost o vydání kol pro spotřební daně v elektronické podobě.
Uživatelé známek pro spotřební daně musí používat elektronický systém kol pro spotřební daně nejen proto, aby požadovali elektronické vydání kol pro spotřební daně, ale také proto, aby hlásili údaje požadované slovenskými vnitrostátními právními předpisy a vedli záznamy o kolkách pro spotřební daně.
Známky spotřební daně mohou tisknout pouze státní podnik Mincovňa Kremnica, štalní podnik (Mincovna Kremnica).
FD SR vykonává fiskální dohled nad předáním spotřebních známek a jejich likvidací v tiskařském domě.
FD SR je povinna zajistit přítomnost jednoho ze svých zaměstnanců v tiskařském podniku při manipulaci se známkami spotřební daně.
FD SR zajišťuje prodej kol pro spotřební daně uživatelům kol pro spotřební daně v tiskařském podniku.
Je možné odebrat známky spotřební daně pouze od zaměstnance finančního ředitelství v tiskařském podniku.
Správci daňových skladů nebo dovozci tabákových výrobků mohou na spotřebitele zabalený tabák připojit známky spotřební daně.
Správci daňových skladů, dovozci tabákových výrobků nebo pověření příjemci mohou na spotřebitelích balené doutníky nebo doutníčky připojit známky spotřební daně.
K označování zabaleného tabáku pro spotřebitele, doutníků a doutníčků lze použít pouze známku spotřební daně s symbolem označující platnou sazbu spotřební daně.
Označování cigaret zabalených do spotřebitelských obalů se známkami spotřební daně může provádět pouze společnost, která cigarety vyrábí.
Nesmí být objednáno méně než 500 známek spotřební daně nebo několik z nich.
Celní úřad oznámí uživateli razítka z spotřební daně elektronickou cestou počet kol, které si může vybrat, cenu kol pro spotřební daně a číslo bankovního účtu FD SR nejpozději do tří pracovních dnů.
Celní úřad může snížit počet objednaných kol pro spotřební daně, pokud nastanou okolnosti, na jejichž základě lze důvodně předpokládat, že by mohlo dojít k porušení právních předpisů.


\end{document}
